\documentclass{article}

\usepackage{amsmath,amsfonts,amssymb,amsthm}
\usepackage{enumerate}
\usepackage{arydshln}
\usepackage{listings,color}
\usepackage{graphicx}

\definecolor{dkgreen}{rgb}{0,0.6,0}
\definecolor{gray}{rgb}{0.5,0.5,0.5}
\definecolor{mauve}{rgb}{0.58,0,0.82}

% Opening
\title{Numerical Analysis HW7\\
Ch10 - 1,3,4,13 (pg267)\\
Ch11 - 1,6,8 (pg280)\\}
\author{Neal D. Nesbitt}

\begin{document}
\maketitle

\theoremstyle{definition}
\newtheorem{problem}{Problem}
\newtheorem{solution}{Solution}[problem]

\lstset{basicstyle=\ttfamily,
		language=Matlab,
		keywordstyle=\color{blue},
		commentstyle=\color{dkgreen},
		stringstyle=\color{mauve},
		identifierstyle=\bf
		}

\begin{problem}
Determine the number of flops as a function of the number of equations $n$ for
\begin{enumerate}[(a)]
\item factorization
\item forward substitution
\item back substitution
\end{enumerate}
of LU factorization
\end{problem}

\begin{solution}
\begin{lstlisting}

%--------------------------------------------------------------------------
% Forward Elimination:

% for each pivot column
for k = 1:(n-1)
    
    %**********************************************************************
    % Partial Pivoting:
    
    % find the maximum leading term in the current pivot column
    [~,i] = max(abs(A(k:n,k)));
    % if it is not the current row
    if i ~= 1
        % switch the current row and the row with the largest leading term 
        A([k,i+k-1],:)=A([i+k-1,k],:);
        P([k,i+k-1],:)=P([i+k-1,k],:);
        
        display([P,A]);   % display the result of the pivot
    end    
    %**********************************************************************
    
    % for each row past the diagonal 
    for i = k+1:n
        
        % eliminate the leading term using row operations:
        
        % find the factor that changes the column's leading term
        % to match the leading term of the current elimination row
        factor = A(i,k)/A(k,k);
        
        % set the leading term (that normally is eliminated)
        % instead to its corresponding factor to store until needed
        A(i,k)=factor;
        
        % multiply the factor through the pivot row
        % and subtract the result from the current elimination row
        A(i,(k+1):n) = A(i,(k+1):n) - factor*A(k,(k+1):n);
    end
    % (n-k) cycles in i,
    %   1 flop per for finding elimination row factor, 
    %   2(n-k) flops per with:
    %       n-k for multiplying factor through pivot row and
    %       n-k for subtracting the resulting row from the elimination row
    % Total Flops:(n-k)[2(n-k)+1]
    
    display(A);   % display after each pivot column elimination
end
% (n-1) cycles in k
% Total Flops: sum_{k=1}^{n-1} (n-k)[2(n-k)+1] = 
% = n(n-1)(4n-1)/6 = (2/3)n^3 -(5/6)n^2 +(1/6)n


%--------------------------------------------------------------------------
% Seperation:

% set up the place to store L
L = eye(n);

% copy each column of A below the diagonal into L
% and set the value in A to zero
for i = 1:n-1
    % copy each column of A below the diagonal into L
    L((i+1):n,i) = A((i+1):n,i);
    A((i+1):n,i) = 0; % and set original the value in A to zero
end

U = A;
\end{lstlisting}

So for Forward Elimination, or factoring an input matrix $A_{m,n}$ into the lower triangular factor matrix $L$, and upper triangular matrix $U$:\\
\\
For each pivot column there are $(n-1)$ cycles in $k$:\\
For each row elimination there are $(n-k)$ cycles in $i$, each with
\begin{itemize}
\item 1 flop per for finding elimination row factor, 
\item $2(n-k)$ flops per with:
\subitem $n-k$ for multiplying factor through pivot row and
\subitem $n-k$ for subtracting the resulting row from the elimination row
\end{itemize}
Giving a total of  $(n-k)[2(n-k)+1]$ flops per row of elimination.\\
\\
Then for the $(n-1)$ cycles in $k$, the total number of flops is:
\begin{align*}
&\sum_{k=1}^{n-1} (n-k)[2(n-k)+1]\\
&= \sum_{k=1}^{n-1} (2(n-k)^{2}+n-k) = \sum_{k=1}^{n-1} (2n^{2}-4nk+2k^{2}+n-k)\\
&= \sum_{k=1}^{n-1}(2n^{2}+n) +\sum_{k=1}^{n-1} 2k^{2} +\sum_{k=1}^{n-1} (-4nk-k)\\
&= (n-1)(2n^{2}+n) +2\sum_{k=1}^{n-1} k^{2} -(4n+1)\sum_{k=1}^{n-1} k\\
&= (n-1)(2n^{2}+n) +2\left(\frac{(n-1)n(2(n-1)+1)}{6}\right) -(4n+1)\left(\frac{(n-1)n}{2}\right)\\
&= (n-1)n\left[ (2n+1) + \frac{(2n-1)}{3} -\frac{(4n+1)}{2} \right]\\
&= (n-1)n\left[ (12n+6) +(4n-4) -(12n+3) \right] /6 \\
&= (n-1)n(4n-1)/6 = \boxed{\frac{2}{3}n^{3}-\frac{5}{6}n^{2}+\frac{1}{6}n} \\
\end{align*}
\end{solution}

\begin{solution}
\begin{lstlisting}
%==========================================================================
% Forward Substitution:

D = zeros(size(B));     % Set up our intermediary solution
D(1,:) = P(1,:)*B;      % Find the initial values to start substitution
% (Accounting for row changes in the decomposition)
% (nB)n flops for picking the correct row of B

for i=2:n
    D(i,:) = (P(i,:)*B)-(L(i,1:(i-1))*D(1:(i-1),:));
end
% (n-1) cycles in i,
%   (nb)n flops for picking the correct row of B
%   (nb)(i-1) flops for producing the variables to subtract
%   (nb) flops for subtracting the row to find the next iteration's values
% Total Flops: (n-1)((nB)n+(nB))+(nB)sum_{i=2}^{n}(i-1)
% = (nB)(n-1)((3/2)n+1) = (nB)((3/2)n^{2}-(1/2)n-1)

display(D);
\end{lstlisting}

For forward substitution we have $n$ flops per vector on the right hand side. If we set up the column vectors on the right hand side into a matrix: $B = [\mathbf{b}_{1} \cdots \mathbf{b}_{1}]$, and their respective solutions into the corresponding matrix $X = [\mathbf{x}_{1} \cdots \mathbf{x}_{1}]$
\end{solution}

\setcounter{problem}{2}
\begin{problem}
Use naive Gauss elimination to factor the following system according to the description in Section 10.2:
\begin{align*}
7x_{1} + 2x_{2} - 3x_{3} &= -12 \\
2x_{1} + 5x_{2} - 3x_{3} &= -20 \\
x_{1} - x_{2} - 6x_{3} &= -26
\end{align*}
Then, multiply the resulting $[L]$ and $[U]$ matrices to determine that $[A]$ is produced.
\end{problem}

\begin{problem}
Use LU factorization to solve the system of equations in Problem 3. Show all the steps of the computation. Also solve the system for the alternate right-vand-side vector
\[ b = (12,18,-6) \]
\end{problem}

\setcounter{problem}{12}
\begin{problem}
Use Cholesky factorization to determine $[U]$ so that
\[ [A] = [U]^{T}[U] =
\begin{bmatrix}
2	&	-1	&	0	\\
-1	&	2	&	-1	\\
0	&	-1	&	2
\end{bmatrix} \]
\end{problem}

\setcounter{problem}{0}
\begin{problem}
Determine the matrix inverse for the following system:
\begin{align*}
10x_{1}	+	2x_{2}	-	x_{3}	&=	27		\\
-3x_{1}	-	6x_{2}	+	2x_{3}	&=	-61.5	\\
x_{1}	+	x_{2}	+	5x_{3}	&=	-21.5
\end{align*}
Check your results by verifying $[A][A]^{-1} = [I]$
\end{problem}

\setcounter{problem}{5}
\begin{problem}
Determine $\left\| A \right\|_{f}$, $\left\| A \right\|_{1}$, and $\left\| A \right\|_{\infty}$ for
\[ [A] =
\begin{bmatrix}
8	&	2	&	-10	\\
-9	&	1	&	3	\\
15	&	-1	&	6
\end{bmatrix}
\]
Before determining the norms, scale the matrix by making the maximum element in each row equal to one.
\end{problem}

\setcounter{problem}{7}
\begin{problem}
Use MATLAB to determine the spectral condition number for the following system. Do not normalize the system:
\[
\begin{bmatrix}
1	&	4	&	9	&	16	&	25	\\
4	&	9	&	16	&	25	&	36	\\
9	&	16	&	25	&	36	&	49	\\
16	&	25	&	36	&	49	&	64	\\
25	&	36	&	49	&	64	&	81	
\end{bmatrix}
\]
Compute the condition number based on the row-sum norm.
\end{problem}

\end{document}