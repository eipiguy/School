\documentclass{article}

\usepackage{amsmath,amsfonts,amssymb,amsthm}
\usepackage{enumerate}
\usepackage{arydshln}
\usepackage{listings,color}
\usepackage{graphicx}

\definecolor{dkgreen}{rgb}{0,0.6,0}
\definecolor{gray}{rgb}{0.5,0.5,0.5}
\definecolor{mauve}{rgb}{0.58,0,0.82}

% Opening
\title{Numerical Analysis HW9\\
Ch19 - 2 (pg492)\\
Ch20 - 2 (pg517)\\}
\author{Neal D. Nesbitt}

\begin{document}
\maketitle

\theoremstyle{definition}
\newtheorem{problem}{Problem}[section]
\newtheorem{solution}{Solution}[problem]
\renewcommand*{\thesolution}{\theproblem.\alph{solution}}

\lstset{basicstyle=\ttfamily,
		language=Matlab,
		keywordstyle=\color{blue},
		commentstyle=\color{dkgreen},
		stringstyle=\color{mauve},
		identifierstyle=\bf
		}

\setcounter{section}{19}
\setcounter{problem}{1}
\begin{problem}
Evaluate:
\[ \int_{0}^{4} \left( 1 -e^{-x} \right) dx\]
\begin{enumerate}[(a)]
\item analytically
\item single application of the trapezoidal rule
\item composite trapezoidal rule with $n=2,4$
\item single application of Simpson's 1/3 rule
\item composite Simpson's 1/3 rule with $n=4$
\item Simpson's 3/8 rule
\item composite Simpson's rule, with $n=5$
\end{enumerate}
Compute the percent relative error for each.
\end{problem}

\begin{solution}Analytically:\\
\[ \int_{0}^{4} \left( 1 -e^{-x} \right) dx = \left[ x +e^{-x} \right]_{0}^{4} = 3 +e^{-4} \approx \boxed{3.01831}\]
\end{solution}

\begin{solution}Trapezoid:\\
\begin{align*}
\int_{0}^{4} (1-e^{-x})dx &\approx (b-a)\left[ (1-e^{-a}) + (1-e^{-b}) \right]/2\\
&= 4\left( 2 -(e^{-0} +e^{-4} \right)/2\\
&= 2\left( 1 -e^{-4} \right) \approx \boxed{1.96336}\\
\end{align*}
Where the relative error is 
\[ \left|\varepsilon_{r}\right| = \left| \frac{3.01831 - 1.96336}{3.01831} \right| = \boxed{34.951\%} \]
\end{solution}

\begin{solution}Composite Trapezoid $n=2,4$:\\
\[ h = (a-b)/n \]
Then for $k\in{1,\cdots,n}$
\[ x_{k} = a+kh = b-(n-k)h \]
Plugging in $a=0,b=4,n=2$ gives
\begin{align*}
\int_{0}^{4} (1-e^{-x})dx &\approx h\left[ (1-e^{-x_{0}}) + 2(1-e^{-x_{1}}) + (1-e^{-x_{2}})\right]/2\\
&= 2\left[ (1-e^{0}) + 2(1-e^{-2}) + (1-e^{-4})\right]/2\\
&= 4-(e^{0} +2e^{-2} +e^{-4})\\
&= 3-(2e^{-2} +e^{-4}) \approx \boxed{2.71101}
\end{align*}
Where the relative error is 
\[ \left|\varepsilon_{r}\right| = \left| \frac{3.01831 - 2.71101}{3.01831} \right| = \boxed{10.181\%} \]
While plugging in $n=4$ gives
\begin{align*}
\int_{0}^{4} (1-e^{-x})dx &\approx h\left[ (1-e^{-x_{0}}) + 2(1-e^{-x_{1}}) + 2(1-e^{-x_{2}}) + 2(1-e^{-x_{0}}) + (1-e^{-x_{2}})\right]/2\\
&= \left[ (1-e^{0}) + 2(1-e^{-1}) + 2(1-e^{-2}) + 2(1-e^{-3}) + (1-e^{-4})\right]/2\\
&= \left[ 7-e^{-1} -2e^{-2} -2e^{-3} -e^{-4} \right]/2 \approx \boxed{3.12178}
\end{align*}
Where the relative error is 
\[ \left|\varepsilon_{r}\right| = \left| \frac{3.01831 - 3.12178}{3.01831} \right| = \boxed{3.428\%} \]
\end{solution}

\begin{solution}Simpson 1/3:\\
\begin{align*}
\int_{0}^{4} (1-e^{-x})dx &\approx h\left[ (1-e^{-x_{0}}) + 4(1-e^{-x_{1}}) + (1-e^{-x_{2}})\right]/3\\
&= \frac{2}{3}\left[ (1-e^{0}) + 4(1-e^{-2}) + (1-e^{-4}) \right]\\
&= \frac{2}{3}\left[ 6-(e^{0} +4e^{-2} +e^{-4}) \right]\\
&= \frac{2}{3}\left[ 5-(4e^{-2} +e^{-4}) \right] \approx \boxed{2.96023}
\end{align*}
Where the relative error is 
\[ \left|\varepsilon_{r}\right| = \left| \frac{3.01831 - 2.96023}{3.01831} \right| = \boxed{1.924\%} \]
\end{solution}

\begin{solution}Composite Simpson 1/3 $n=4$:\\
\[ h = (a-b)/n \]
Then for $k\in{1,\cdots,n}$
\[ x_{k} = a+kh = b-(n-k)h \]
Then 
\begin{align*}
\int_{0}^{4} (1-e^{-x})dx &\approx \frac{h}{3} \left[ (1-e^{-x_{0}}) +4(1-e^{-x_{1}}) +2(1-e^{-x_{2}}) +4(1-e^{-x_{3}}) +(1-e^{-x_{4}}) \right]\\
&= \frac{1}{3} \left[ (1-e^{0}) +4(1-e^{-1}) +2(1-e^{-2}) +4(1-e^{-3}) +(1-e^{-4}) \right]\\
&= \frac{1}{3} \left[ 11 -(4e^{-1} +2e^{-2} +4e^{-3} +e^{-4}) \right]\\
&= \frac{1}{3} \left[ 11 -4(e^{-1} +e^{-3}) -2e^{-2} -e^{-4} \right] \approx \boxed{3.01345}
\end{align*}
Where the relative error is 
\[ \left|\varepsilon_{r}\right| = \left| \frac{3.01831 - 3.01345}{3.01831} \right| = \boxed{0.161\%} \]
\end{solution}

\begin{solution}Simpson 3/8:\\
\begin{align*}
\int_{0}^{4} (1-e^{-x})dx &\approx \frac{3h}{8} \left[ (1-e^{-x_{0}}) +3(1-e^{-x_{1}}) +3(1-e^{-x_{2}}) +(1-e^{-x_{3}}) \right]\\
&= \frac{1}{2} \left[ (1-e^{0}) +3(1-e^{-4/3}) +3(1-e^{-8/3}) +(1-e^{-4}) \right]\\
&= \frac{1}{2} \left[ 7 -3e^{-4/3} -3e^{-8/3} -e^{-4} \right]\\
&= \frac{1}{2} \left[ 7 -3(e^{-4/3} +e^{-8/3}) -e^{-4} \right]\\
&\approx \boxed{2.99122}
\end{align*}
Where the relative error is 
\[ \left|\varepsilon_{r}\right| = \left| \frac{3.01831 - 2.99122}{3.01831} \right| = \boxed{0.898\%} \]
\end{solution}

\begin{solution}Composite Simpson $n=5$:\\
\[ h = (a-b)/n \]
Then for $k\in{1,\cdots,n}$
\[ x_{k} = a+kh = b-(n-k)h \]
\begin{align*}
&\int_{0}^{4} (1-e^{-x})dx \\
&\approx \frac{3h}{8} \left[ (1-e^{-x_{0}}) +3(1-e^{-x_{1}}) +3(1-e^{-x_{2}}) +(1-e^{-x_{3}}) \right] + \frac{h}{3}\left[ (1-e^{-x_{3}}) + 4(1-e^{-x_{4}}) + (1-e^{-x_{5}})\right]\\
&= \frac{3h}{8} \left[ (1-e^{-0}) +3(1-e^{-4/5}) +3(1-e^{-8/5}) +(1-e^{-12/5}) \right] + \frac{h}{3}\left[ (1-e^{-12/5}) + 4(1-e^{-16/5}) + (1-e^{-2})\right]\\
&= \frac{3}{10} \left[ 7 -3(e^{-4/5} +e^{-8/5}) -e^{-12/5} \right] + \frac{4}{15}\left[ 6 -e^{-12/5} -4e^{-16/5} -e^{-2} \right]\\
&\approx \frac{3}{10} \left[ 4.95560 \right] + \frac{4}{15}\left[ 5.88157 \right]\\
&= 1.48668 + 1.56842 = \boxed{3.0551}\\
\end{align*}
Where the relative error is 
\[ \left|\varepsilon_{r}\right| = \left| \frac{3.01831 - 3.05510}{3.01831} \right| = \boxed{1.219\%} \]

\end{solution}

\setcounter{section}{20}
\setcounter{problem}{1}
\begin{problem}
Evaluate:
\[ I = \int_{0}^{8} -0.055x^{4} +0.86x^{3} -4.2x^{2} +6.3x +2 dx \]
\begin{enumerate}[(a)]
\item analytically
\item Romberg integration ($\varepsilon_{s} = 0.5\%$)
\item Three point Gauss quadrature
\item MATLAB \verb|quad| function
\end{enumerate}
\end{problem}

\begin{solution}
Analytically:
\begin{align*}
I &= \int_{0}^{8} -0.055x^{4} +0.86x^{3} -4.2x^{2} +6.3x +2 dx\\
&= \left[ -\frac{0.055}{5}x^{5} +\frac{0.86}{4}x^{4} -\frac{4.2}{3}x^{3} +\frac{6.3}{2}x^{2} +2x \right]_{0}^{8}\\
&= -\frac{0.055}{5}8^{5} +\frac{0.86}{4}8^{4} -\frac{4.2}{3}8^{3} +\frac{6.3}{2}8^{2} +16\\
&= -(0.011)32768 +(0.215)4096 -(1.4)512 +(3.15)64 +16\\
&= -360.448 +880.64 -716.8 +201.6 +16 = \boxed{20.992}\\
\end{align*}
\end{solution}

\begin{solution}
The matrix containing the steps for Romberg Integration (as in Fig 20.1), where we stop at a relative error of $\varepsilon_{s}=0.5\%$, is given by:
\[
\begin{matrix}
2.5600	&	5.9733	&	20.9920	&	20.9920	\\
5.1200	&	20.0533	&	20.9920	&			\\
16.3200	&	20.9333	&			&			\\
19.7800	&			&			&			
\end{matrix}
\]
It takes three steps (of doubling the segments), each with full use of Richardson extrapolation to refine the approximation to produce these results.
\end{solution}

\begin{solution}
Three point Gauss quadrature
\end{solution}

\begin{solution}
Using the \verb|quad| function in MATLAB gives the same result of 20.9920.
\end{solution}

\end{document}