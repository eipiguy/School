\documentclass{article}

\usepackage{amsmath,amsfonts,amssymb,amsthm}
\usepackage{enumerate}
\usepackage{arydshln}
\usepackage{listings,color}
\usepackage{graphicx}

\definecolor{dkgreen}{rgb}{0,0.6,0}
\definecolor{gray}{rgb}{0.5,0.5,0.5}
\definecolor{mauve}{rgb}{0.58,0,0.82}

% Opening
\title{Numerical Analysis HW8\\
Ch12 - 3,9 (pg300)\\
Ch13 - 1,3 (pg317)\\}
\author{Neal D. Nesbitt}

\begin{document}
\maketitle

\theoremstyle{definition}
\newtheorem{problem}{Problem}
\newtheorem{solution}{Solution}[problem]

\lstset{basicstyle=\ttfamily,
		language=Matlab,
		keywordstyle=\color{blue},
		commentstyle=\color{dkgreen},
		stringstyle=\color{mauve},
		identifierstyle=\bf
		}

\setcounter{problem}{2}
\begin{problem}
Use the Gauss-Seidel method to solve the following system until the percent relative error falls below $\epsilon_{r}= \%$\\
\[
\begin{matrix}
10x_{1}	&+2x_{2}	&-x_{3}	&=	27\\
-3x_{1}	&-6x_{2}	&+2x_{3}	&=	-61.5\\
x_{1}	&+x_{2}	&+5x_{3}	&=	-21.5
\end{matrix}
\]
\end{problem}

\setcounter{problem}{8}
\begin{problem}
Determine the solutions of the simultaneous nonlinear equations:\\
\[ 
\begin{matrix}
x^{2} = 5-y^{2}\\
x^{2} = y+1
\end{matrix}
\]
\begin{enumerate}[(a)]
\item Graphically.
\item Successive substitution using initial guesses of $x=y=1.5$
\item Newton-Raphson using initial guesses of $x=y=1.5$
\end{enumerate}
\end{problem}

\setcounter{problem}{0}
\begin{problem}
Repeat example 13.1, but for three masses with the $m\text{'s}=40\text{ kg}$ and the $k\text{'s}=240\text{ N/m}$. Produce a plot like Fig. 13.4 to identify the principle modes of vibration.
\end{problem}

\setcounter{problem}{2}
\begin{problem}
Use the power method to determine the lowest eigenvalue and corresponding eigenvector for the system
\[
\begin{bmatrix}
2-\lambda	&	8			&	10			\\
8			&	4-\lambda	&	5			\\
10			&	5			&	7-\lambda
\end{bmatrix}
\]
\end{problem}

\end{document}