\documentclass{article}

\usepackage{amsmath,amsfonts,amssymb,amsthm}
\usepackage{enumerate}
\usepackage{arydshln}
\usepackage{listings,color}
\usepackage{graphicx}

\definecolor{dkgreen}{rgb}{0,0.6,0}
\definecolor{gray}{rgb}{0.5,0.5,0.5}
\definecolor{mauve}{rgb}{0.58,0,0.82}

% Opening
\title{Receding Horizon Control for MJS}
\author{Neal D. Nesbitt}

\begin{document}
\maketitle

\theoremstyle{definition}
\newtheorem{definition}{Definition}[section]
\newtheorem{corollary}{Corollary}[section]
\newtheorem{theorem}{Theorem}[section]

\lstset{basicstyle=\ttfamily,
		language=Matlab,
		keywordstyle=\color{blue},
		commentstyle=\color{dkgreen},
		stringstyle=\color{mauve},
		identifierstyle=\bf
		}

Given the control system
\[ x_{k} = A_{k-1}x_{k-1} + B_{k-1}u_{k-1} \]
where $u$ is dependent on $x$, we can recursively calculate the highest order term in terms of all the lower ones: $\forall k \in\mathbb{N}$

\begin{align*}
x_{1} &= A_{0}x_{0} + B_{0}u_{0} \\
x_{2} &= A_{1}\left( A_{0}x_{0} + B_{0}u_{0} \right) + B_{1}u_{1} \\
&= A_{1}A_{0}x_{0} + A_{1}B_{0}u_{0} + B_{1}u_{1} \\
x_{3} &= A_{2}x_{2} + B_{2}u_{2} \\
&= A_{2}\left( A_{1}A_{0}x_{0} + A_{1}B_{0}u_{0} + B_{1}u_{1} \right) + B_{2}u_{2} \\
&= A_{2}A_{1}A_{0}x_{0} + A_{2}A_{1}B_{0}u_{0} + A_{2}B_{1}u_{1} + B_{2}u_{2} \\
& \qquad \vdots \\
x_{k} &= \left( A_{k-1}\dots A_{1}A_{0} \right) x_{0} + \left( A_{k-2}\dots A_{2}A_{1} \right)B_{0}u_{0} + \left( A_{k-3}\dots A_{3}A_{2} \right)B_{1}u_{1} + \cdots \\ 
& \qquad + \left( A_{k-2}A_{k-1} \right) B_{k-3}u_{k-3} + A_{k-1}B_{k-2}u_{k-2} + B_{k-1}u_{k-1}
\end{align*}

\[ x_{k+1} = \left[ \prod_{n = k}^{0} A_{n} \right]x_{0} + \sum_{j = 1}^{k} \left[ \prod_{n = k}^{j} A_{n} \right]B_{j-1}u_{j-1} + B_{k}u_{k} \]

\paragraph{}
So consider the following cost function:
\begin{align*}
J(x_{0}) &= \sum_{k = 1}^{L} x_{k}^{\text{T}}R_{k}x_{k} + \sum_{k = 0}^{L-1} u_{k}^{\text{T}}Q_{k}u_{k} \\
J(x_{0}) &= \sum_{k = 1}^{L} x_{k}^{\text{T}}R_{k}x_{k} + u_{k-1}^{\text{T}}Q_{k-1}u_{k-1}
\end{align*}
Notice that there is no well defined $R_{0}$ or $Q_{L}$, and $x_{0}$ and $u_{L}$ are not taken into account.

To begin minimizing the cost function, $J$, with respect to the control vector, $u$, we take it's derivative with respect to the highest order term of $u$, since it has no other variables that depend on it. Then, because
\[ \frac{\partial}{\partial u_{k-1}} \left[ x_{k} \right] = \frac{\partial}{\partial u_{k-1}} \left[ A_{k-1}x_{k-1} + B_{k-1}u_{k-1} \right] = B_{k-1} \]
the chain rule gives
\[ \frac{\partial}{\partial u_{L-1}} \left[ J(x_{0}) \right] = 2 \left( x_{L}^{\text{T}}R_{L}B_{L-1} + u_{L-1}^{\text{T}}Q_{L-1} \right) \]
Set this equal to zero to find critical points.
\begin{align*}
2 \left( x_{L}^{\text{T}}R_{L}B_{L-1} + u_{L-1}^{\text{T}}Q_{L-1} \right) &= 0 \\
x_{L}^{\text{T}}R_{L}B_{L-1} + u_{L-1}^{\text{T}}Q_{L-1} &= 0 \\
u_{L-1}^{\text{T}}Q_{L-1} &= -x_{L}^{\text{T}}R_{L}B_{L-1} \\
u_{L-1}^{\text{T}} &= -x_{L}^{\text{T}}R_{L}B_{L-1}Q_{L-1}^{-1} \\
u_{L-1} &= -Q_{L-1}^{-1} B_{L-1}^{\text{T}} R_{L} x_{L} \\
u_{L-1} &= -Q_{L-1}^{-1} B_{L-1}^{\text{T}} R_{L} \left( A_{L-1}x_{L-1} + B_{L-1}u_{L-1} \right) \\
Q_{L-1}u_{L-1} &= -B_{L-1}^{\text{T}} R_{L} A_{L-1} x_{L-1} - B_{L-1}^{\text{T}} R_{L} B_{L-1} u_{L-1} \\
B_{L-1}^{\text{T}} R_{L} B_{L-1} u_{L-1} + Q_{L-1}u_{L-1} &= -B_{L-1}^{\text{T}} R_{L} A_{L-1} x_{L-1} \\
\left( B_{L-1}^{\text{T}} R_{L} B_{L-1} + Q_{L-1} \right) u_{L-1} &= -B_{L-1}^{\text{T}} R_{L} A_{L-1} x_{L-1}
\end{align*}
So assuming that $\left( B_{L-1}^{\text{T}} R_{L} B_{L-1} + Q_{L-1} \right)$ is invertible for all $L\in\mathbb{N}$, then
\[ \boxed{u_{L-1} = -\left( B_{L-1}^{\text{T}} R_{L} B_{L-1} + Q_{L-1} \right)^{-1} B_{L-1}^{\text{T}} R_{L} A_{L-1} x_{L-1}} \]

\paragraph{}
So substitute this into the original system to find $x_{L}$:
\begin{align*}
x_{L} &= A_{L-1}x_{L-1} + B_{L-1}u_{L-1} \\
x_{L} &= A_{L-1}x_{L-1} - B_{L-1} \left( B_{L-1}^{\text{T}} R_{L} B_{L-1} + Q_{L-1} \right)^{-1} B_{L-1}^{\text{T}} R_{L} A_{L-1} x_{L-1} \\
x_{L} &= \left( A_{L-1} - B_{L-1} \left( B_{L-1}^{\text{T}} R_{L} B_{L-1} + Q_{L-1} \right)^{-1} B_{L-1}^{\text{T}} R_{L} A_{L-1} \right) x_{L-1} \\
\end{align*}
and let
\[ \boxed{ G_{L-1} = \left( B_{L-1}^{\text{T}} R_{L} B_{L-1} + Q_{L-1} \right)^{-1} B_{L-1}^{\text{T}} R_{L} A_{L-1} } \]
so that 
\[ \boxed{ u_{L-1} = -G_{L-1} x_{L-1} } \]
and
\[ x_{L} = \left( A_{L-1} - B_{L-1} G_{L-1} \right) x_{L-1} \]
And call
\[ \boxed{ H_{L-1} = \left( A_{L-1} - B_{L-1} G_{L-1} \right) } \]
so that 
\[ \boxed{ x_{L} = H_{L-1} x_{L-1} } \]

\paragraph{}
Now reevaluate the cost function with the resulting derivations:
\begin{align*}
x_{L}^{\text{T}} R_{L} x_{L} &= \left( H_{L-1} x_{L-1} \right)^{\text{T}} R_{L} H_{L-1} x_{L-1} \\
&= x_{L-1}^{\text{T}} H_{L-1}^{\text{T}} R_{L} H_{L-1} x_{L-1} \\
u_{L-1}^{\text{T}} Q_{L-1} u_{L-1} &= \left( G_{L-1} x_{L-1} \right)^{\text{T}} R_{L} G_{L-1} x_{L-1} \\
&= x_{L-1}^{\text{T}} G_{L-1}^{\text{T}} R_{L} G_{L-1} x_{L-1}
\end{align*}

Then
\[ \boxed{ x_{L}^{\text{T}} R_{L} x_{L} + u_{L-1}^{\text{T}} Q_{L-1} u_{L-1} = x_{L-1}^{\text{T}} \left( H_{L-1}^{\text{T}} R_{L} H_{L-1} + G_{L-1}^{\text{T}} R_{L} G_{L-1} \right) x_{L-1} } \]

\paragraph{}
So consider $L \ge 2$. Substitute the result into the original system equation to reduce the number of terms by one.
\begin{align*}
J(x_{0}) &= \left( \sum_{k = 1}^{L-2} x_{k}^{\text{T}}R_{k}x_{k} + u_{k-1}^{\text{T}}Q_{k-1}u_{k-1} \right) + x_{L-1}^{\text{T}}R_{L-1}x_{L-1} + u_{L-2}^{\text{T}}Q_{L-2}u_{L-2} + x_{L}^{\text{T}}R_{L}x_{L} + u_{L-1}^{\text{T}}Q_{L-1}u_{L-1}
\end{align*}
The trailing terms become
\begin{align*}
&x_{L-1}^{\text{T}}R_{L-1}x_{L-1} + x_{L}^{\text{T}}R_{L}x_{L} + u_{L-1}^{\text{T}}Q_{L-1}u_{L-1}\\
&= x_{L-1}^{\text{T}}R_{L-1}x_{L-1} + x_{L-1}^{\text{T}} \left( H_{L-1}^{\text{T}} R_{L} H_{L-1} + G_{L-1}^{\text{T}} R_{L} G_{L-1} \right) x_{L-1} \\
&= x_{L-1}^{\text{T}} \left[ R_{L-1} + H_{L-1}^{\text{T}} R_{L} H_{L-1} + G_{L-1}^{\text{T}} R_{L} G_{L-1} \right] x_{L-1}
\end{align*}
Then, to simplify the math, call
\[ \boxed{ R^{\star}_{L-1} = R_{L-1} + H_{L-1}^{\text{T}} R_{L} H_{L-1} + G_{L-1}^{\text{T}} R_{L} G_{L-1} } \]
so that the last terms of the cost function reduce to
\[ \boxed{ x_{L-1}^{\text{T}}R_{L-1}x_{L-1} + u_{L-2}^{\text{T}}Q_{L-2}u_{L-2} + x_{L}^{\text{T}}R_{L}x_{L} + u_{L-1}^{\text{T}}Q_{L-1}u_{L-1} = x_{L-1}^{\text{T}} R^{\star}_{L-1} x_{L-1} + u_{L-2}^{\text{T}}Q_{L-2}u_{L-2} } \]

Now again since the last terms do not have anything else dependent on them, and since
\[ \frac{\partial}{\partial u_{k-1}} \left[ x_{k} \right] = \frac{\partial}{\partial u_{k-1}} \left[ A_{k-1}x_{k-1} + B_{k-1}u_{k-1} \right] = B_{k-1} \]
the chain rule with the new derivation gives
\[ \frac{\partial}{\partial u_{L-2}} \left[ J(x_{0}) \right] = 2 \left( x_{L-1}^{\text{T}} R^{\star}_{L-1} B_{L-2} + u_{L-1}^{\text{T}}Q_{L-1} \right) \]
and minimizing with respect to this term is equivalent to the process above, producing the following for $L \ge 2$:
\[ \boxed{u_{L-2} = -\left( B_{L-2}^{\text{T}} R^{\star}_{L-1} B_{L-2} + Q_{L-2} \right)^{-1} B_{L-2}^{\text{T}} R^{\star}_{L-1} A_{L-2} x_{L-2}} \]
\[ \boxed{ G_{L-2} = \left( B_{L-2}^{\text{T}} R^{\star}_{L-1} B_{L-2} + Q_{L-2} \right)^{-1} B_{L-2}^{\text{T}} R^{\star}_{L-1} A_{L-2} } \]
\[ \boxed{ u_{L-2} = -G_{L-2} x_{L-2} } \]
\[ x_{L-1} = \left( A_{L-2} - B_{L-2} G_{L-2} \right) x_{L-2} \]
\[ \boxed{ H_{L-2} = \left( A_{L-2} - B_{L-2} G_{L-2} \right) } \]
\[ \boxed{ x_{L-1} = H_{L-2} x_{L-2} } \]
and for any $j\in\mathbb{N}$, where $L > j$, this can be repeated where
\[ \boxed{u_{L-(j+1)} = -\left( B_{L-(j+1)}^{\text{T}} R^{\star}_{L-j} B_{L-(j+1)} + Q_{L-(j+1)} \right)^{-1} B_{L-(j+1)}^{\text{T}} R^{\star}_{L-j} A_{L-(j+1)} x_{L-(j+1)}} \]
\[ \boxed{ G_{L-(j+1)} = \left( B_{L-(j+1)}^{\text{T}} R^{\star}_{L-j} B_{L-(j+1)} + Q_{L-(j+1)} \right)^{-1} B_{L-(j+1)}^{\text{T}} R^{\star}_{L-j} A_{L-(j+1)} } \]
\[ \boxed{ u_{L-(j+1)} = -G_{L-(j+1)} x_{L-(j+1)} } \]
\[ x_{L-j} = \left( A_{L-(j+1)} - B_{L-(j+1)} G_{L-j} \right) x_{L-(j+1)} \]
\[ \boxed{ H_{L-(j+1)} = \left( A_{L-(j+1)} - B_{L-(j+1)} G_{L-(j+1)} \right) } \]
\[ \boxed{ x_{L-j} = H_{L-(j+1)} x_{L-(j+1)} } \]
\[ \boxed{ R^{\star}_{L-(j+1)} = R_{L-(j+1)} + H_{L-(j+1)}^{\text{T}} R^{\star}_{L-j} H_{L-(j+1)} + G_{L-(j+1)}^{\text{T}} R^{\star}_{L-j} G_{L-(j+1)} } \]
all the way to the last terms in the cost function where:
\[ x_{1}^{\text{T}}R_{1}x_{1} + u_{0}^{\text{T}}Q_{0}u_{0} + x_{2}^{\text{T}}R_{2}x_{2} + u_{1}^{\text{T}}Q_{1}u_{1} = x_{1}^{\text{T}} R^{\star}_{1} x_{1} + u_{0}^{\text{T}}Q_{0}u_{0} \]
\[ \boxed{u_{0} = -\left( B_{0}^{\text{T}} R^{\star}_{1} B_{0} + Q_{0} \right)^{-1} B_{0}^{\text{T}} R^{\star}_{1} A_{0} x_{0}} \]
\[ \boxed{ G_{0} = \left( B_{0}^{\text{T}} R^{\star}_{1} B_{0} + Q_{0} \right)^{-1} B_{0}^{\text{T}} R^{\star}_{1} A_{0} } \]
\[ \boxed{ u_{0} = -G_{0} x_{0} } \]
\[ \boxed{ x_{1} = \left( A_{0} - B_{0} G_{0} \right) x_{0} } \]

\paragraph{}
Then if $L>2$, for any $j\in\mathbb{Z}^{+}$ such that $j<L$, an explicit formula for $R^{\star}_{j}$ is needed.

Begin by recalling the initial case,
\[ \boxed{ R^{\star}_{L-1} = R_{L-1} + G_{L-1}^{\text{T}} R_{L} G_{L-1} + H_{L-1}^{\text{T}} R_{L} H_{L-1} } \]
and noticing that we can write this as
\[ R^{\star}_{L-1} = R_{L-1} +
\begin{bmatrix}
G_{L-1}^{\text{T}} & H_{L-1}^{\text{T}}
\end{bmatrix}
R_{L}
\begin{bmatrix}
G_{L-1} \\
H_{L-1}
\end{bmatrix}
 \]
Substituting this into the next iteration:
\begin{align*}
R^{\star}_{L-2} &= R_{L-2} \\
&\qquad +
\begin{bmatrix}
G_{L-2}^{\text{T}} & H_{L-2}^{\text{T}}
\end{bmatrix}
R_{L-1}
\begin{bmatrix}
G_{L-2} \\
H_{L-2}
\end{bmatrix} +
\begin{bmatrix}
G_{L-2}^{\text{T}} & H_{L-2}^{\text{T}}
\end{bmatrix}
\left(
\begin{bmatrix}
G_{L-1}^{\text{T}} & H_{L-1}^{\text{T}}
\end{bmatrix}
R_{L}
\begin{bmatrix}
G_{L-1} \\
H_{L-1}
\end{bmatrix}
\right)
\begin{bmatrix}
G_{L-2} \\
H_{L-2}
\end{bmatrix}
\end{align*}
and if we expand algebraically instead,
\begin{align*}
R^{\star}_{L-2} &= R_{L-2} \\
&\qquad + G_{L-2}^{\text{T}} \left( R_{L-1} + G_{L-1}^{\text{T}} R_{L} G_{L-1} + H_{L-1}^{\text{T}} R_{L} H_{L-1} \right) G_{L-2} \\
&\qquad + H_{L-2}^{\text{T}} \left( R_{L-1} + G_{L-1}^{\text{T}} R_{L} G_{L-1} + H_{L-1}^{\text{T}} R_{L} H_{L-1} \right) H_{L-2} \\
R^{\star}_{L-2} &= R_{L-2} \\
&\qquad + G_{L-2}^{\text{T}} R_{L-1} G_{L-2} + H_{L-2}^{\text{T}} R_{L-1} H_{L-2} \\ 
&\qquad + G_{L-2}^{\text{T}} G_{L-1}^{\text{T}} R_{L} G_{L-1} G_{L-2} + H_{L-2}^{\text{T}} G_{L-1}^{\text{T}} R_{L} G_{L-1} H_{L-2} \\
&\qquad + G_{L-2}^{\text{T}} H_{L-1}^{\text{T}} R_{L} H_{L-1} G_{L-2} + H_{L-2}^{\text{T}} H_{L-1}^{\text{T}} R_{L} H_{L-1} H_{L-2}
\end{align*}
Where in vector notation this becomes
\begin{align*}
R^{\star}_{L-2} &= R_{L-2} \\
&\qquad + \begin{bmatrix}
G_{L-2}^{\text{T}} & H_{L-2}^{\text{T}}
\end{bmatrix}
R_{L-1}
\begin{bmatrix}
G_{L-2} \\
H_{L-2}
\end{bmatrix} \\
&\qquad + 
\begin{bmatrix}
G_{L-1}^{\text{T}} G_{L-2}^{\text{T}} & H_{L-2}^{\text{T}} G_{L-1}^{\text{T}} & G_{L-2}^{\text{T}} H_{L-1}^{\text{T}} & H_{L-1}^{\text{T}} H_{L-2}^{\text{T}}
\end{bmatrix}
R_{L-1}
\begin{bmatrix}
G_{L-1} G_{L-2} \\
G_{L-1} H_{L-2} \\
H_{L-1} G_{L-2} \\
H_{L-1} H_{L-2}
\end{bmatrix}
\end{align*}

\paragraph{}
So consider following the improper notation:
\[ \boxed{
\begin{bmatrix}
A \\
B
\end{bmatrix}
\begin{bmatrix}
C \\
D \\
E
\end{bmatrix}
=
\begin{bmatrix}
AC \\
AD \\
AE \\
BC \\
BD \\
BE \\
\end{bmatrix}
}
\]
so that
\[
\begin{bmatrix}
A \\
B
\end{bmatrix}^{2}
=
\begin{bmatrix}
AA \\
AB \\
BA \\
BB
\end{bmatrix}
\]
\[ 
\begin{bmatrix}
A \\
B
\end{bmatrix}^{3}
=
\begin{bmatrix}
AAA \\
AAB \\
ABA \\
ABB \\
BAA \\
BAB \\
BBA \\
BBB
\end{bmatrix}
\]
and so on.

\paragraph{}
Additionally, call
\[ \boxed{
\begin{bmatrix}
A \\
B
\end{bmatrix}^{\text{T}\star}
=
\begin{bmatrix}
A^{\text{T}} & B^{\text{T}}
\end{bmatrix}
} \]

\paragraph{}
so that we can write
\[ \boxed{
R^{\star}_{L-j} = R_{L-j} + \sum_{k=0}^{j-1} \left( \left( \prod_{h = k+1}^{j}
\begin{bmatrix}
G_{L-h} \\
H_{L-h}
\end{bmatrix}
\right)^{\text{T}\star}
R_{L-k}
\left( \prod_{h = k+1}^{j}
\begin{bmatrix}
G_{L-h} \\
H_{L-h}
\end{bmatrix}
\right)
\right)
} \]
We must then use this to find an explicit formula for $G$, and consequently $H$.

\paragraph{}
So again, consider the initial cases for $G$ and $H$,
\[ \boxed{ G_{L-1} = \left( B_{L-1}^{\text{T}} R_{L} B_{L-1} + Q_{L-1} \right)^{-1} B_{L-1}^{\text{T}} R_{L} A_{L-1} } \]
\[ \boxed{ H_{L-1} = \left( A_{L-1} - B_{L-1} G_{L-1} \right) } \]
and for the first case including an $R^{\star}$ term,
\[ G_{L-2} = \left( B_{L-2}^{\text{T}} R^{\star}_{L-1} B_{L-2} + Q_{L-2} \right)^{-1} B_{L-2}^{\text{T}} R^{\star}_{L-1} A_{L-2} \]
replace in the formula for $R^{\star}_{L-1}$.
\begin{align*}
G_{L-2} &= \left( B_{L-2}^{\text{T}} R^{\star}_{L-1} B_{L-2} + Q_{L-2} \right)^{-1} B_{L-2}^{\text{T}} R^{\star}_{L-1} A_{L-2} \\
G_{L-2} &= \left( B_{L-2}^{\text{T}} \left( R_{L-1} + G_{L-1}^{\text{T}} R_{L} G_{L-1} + H_{L-1}^{\text{T}} R_{L} H_{L-1} \right) B_{L-2} + Q_{L-2} \right)^{-1} \\
&\qquad B_{L-2}^{\text{T}} \left( R_{L-1} + G_{L-1}^{\text{T}} R_{L} G_{L-1} + H_{L-1}^{\text{T}} R_{L} H_{L-1} \right) A_{L-2} \\
\end{align*}
Then compute separately
\begin{align*}
H_{L-1}^{\text{T}} R_{L} H_{L-1} &= \left( A_{L-1} - B_{L-1} G_{L-1} \right)^{\text{T}} R_{L} \left( A_{L-1} - B_{L-1} G_{L-1} \right) \\
&= \left( A_{L-1}^{\text{T}} - G_{L-1}^{\text{T}} B_{L-1}^{\text{T}} \right) R_{L} \left( A_{L-1} - B_{L-1} G_{L-1} \right) \\
&= \left( A_{L-1}^{\text{T}} R_{L} - G_{L-1}^{\text{T}} B_{L-1}^{\text{T}} R_{L} \right) \left( A_{L-1} - B_{L-1} G_{L-1} \right) \\
&= A_{L-1}^{\text{T}} R_{L} A_{L-1} + G_{L-1}^{\text{T}} B_{L-1}^{\text{T}} R_{L} B_{L-1} G_{L-1} - A_{L-1}^{\text{T}} R_{L} B_{L-1} G_{L-1} - G_{L-1}^{\text{T}} B_{L-1}^{\text{T}} R_{L} A_{L-1} \\
&= 
\begin{bmatrix}
A_{L-1}^{\text{T}} & G_{L-1}^{\text{T}} B_{L-1}^{\text{T}}
\end{bmatrix}
R_{L}
\begin{bmatrix}
A_{L-1} \\
B_{L-1} G_{L-1}
\end{bmatrix}
-
\begin{bmatrix}
A_{L-1}^{\text{T}} & G_{L-1}^{\text{T}} B_{L-1}^{\text{T}}
\end{bmatrix}
R_{L}
\begin{bmatrix}
B_{L-1} G_{L-1} \\
A_{L-1}
\end{bmatrix} \\ 
&= 
\begin{bmatrix}
A_{L-1}^{\text{T}} & G_{L-1}^{\text{T}} B_{L-1}^{\text{T}}
\end{bmatrix}
R_{L}
\left(
\begin{bmatrix}
A_{L-1} \\
B_{L-1} G_{L-1}
\end{bmatrix}
-
\begin{bmatrix}
B_{L-1} G_{L-1} \\
A_{L-1}
\end{bmatrix}
\right)
\end{align*}

\end{document}