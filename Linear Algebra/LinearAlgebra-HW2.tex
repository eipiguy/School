\documentclass{article}

\usepackage{amsmath,amsfonts,amssymb,amsthm}
\usepackage{listings,color}
\usepackage{graphicx}


% Opening
\title{Linear Algebra HW2\\
Exercises 169, 185, 187, 202, 205, 214, 232, 234}
\author{Neal D. Nesbitt}

\begin{document}
\maketitle

\theoremstyle{definition}
\newtheorem{problem}{Problem}[section]
\newtheorem{solution}{Solution}[problem]
\renewcommand{\thesolution}{\theproblem}

\setcounter{section}{5}
\setcounter{problem}{168}
\begin{problem}
Let $F$ be a field of characteristic different from 2 and let $V$ be  a vector space over $F$ containing a linearly independent subset $\{ v_{1}, v_{2}, v_{3} \}$. Show that the set $\{ v_{1}+v_{2}, v_{2}+v_{3}, v_{1}+v_{3} \}$ is also linearly independent.
\end{problem}

\begin{solution}
\begin{proof}
Let our notation be as above, and assume by way of contradiction that $\{ v_{1}+v_{2}, v_{2}+v_{3}, v_{1}+v_{3} \}$ is not linearly independent. Then it is linearly dependent, and there are some $a_{1},a_{2},a_{3} \in F$ such that $a_{1} \ne 0, a_{2} \ne 0, a_{3} \ne 0$ and:
\[ a_{1}( v_{1}+v_{2} ) +a_{2}( v_{2}+v_{3} ) +a_{3}( v_{1}+v_{3} ) = 0 \]

\paragraph{}
Now distribute the scalars over the vectors and use commutativity to reorder them. If we recollect the vectors as instead distributed over the scalars, we find:
\begin{align*} 
a_{1}( v_{1}+v_{2} ) +a_{2}( v_{2}+v_{3} ) +a_{3}( v_{1}+v_{3} ) &= 0 \\
a_{1}v_{1} +a_{1}v_{2} +a_{2}v_{2} +a_{2}v_{3} +a_{3}v_{1} +a_{3}v_{3} &= 0 \\
a_{1}v_{1} +a_{3}v_{1} +a_{1}v_{2} +a_{2}v_{2} +a_{2}v_{3} +a_{3}v_{3} &= 0 \\
( a_{1}+a_{3} )v_{1} +( a_{1}+a_{2} )v_{2} +( a_{2}+a_{3} )v_{3} &= 0 \\
\end{align*}

\paragraph{}
Since we know $F$ is not of characteristic two, and each $a$ is non-zero, then $a_{1}+a_{3} \ne 0, a_{1}+a_{2} \ne 0, a_{2}+a_{3} \ne 0$ also, showing $\{ v_{1}, v_{2}, v_{3} \}$ must be linearly dependent.

\paragraph{}
This contradicts our assumption that $\{ v_{1}, v_{2}, v_{3} \}$ is linearly independent. So our assumption that $\{ v_{1}+v_{2}, v_{2}+v_{3}, v_{1}+v_{3} \}$ is linearly dependent must be wrong, and it must instead be linearly independent as desired.
\end{proof}
\end{solution}

\setcounter{problem}{184}
\begin{problem}
Let $\{ v_{1}, \cdots, v_{n} \}$ be a basis for a vector space $V$ over a field $F$. Is the set $\{ v_{1}+v_{2}, v_{2}+v_{3}, \cdots, v_{n-1}+v_{n}, v_{n}+v_{1} \}$ necessarily also a basis for $V$ over $F$?
\end{problem}

\begin{solution}
No, this is not always the case.

\begin{proof}
Take $F$ to be a field of characteristic 2, and let $\{ v_{1}, \cdots, v_{n} \}$ be a basis for a vector space $V$ over $F$. Then take $a \in F$ such that $a\ne 0$. Since every basis is linearly independent,
\[ \sum_{i=1}^{n} av_{i} \ne 0 \]
Now notice how this vector added to itself is a linear combination of $\{ v_{1}+v_{2}, v_{2}+v_{3}, \cdots, v_{n-1}+v_{n}, v_{n}+v_{1} \}$.
\begin{align*}
\sum_{i=1}^{n} av_{i} + \sum_{i=1}^{n} av_{i} &= a\left( \sum_{i=1}^{n} v_{i} +\sum_{i=1}^{n} v_{i} \right)\\
&=  a\left( v_{1} +\sum_{i=2}^{n} v_{i} +\sum_{i=1}^{n-1} v_{i} +v_{n} \right)\\
&= a\left( v_{1} +\sum_{i=1}^{n-1} v_{i+1} +\sum_{i=1}^{n-1} v_{i} +v_{n} \right)\\
&= a\left( v_{1} +\sum_{i=1}^{n-1} ( v_{i+1} +v_{i} ) +v_{n} \right)\\
&= a\left( \sum_{i=1}^{n-1} ( v_{i} +v_{i+1} ) +(v_{1}+v_{n}) \right)\\
\sum_{i=1}^{n} av_{i} + \sum_{i=1}^{n} av_{i} &= \sum_{i=1}^{n-1} a( v_{i} +v_{i+1} ) +a(v_{1}+v_{n})
\end{align*}

\paragraph{}
But since $F$ is of characteristic 2, 
\[ \sum_{i=1}^{n} av_{i} + \sum_{i=1}^{n} av_{i} = (a+a)\sum_{i=1}^{n} v_{i} = a(1+1)\sum_{i=1}^{n} v_{i} = 0 \sum_{i=1}^{n} v_{i} = 0\]
Then we must also have that
\[  \sum_{i=1}^{n-1} a( v_{i} +v_{i+1} ) +a(v_{1}+v_{n}) = 0 \]
showing $\{ v_{1}+v_{2}, v_{2}+v_{3}, \cdots, v_{n-1}+v_{n}, v_{n}+v_{1} \}$ is not linearly independent, and therefore cannot be a basis. This shows that not every set of the given type constitutes a basis of the original vector space.
\end{proof}
\end{solution}

\setcounter{problem}{186}
\begin{problem}
For which values of $a \in \mathbb{R}$ is the set
\[
\left\{
\begin{bmatrix}
a	&	2a	\\
2	&	3a
\end{bmatrix}
,
\begin{bmatrix}
1	&	2	\\
2a	&	3
\end{bmatrix}
,
\begin{bmatrix}
1	&	2a	\\
a+1	&	a+2
\end{bmatrix}
,
\begin{bmatrix}
1	&	a+1	\\
2	&	2a+1
\end{bmatrix}
\right\}
\]
a basis for $\mathcal{M}_{2\times 2}\left( \mathbb{R} \right)$ as a vector space over $\mathbb{R}$?
\end{problem}

\begin{solution}
Let the notation be as above.

\paragraph{}
A basis is a linearly independent set that spans the space. Since $\mathcal{M}_{2\times 2}\left( \mathbb{R} \right)$ as a vector space over $\mathbb{R}$ has dimension 4, then if the given set is linearly independent it will be a subspace with the same dimension (since the set has 4 elements), and therefore must span the whole space.

\paragraph{}
So let us consider the conditions on $a$ for which the set is linearly dependent and take the complement.

\paragraph{}
If the set is linearly dependent, there exist $c_{1},c_{2},c_{3},c_{4} \in \mathbb{R}$ such that $c_{1}\ne 0,c_{2}\ne 0,c_{3}\ne 0,c_{4}\ne 0$. Then 
\begin{align*}
c_{1}
\begin{bmatrix}
a	&	2a	\\
2	&	3a
\end{bmatrix}
+c_{2}
\begin{bmatrix}
1	&	2	\\
2a	&	3
\end{bmatrix}
+c_{3}
\begin{bmatrix}
1	&	2a	\\
a+1	&	a+2
\end{bmatrix}
+c_{4}
\begin{bmatrix}
1	&	a+1	\\
2	&	2a+1
\end{bmatrix}
&= 0 \\
c_{1}
\left(
a
\begin{bmatrix}
1	&	2	\\
0	&	3
\end{bmatrix}
+
\begin{bmatrix}
0	&	0	\\
2	&	0
\end{bmatrix}
\right)&\\
+c_{2}
\left(
a
\begin{bmatrix}
0	&	0	\\
2	&	0
\end{bmatrix}
+
\begin{bmatrix}
1	&	2	\\
0	&	3
\end{bmatrix}
\right)&\\
+c_{3}
\left(
a
\begin{bmatrix}
0	&	2	\\
1	&	1
\end{bmatrix}
+
\begin{bmatrix}
1	&	0	\\
1	&	2
\end{bmatrix}
\right)&\\
+c_{4}
\left(
a
\begin{bmatrix}
0	&	1	\\
0	&	2
\end{bmatrix}
+
\begin{bmatrix}
1	&	1	\\
2	&	1
\end{bmatrix}
\right) &= 0
\end{align*}

\begin{multline*}
a
\left(
c_{1}
\begin{bmatrix}
1	&	2	\\
0	&	3
\end{bmatrix}
+c_{2}
\begin{bmatrix}
0	&	0	\\
2	&	0
\end{bmatrix}
+c_{3}
\begin{bmatrix}
0	&	2	\\
1	&	1
\end{bmatrix}
+c_{4}
\begin{bmatrix}
0	&	1	\\
0	&	2
\end{bmatrix}
\right)
=\\
-
\left(
c_{1}
\begin{bmatrix}
0	&	0	\\
2	&	0
\end{bmatrix}
+c_{2}
\begin{bmatrix}
1	&	2	\\
0	&	3
\end{bmatrix}
+c_{3}
\begin{bmatrix}
1	&	0	\\
1	&	2
\end{bmatrix}
+c_{4}
\begin{bmatrix}
1	&	1	\\
2	&	1
\end{bmatrix}
\right)
\end{multline*}

\[
a \left[
\begin{array}{c@{} c@{} c@{} c@{}}
\begin{bmatrix}
1	&	2	\\
0	&	3
\end{bmatrix}
&
\begin{bmatrix}
0	&	0	\\
2	&	0
\end{bmatrix}
&
\begin{bmatrix}
0	&	2	\\
1	&	1
\end{bmatrix}
&
\begin{bmatrix}
0	&	1	\\
0	&	2
\end{bmatrix}
\end{array}
\right]
\begin{bmatrix}
c_{1}	\\
c_{2}	\\
c_{3}	\\
c_{4}
\end{bmatrix}
=\\
-
\left[
\begin{array}{c@{} c@{} c@{} c@{}}
\begin{bmatrix}
0	&	0	\\
2	&	0
\end{bmatrix}
&
\begin{bmatrix}
1	&	2	\\
0	&	3
\end{bmatrix}
&
\begin{bmatrix}
1	&	0	\\
1	&	2
\end{bmatrix}
&
\begin{bmatrix}
1	&	1	\\
2	&	1
\end{bmatrix}
\end{array}
\right]
\begin{bmatrix}
c_{1}	\\
c_{2}	\\
c_{3}	\\
c_{4}
\end{bmatrix}
\]
\[
a \left[
\begin{array}{c@{} c@{} c@{} c@{}}
\begin{bmatrix}
1	&	2	\\
0	&	3
\end{bmatrix}
&
\begin{bmatrix}
0	&	0	\\
2	&	0
\end{bmatrix}
&
\begin{bmatrix}
0	&	2	\\
1	&	1
\end{bmatrix}
&
\begin{bmatrix}
0	&	1	\\
0	&	2
\end{bmatrix}
\end{array}
\right]
=\\
-
\left[
\begin{array}{c@{} c@{} c@{} c@{}}
\begin{bmatrix}
0	&	0	\\
2	&	0
\end{bmatrix}
&
\begin{bmatrix}
1	&	2	\\
0	&	3
\end{bmatrix}
&
\begin{bmatrix}
1	&	0	\\
1	&	2
\end{bmatrix}
&
\begin{bmatrix}
1	&	1	\\
2	&	1
\end{bmatrix}
\end{array}
\right]
\]\\
but since
\begin{align*}
\begin{bmatrix}
a	&	2a	\\
0	&	3a
\end{bmatrix}
&= \begin{bmatrix}
0	&	0	\\
-2	&	0
\end{bmatrix}\\
\begin{bmatrix}
0	&	0	\\
2a	&	0
\end{bmatrix}
&=
\begin{bmatrix}
-1	&	-2	\\
0	&	-3
\end{bmatrix}\\
\begin{bmatrix}
0	&	2a	\\
a	&	a
\end{bmatrix}
&= \begin{bmatrix}
-1	&	0	\\
-1	&	-2
\end{bmatrix}\\
\begin{bmatrix}
0	&	a	\\
0	&	2a
\end{bmatrix}
&= \begin{bmatrix}
-1	&	-1	\\
-2	&	-1
\end{bmatrix}
\end{align*}
each have no solution for $a \in \mathbb{R}$, (since $0\ne -2, 0\ne -1$, etc \dots) then the row vectors will never be equal, and the set must always be linearly independent. So since it also spans the entire space, it is thus a basis for $\mathcal{M}_{2\times 2}\left( \mathbb{R} \right)$ as a vector space over $\mathbb{R}$, for any $a\in\mathbb{R}$.
\end{solution}

\setcounter{problem}{201}
\begin{problem}\label{p202}
Let $V$ be a vector space of finite dimension $n$ over a field $F$, and let $W$ be a subspace of $V$ of dimension $n-1$. If $U$ is a subspace not contained in $W$, show that $\dim(W\cap U) = \dim(U)-1$.
\end{problem}

\begin{solution}\begin{proof}
Let our notation be as above. Since we know that 
\begin{align*}
\dim(W\cap U) + \dim(W\cup U)&= \dim(W) +\dim(U)\\
\dim(V) &= n\\
\dim(W) &= n-1
\end{align*}

\paragraph{}
And since $U\not\subset W\subset V$ and $U\subset V$ implies $\dim(V)\ge\dim(U),\dim(V)\ge\dim(W)$, then $\dim(V) = n \ge \dim(W\cup U) > \dim(W)=n-1$, 
implying that $\dim(V) = n = \dim(W\cup U)$ since dimensions only take on integer values.

\paragraph{}
Thus we can substitute our findings into our initial equation to find:
\begin{align*}
\dim(W\cap U) &= \dim(W) +\dim(U) -\dim(W\cup U)\\
\dim(W\cap U) &= (n-1) +\dim(U) -n = \dim(U)-1\\
\end{align*}
\end{proof}
\end{solution}

\setcounter{problem}{204}
\begin{problem}
Let \[W = \mathbb{R}
\left\{
\begin{bmatrix}
2\\
1\\
3\\
1
\end{bmatrix},
\begin{bmatrix}
1\\
2\\
0\\
1
\end{bmatrix},
\begin{bmatrix}
-1\\
1\\
-3\\
0
\end{bmatrix}
\right\} \subseteq \mathbb{R}
\]
Determine the dimension of $W$ and find a basis for it.
\end{problem}

\begin{solution}
Let our notation be as above. Since 
\[
\begin{bmatrix}
-1\\
1\\
-3\\
0
\end{bmatrix}
=
\begin{bmatrix}
1\\
2\\
0\\
1
\end{bmatrix}
-
\begin{bmatrix}
2\\
1\\
3\\
1
\end{bmatrix}
\]
and for any $a\in \mathbb{R}$
\[
a
\begin{bmatrix}
1\\
2\\
0\\
1
\end{bmatrix}
\ne
\begin{bmatrix}
2\\
1\\
3\\
1
\end{bmatrix}
\]
then the set 
\[
W \supset V=
\left\{
\begin{bmatrix}
1\\
2\\
0\\
1
\end{bmatrix}
,
\begin{bmatrix}
2\\
1\\
3\\
1
\end{bmatrix}
\right\}
\]
is linearly independent, but $W$ is not. Thus $\dim(W) = \dim(V) = 2$ and $V$ serves as a basis for the space.
\end{solution}

\setcounter{problem}{213}
\begin{problem}
Let $V$ be a vector space of finite dimension $n$ over a field $F$ and let $W$ and $Y$ be distinct subspaces of $V$, each of dimension $n-1$. What is $\dim(W\cap Y)$?
\end{problem}

\begin{solution}
Let our notation be as above. Since $W$ and $Y$ are distinct, $W\ne Y$, then $\dim(W\cup Y) = n$ since $\dim(V)=n\ge\dim(W\cup Y)>\dim(W)=n-1$, and dimensions only take on integer values.

\paragraph{}
Then since 
\begin{align*}
\dim(W\cap Y) + \dim(W\cup Y) &= \dim(W) + \dim(Y)\\
\dim(W\cap Y) + n &= 2n-2\\
\dim(W\cap Y) &= n-2\\
\end{align*}
we have found our solution.
\end{solution}

\setcounter{problem}{231}
\begin{problem}
Let $V$ be a vector space over a field $F$ and let $D$ be a finite minimal linearly dependent subset of $V$. Find $\dim(FD)$.
\end{problem}

\begin{solution}
Let our notation be as above. Then since $D$ minimal linearly dependent, that means that removing any one element makes it linearly independent, and thus the dimension of $D$ is one less than the number of elements it contains.

\paragraph{}
So since linear combinations of a field over some subset of vectors in a space produces a subspace,
$\dim(V)\ge\dim(FD)$ and $\dim(FD) = |D| -1$.
\end{solution}

\setcounter{section}{6}
\setcounter{problem}{233}
\begin{problem}
Let $\alpha:\mathbb{R}\to\mathbb{R}$ be a linear transformation satisfying $\alpha\left(\begin{bmatrix}
1\\0\\1
\end{bmatrix}\right)=\begin{bmatrix}
-1\\3\\4
\end{bmatrix}$, $\alpha\left(\begin{bmatrix}
1\\-1\\1
\end{bmatrix}\right)=\begin{bmatrix}
0\\1\\0
\end{bmatrix}$, and $\alpha\left(\begin{bmatrix}
1\\2\\-1
\end{bmatrix}\right)=\begin{bmatrix}
3\\1\\4
\end{bmatrix}$. What is $\alpha\left( \begin{bmatrix} 1\\0\\0 \end{bmatrix} \right)$?
\end{problem}

\begin{solution}
Since
\[ 
\begin{bmatrix}
1\\
0\\
0
\end{bmatrix}
=
\frac{1}{2}
\left(
\begin{bmatrix}
1\\
2\\
-1
\end{bmatrix}
+2
\begin{bmatrix}
1\\
-1\\
1
\end{bmatrix}
-
\begin{bmatrix}
1\\
0\\
1
\end{bmatrix}
\right)
\]
then
\begin{align*}
\alpha 
\left(\begin{bmatrix}
1\\
0\\
0
\end{bmatrix}\right)
&=
\frac{1}{2}
\alpha
\left(
\begin{bmatrix}
1\\
2\\
-1
\end{bmatrix}
+2
\begin{bmatrix}
1\\
-1\\
1
\end{bmatrix}
-
\begin{bmatrix}
1\\
0\\
1
\end{bmatrix}
\right)\\
&= \frac{1}{2}
\alpha
\left(
\begin{bmatrix}
1\\
2\\
-1
\end{bmatrix}
\right)
+\alpha\left(
\begin{bmatrix}
1\\
-1\\
1
\end{bmatrix}
\right)
-\frac{1}{2}\alpha\left(
\begin{bmatrix}
1\\
0\\
1
\end{bmatrix}
\right)\\
&= \frac{1}{2}
\begin{bmatrix}
3\\
1\\
4
\end{bmatrix}
+
\begin{bmatrix}
0\\
1\\
0
\end{bmatrix}
-\frac{1}{2}
\begin{bmatrix}
-1\\
3\\
4
\end{bmatrix}\\
&=
\begin{bmatrix}
2\\
0\\
0
\end{bmatrix}
\end{align*}
\end{solution}

\end{document}