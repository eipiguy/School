\documentclass{article}

\usepackage{amsmath,amsfonts,amssymb,amsthm}
\usepackage{listings,color}
\usepackage{graphicx}


% Opening
\title{Linear Algebra HW1\\
Exercises 1, 9, 12, 18, 61, 70, 87, 104, 117}
\author{Neal D. Nesbitt}

\begin{document}
\maketitle

\theoremstyle{definition}
\newtheorem{problem}{Problem}
\newtheorem{solution}{Solution}[problem]
\renewcommand{\thesolution}{\theproblem}

\begin{problem}
Let $F$ be a field and let $G = F \times F$. Define operations of addition and multiplication on $G$ by setting $(a,b) + (c,d) = (a+b,c+d)$ and $(a,b)\cdot (c,d) = (ac,bd)$. Do these operations define the structure of a field on $G$?
\end{problem}

\begin{solution}
These operations do not define a field on $G$.
\begin{proof}
Let $F$ and $(G,+,\cdot)$ be defined as above. Then for any $[a \ne 0] \in F$, $\left[(a,0) \ne (0,0)\right] \in G$ has no inverse in $G$, since $0$ has no inverse in $F$, and multiplication in $G$ is done component-wise.\\
\\
Thus $(G,+,\cdot)$ does not have inverses for every non-identity element, and is not a field.
\end{proof}
\end{solution}

\setcounter{problem}{8}
\begin{problem}
Let $F$ be a field and define a new operation $\star$ on $F$ by setting $a \star b = a + b + ab$. When is $(F, +, \star)$ a field?
\end{problem}

\begin{solution}
Let $(F, +, \star)$ be defined as above, and notice the following:

\paragraph{Distributivity}
$\star$ does not distribute across addition for any $a,b,c \in F$.
\begin{align*}
a \star (b+c) &= \boxed{a + b + c + ab + ac} \\
(a \star b) + (a \star c) &= (a + b + ab) + (a + c + ac) \\
&= \boxed{2a + b + c + ab + ac}\\
\end{align*}
Meaning that $a=0$ is the only element that distributes:
\begin{align*}
a \star (b+c) &= (a \star b) + (a \star c) \\
a + b + c + ab + ac &= 2a + b + c + ab + ac \\
a &= 2a \\
0 &= a
\end{align*}
Thus $F$ is only a field in the trivial case where $F=\{0\}$.
\end{solution}

\setcounter{problem}{11}
\begin{problem}
Let $F$ be a field. Show that the function $a \mapsto a^{-1}$ is a permutation of $F \backslash \{0_{F}\}$
\end{problem}

\begin{solution}
\begin{proof}
Let the notation be as above. A permutation is a function that is bijective and closed. Since every non-identity element has an inverse, then $F \backslash \{0_{F}\}$ will be closed under the inverse function.

So it remains to verify bijectivity. Since every non-identity element in $F$ has an inverse, then the function must be surjective (onto) in $F \backslash \{0_{F}\}$, and since no two elements have the same inverse (inverses are unique), then the function is injective (one-to-one) in $F \backslash \{0_{F}\}$ as well. This shows together that the inverse function is bijective and, with closure, a permutation on $F \backslash \{0_{F}\}$.
\end{proof}
\end{solution}

\setcounter{problem}{17}
\begin{problem}
Show that for all $z \in \mathbb{C}$, $|z+1| \le |z+1|^{2} + |z|$.
\end{problem}

\begin{solution}
\begin{proof}
Let $z \in \mathbb{C}$. Note that for any $w\in\mathbb{C}$, $0\le |w|$. Consider the following cases: 

\paragraph{}
If $|z+1|\le |z|$, then 
\begin{align*}
|z+1| - |z| &\le 0 \le |z+1|^{2} \\
|z+1| &\le |z+1|^{2} + |z|
\end{align*}
and we are finished.

\paragraph{}
So instead assume that $|z+1| > |z|$, so that $|z+1| - |z| > 0$. Then by the triangle inequality,
\begin{align*}
1 = |1 + z - z| &\le |z+1| + |z| \\
\left( |z+1| - |z| \right) &\le \left( |z+1| + |z| \right) \left( |z+1| - |z| \right) \\
|z+1| - |z| &\le |z+1|^{2} - |z|^{2} \\
|z+1| + |z|^{2} &\le |z+1|^{2} + |z| \\
|z+1| &\le |z+1|^{2} + |z| \\
\end{align*}
and again we have reached our desired result.
\end{proof}
\end{solution}

\setcounter{problem}{60}
\begin{problem}
Let $V$ be a non-trivial vector space over $\mathbb{R}$. For each $v \in V$ and each complex number $a + bi$, let us define $(a+bi)v = av$. Does $V$, together with this new scalar multiplication, form a vector space over $\mathbb{C}$?
\end{problem}

\begin{solution}
The construction above does not form a vector space over $\mathbb{C}$.
\begin{proof}
Let the notation be as above, and let $(a+bi),(c+di) \in \mathbb{C}$ and $v \in V$. Then if we check the associativity of vector multiplication we find:
\begin{align*}
\left[(a+bi)(c+di)\right]v &= \left( (ac-bd) + (ad+bc)i \right)v = (ac-bd)v\\
(a+bi)\left[(c+di)\right]v &= (a+bi)[cv] = acv
\end{align*}
which means that only real numbers associate with scalar multiplication, and this construction does not form a vector space over $\mathbb{C}$.
\end{proof}
\end{solution}

\setcounter{problem}{69}
\begin{problem}
Show that $\mathbb{Z}$ is not a vector space over any field.
\end{problem}

\begin{solution}
\begin{proof}
Let $F$ be a field, and assume by way of contradiction that $\mathbb{Z}$ forms a vector space over $F$.

Then if we pick $a \in F$ such that $a1>1$ (without loss of generality since $[a1<1]\to [1<a^{-1}1]$).
Since $\forall a\in F$, $1_{F}=a^{-1}a$, $0=a0$, and $1 = 1_{F}1$, then
\begin{align*}
a1 &> 1_{F}1 > 0	\\
a1 &> a^{-1}a1 > a0	\\
1 &> a^{-1}1 > 0
\end{align*}
but there is no integer between zero and one. Therefore $a^{-1}1 \notin \mathbb{Z}$ and is thus not in the vector space, contradicting the need for closure under scalar multiplication. Since $F$ was arbitrary, $\mathbb{Z}$ cannot form a vector space over any $F$.
\end{proof}
\end{solution}

\setcounter{problem}{86}
\begin{problem}
Let $F$ be a field, and let $V=F^{F}$, which is a vector space over $F$.

\paragraph{}
Let $W$ be the set of all functions $f\in V$ such that $f(1)=f(-1)$. Is $W$ a subspace of $V$?
\end{problem}

\begin{solution}
Let the notation be as above. $W$ is a subspace of $V$. To check its validity we verify its closure under addition and scalar multiplication for all $v,w \in W$ and $a \in F$.

\paragraph{Addition}
Since $v(1)=v(-1)$ and $w(1)=w(-1)$, then $(v+w)(1)=v(1)+w(1)=v(-1)+w(-1)=(v+w)(-1)$, showing $W$ is closed under vector addition.

\paragraph{Scalar Multiplication}
Since $v(1)=v(-1)$, then $av(1)=av(-1)$, showing $W$ is closed under scalar multiplication, and is thus a vector space as constructed above.
\end{solution}

\setcounter{problem}{103}
\begin{problem}
Find subspaces $W$ and $Y$ of $\mathbb{R}^{3}$ having the property that $W\cup Y$ is not a subspace of $\mathbb{R}^{3}$.
\end{problem}

\begin{solution}
Let $W\subset\mathbb{R}^{3}$ be the x-y plane, and let $W\subset\mathbb{R}^{3}$ be the x-z plane. Then both are subspaces individually since they are equivalent to $\mathbb{R}^{2}$, but if we take a non-identity vector from each, their sum will not be in either one of the planes. For example $(1,1,0) + (0,1,1) = (1,2,1)$ which is not on either of the planes.
\end{solution}

\setcounter{problem}{116}
\begin{problem}
Define $\cdot$ on $\mathbb{R}^{2}$ such that for any $\begin{bmatrix}
a \\
b
\end{bmatrix},
\begin{bmatrix}
c \\
d
\end{bmatrix}
\in \mathbb{R}^{2}$
\[
\begin{bmatrix}
a \\
b
\end{bmatrix}
\cdot
\begin{bmatrix}
c \\
d
\end{bmatrix}
=
\begin{bmatrix}
2ac-bd	\\
ad + bc
\end{bmatrix}
\]

Show this definition of vector multiplication makes $\mathbb{R}^{2}$ an $\mathbb{R}$-algebra.
\end{problem}

\begin{solution}
Let the notation be as given. Then to verify the $\mathbb{R}$-algebra we must check that the vector multiplication distributes across vector addition, and that it associates with scalar multiplication.

So let $u=(u_{1},u_{2}),v=(v_{1},v_{2}),w=(w_{1},w_{2}) \in \mathbb{R}^{2}$ and $a \in \mathbb{R}$.

\paragraph{Distributivity}
\begin{align*}
u\cdot(v+w) &=
\begin{bmatrix}
u_{1}	\\
u_{2}
\end{bmatrix}
\cdot
\begin{bmatrix}
v_{1}+w_{1}	\\
v_{2}+w_{2}
\end{bmatrix}
\\
&=
\begin{bmatrix}
2u_{1}(v_{1}+w_{1}) - u_{2}(v_{2}+w_{2})	\\
u_{1}(v_{2}+w_{2}) + u_{2}(v_{1}+w_{1})
\end{bmatrix}
\\
&=
\begin{bmatrix}
2u_{1}v_{1} - u_{2}v_{2}	\\
u_{1}v_{2} + u_{2}v_{2}
\end{bmatrix}
+
\begin{bmatrix}
2u_{1}w_{1} - u_{2}w_{2}	\\
u_{1}w_{2} + u_{2}w_{1}
\end{bmatrix}
\\
&=
\left(
\begin{bmatrix}
u_{1}	\\
u_{2}
\end{bmatrix}
\cdot
\begin{bmatrix}
v_{1}	\\
v_{2}
\end{bmatrix}
\right)
+
\left(
\begin{bmatrix}
u_{1}	\\
u_{2}
\end{bmatrix}
\cdot
\begin{bmatrix}
w_{1}	\\
w_{2}
\end{bmatrix}
\right)
\\
&= (u\cdot v)+(u\cdot w)
\end{align*}

And since 
\[ 
\begin{bmatrix}
u_{1}	\\
u_{2}
\end{bmatrix}
\cdot
\begin{bmatrix}
v_{1}	\\
v_{2}
\end{bmatrix}
=
\begin{bmatrix}
2u_{1}v_{1} - u_{2}v_{2}	\\
u_{1}v_{2} + u_{2}v_{1}
\end{bmatrix}
=
\begin{bmatrix}
2v_{1}u_{1} - v_{2}u_{2}	\\
v_{1}u_{2} + v_{2}u_{1}
\end{bmatrix}
=
\begin{bmatrix}
v_{1}	\\
v_{2}
\end{bmatrix}
\cdot
\begin{bmatrix}
u_{1}	\\
u_{2}
\end{bmatrix}
\]
Then vector multiplication is commutative, and vector multiplication distributes over vector addition from the left or the right:
\begin{align*}
u\cdot(v+w) &= (u\cdot v) + (u\cdot w)	\\
(v+w)\cdot u &= (v\cdot u) + (w\cdot u)
\end{align*}

\paragraph{Associativity}
\begin{align*}
a(u\cdot v) &= a
\begin{bmatrix}
2u_{1}v_{1} - u_{2}v_{2}	\\
u_{1}v_{2} + u_{2}v_{1}
\end{bmatrix}
\\
&= 
\begin{bmatrix}
a(2u_{1}v_{1} - u_{2}v_{2})	\\
a(u_{1}v_{2} + u_{2}v_{1})
\end{bmatrix}
\\
&= a
\begin{bmatrix}
2au_{1}v_{1} - au_{2}v_{2}	\\
au_{1}v_{2} + au_{2}v_{1}
\end{bmatrix}
\\
&= (au)\cdot v
\end{align*}
And thus vector multiplication associates properly, and with the distribution property above this verifies $\mathbb{R}^{2}$ as an $\mathbb{R}$-algebra.
\end{solution}

\end{document}