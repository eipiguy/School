\documentclass{article}

\usepackage{amsmath,amsfonts,amssymb,amsthm}
\usepackage{listings,color}
\usepackage{graphicx}


% Opening
\title{Linear Algebra HW4\\
Exercises 476, 516, 542, 585, 622, 670, 683, 760, 880}
\author{Neal D. Nesbitt}

\begin{document}
\maketitle

\theoremstyle{definition}
\newtheorem{problem}{Problem}[section]
\newtheorem{solution}{Solution}[problem]
\renewcommand{\thesolution}{\theproblem}

\setcounter{section}{9}
\setcounter{problem}{475}
\begin{problem}
Let $F$ be a field. Find a matrix $A \in \mathcal{M}_{4\times 4}(F)$ satisfying $A^{4}=I\ne A^{3}$.
\end{problem}

\begin{solution}
Let the notation be as given in the problem. 

Examine
$\begin{bmatrix}
0 & 1 & 0 & 0 \\
0 & 0 & 1 & 0 \\
0 & 0 & 0 & 1 \\
1 & 0 & 0 & 0
\end{bmatrix}$:
\begin{align*}
\begin{bmatrix}
0 & 1 & 0 & 0 \\
0 & 0 & 1 & 0 \\
0 & 0 & 0 & 1 \\
1 & 0 & 0 & 0
\end{bmatrix}
^{2}
&=
\begin{bmatrix}
0 & 0 & 1 & 0 \\
0 & 0 & 0 & 1 \\
1 & 0 & 0 & 0 \\
0 & 1 & 0 & 0
\end{bmatrix}\\
\begin{bmatrix}
0 & 1 & 0 & 0 \\
0 & 0 & 1 & 0 \\
0 & 0 & 0 & 1 \\
1 & 0 & 0 & 0
\end{bmatrix}
^{3}
&=
\begin{bmatrix}
0 & 0 & 0 & 1 \\
1 & 0 & 0 & 0 \\
0 & 1 & 0 & 0 \\
0 & 0 & 1 & 0
\end{bmatrix}\\
\begin{bmatrix}
0 & 1 & 0 & 0 \\
0 & 0 & 1 & 0 \\
0 & 0 & 0 & 1 \\
1 & 0 & 0 & 0
\end{bmatrix}
^{4} &= I
\end{align*}
and thus this matrix fulfills the required conditions.
\end{solution}

\setcounter{problem}{515}
\begin{problem}
Let $n$ be a positive integer, let $F$ be a field, and let $A \in \mathcal{M}_{n\times n}(F)$ satisfy the condition $A=AA^{T}$. Show that $A^{2}=A$.
\end{problem}

\begin{solution}
Let our notation be as given in the problem. Then $A=AA^{T}$ implies $A^{T}=(AA^{T})^{T}=(A^{T})^{T}A^{T}=AA^{T}=A$. Thus $A=AA^{T}=A^{2}$.
\end{solution}

\setcounter{problem}{541}
\begin{problem}
Let $n$ and $p$ be positive integers and let $F$ be a field. Let $A \in \mathcal{M}_{n\times n}(F)$ and let $B,C \in \mathcal{M}_{n\times p}(F)$ be matrices satisfying the condition that $A$ and $(I+C^{T}A^{-1}B)$ are nonsingular. Show that $A+BC^{T}$ is non-singular, and that
\[ 
\left( A +BC^{T} \right)^{-1} = A^{-1} - A^{-1}B\left( I+C^{T}A^{-1}B \right)^{-1}C^{T}A^{-1}
\]
\end{problem}

\begin{solution}
Let our notation be as given in the problem. Then since $(I+C^{T}A^{-1}B)$ is non-singular, by Proposition 9.1 on pg 154, $(I+BC^{T}A^{-1})$ is also nonsingular. So since $A$ is also non-singular, then their product will be as well by the same proposition:
\[ (I+BC^{T}A^{-1})A = A+BC^{T} \]

\paragraph{}
Also by the same proposition
\begin{align*}
[A+BC^{T}]^{-1} &= [(I+BC^{T}A^{-1})A]^{-1} \\
&= A^{-1}(I+BC^{T}A^{-1})^{-1} \\
&= A^{-1}( I - B(I+C^{T}A^{-1}B)^{-1}C^{T}A^{-1} ) \\
&= A^{-1} - A^{-1}B(I+C^{T}A^{-1}B)^{-1}C^{T}A^{-1}
\end{align*}
\end{solution}

\setcounter{section}{10}
\setcounter{problem}{584}
\begin{problem}
Find all solutions to the system
\[
\begin{bmatrix}
1 & 2 & 3 & 4 \\
2 & 1 & 2 & 3 \\
3 & 2 & 1 & 2 \\
4 & 3 & 2 & 1
\end{bmatrix}
\begin{bmatrix}
X_{1}\\ X_{2}\\ X_{3}\\ X_{4}
\end{bmatrix}
=
\begin{bmatrix}
5\\ 1\\ 1\\ -5
\end{bmatrix}
\]
\end{problem}

\begin{solution}
Using Gaussian elimination reduces the augmented matrix as follows:
\begin{align*}
\left[
\begin{array}{c c c c | c}
1 & 2 & 3 & 4 & 5\\
2 & 1 & 2 & 3 & 1\\
3 & 2 & 1 & 2 & 1\\
4 & 3 & 2 & 1 & -5
\end{array}
\right]
&
\to\left[
\begin{array}{c c c c | c}
1 & 2 & 3 & 4 & 5\\
0 & -3 & -4 & -5 & -9\\
0 & -4 & -8 & -10 & -14\\
0 & -5 & -10 & -15 & -25
\end{array}
\right]
\\
\to\left[
\begin{array}{c c c c | c}
1 & 2 & 3 & 4 & 5\\
0 & -3 & -4 & -5 & -9\\
0 & 0 & -8/3 & -10/3 & -2\\
0 & 0 & -10/3 & -20/3 & -10
\end{array}
\right]
&
\to\left[
\begin{array}{c c c c | c}
1 & 2 & 3 & 4 & 5\\
0 & -3 & -4 & -5 & -9\\
0 & 0 & -8/3 & -10/3 & -2\\
0 & 0 & -10/3 & -20/3 & -10
\end{array}
\right]\\
\to\left[
\begin{array}{c c c c | c}
1 & 2 & 3 & 4 & 5\\
0 & -3 & -4 & -5 & -9\\
0 & 0 & -8/3 & -10/3 & -2\\
0 & 0 & 0 & -5/2 & -15/2
\end{array}
\right]
\end{align*}
Then by back substitution
\[
\begin{bmatrix}
X_{1}\\ X_{2}\\ X_{3}\\ X_{4}
\end{bmatrix}
=
\begin{bmatrix}
-2\\ 2\\ -3\\ 3
\end{bmatrix}
\]
\end{solution}

\setcounter{problem}{621}
\begin{problem}
Let $k$ and $n$ be positive integers and let $F$ be a field. For matrices $A,B \in \mathcal{M}_{k\times n}(F)$, show that the rank of $A+B$ is no more than the sum of the ranks of $A$ and of $B$.
\end{problem}

\begin{solution}
Let our notation be as given in the problem, and let us call $a\le n$ the rank of $A$, and $b\le n$ the rank of $B$. 

\paragraph{}
Then the column space of each matrix has a respective basis, $\mathcal{B}(A) = \{ u_{1},\cdots,u_{a} \}$ and $\mathcal{B}(B) = \{ v_{1},\cdots,v_{b} \}$. The column space of $A+B$ can have at most $a+b$ vectors, where this is only if all of the basis vectors for $A\cup B$ are linearly independent: $\mathcal{B}(A) \cup \mathcal{B}(B) =  \{ u_{1},\cdots,u_{a} ,v_{1},\cdots,v_{b} \}$.

When they are independent like this, the rank of $A+B$ is its maximum: the number of basis vectors, $a+b$, which is no more than the sum of the ranks of $A$ and $B$.
\end{solution}

\setcounter{section}{11}
\setcounter{problem}{669}
\begin{problem}
Let $n$ be a positive integer and let $A=[a_{ij}] \in \mathcal{M}_{n\times n}(\mathbb{R})$ be the matrix defined by
\[
a_{ij} =
\left\{
\begin{array}{l l}
0 & \text{if }i=j\\
1 & \text{otherwise}
\end{array}
\right.
\]
Calculate $|A|$.
\end{problem}

\begin{solution}
Let our notation be that same as given in the problem.

Then since for every $j\in\mathbb{N}$ such that $j\le n$, we know $a_{jj}=0$, then by Proposition 11.11 on pg 231, 
\[
|A| = \sum_{j=1}^{n}(-1)^{j+j}a_{jj}\cdot|A_{ij}| = \sum_{j=1}^{n} 0\cdot|A_{ij}| = 0
\]
\end{solution}

\setcounter{problem}{682}
\begin{problem}
Let $F$ be a field, let $n$ be a positive integer, and let $A = [a_{ij}] \in \mathcal{M}_{n\times n}(F)$ be nonsingular. Show that $\text{adj}(\text{adj}(A)) = |A|^{n-2}A$.
\end{problem}

\begin{solution}
Let our notation be the same as given in the problem. Then notice that $A$ non-singular means that $A^{-1}$ is well defined. 

\paragraph{}
Now look at Proposition 11.13 on pg 235. Since $A$ is nonsingular, then 
\begin{align*}
A[\text{adj}(A)] &= |A|I \\
\text{adj}(A) &= |A|A^{-1} \\
\text{adj}(\text{adj}(A)) &= \left| |A|A^{-1} \right| (|A|A^{-1})^{-1}
\end{align*}
Now we take note that multiplying a whole matrix by a scalar $c\in F$ is equivalent to multiplying each column by the same scalar:
\[
cA = \left( \prod_{i=1}^{n} E_{i;c} \right)A
\]
and thus by the properties of determinant functions and elementary matrices:
\begin{align*}
\text{adj}(\text{adj}(A)) &= \left| |A|A^{-1} \right| (|A|A^{-1})^{-1} \\
&= \left| \left( \prod_{i=1}^{n} E_{i;|A|} \right)A^{-1} \right| (\left( \prod_{i=1}^{n} E_{i;|A|} \right)A^{-1})^{-1} \\
&= |A|^{n}|A^{-1}| A \left(\prod_{i=1}^{n} E_{i;|A|}\right)^{-1} \\
&= |A|^{n-1} A \prod_{i=n}^{1} E_{i;|A|}^{-1} \\
&= |A|^{n-1} A \prod_{i=n}^{1} E_{i;|A|^{-1}} \\
&= |A|^{n-1} A |A|^{-1} \\
\text{adj}(\text{adj}(A)) &= |A|^{n-2} A
\end{align*}
\end{solution}

\setcounter{section}{12}
\setcounter{problem}{759}
\begin{problem}
Fine the eigenvalues of the matrix
$\begin{bmatrix}
5 & 6 & -3 \\
-1 & 0 & 1 \\
2 & 2 & -1
\end{bmatrix}
\in \mathcal{M}_{3\times 3}(\mathbb{R})$
and, for each such eigenvalue, find the associated eigenspace.
\end{problem}

\begin{solution}
If we call the matrix above $A$, we can compute the characteristic polynomial by letting $\lambda\in\mathbb{C}$ be an eigenvalue, and setting the determinant of $|A-\lambda I|$ to zero:
\begin{align*}
|A-\lambda I| &= 
\begin{vmatrix}
5-\lambda & 6 & -3 \\
-1 & -\lambda & 1 \\
2 & 2 & -1-\lambda
\end{vmatrix}
\\
&= (5-\lambda)(1+\lambda)\lambda +12 +6 -\left( 6\lambda +6(1+\lambda) +2(5-\lambda) \right) \\
&= (5+4\lambda-\lambda^{2})\lambda +18 -\left( 6\lambda +6 +6\lambda +10 -2\lambda \right) \\
&= (5+4\lambda-\lambda^{2})\lambda +18 -( 10\lambda +16 ) \\
&= (5+4\lambda-\lambda^{2})\lambda -10\lambda +2 \\
&= 5\lambda +4\lambda^{2} -\lambda^{3} -10\lambda +2 \\
&= 2 -5\lambda +4\lambda^{2} -\lambda^{3} \\
&= (2-\lambda)(1 -2\lambda +\lambda^{2}) \\
&= (2-\lambda)(1 -\lambda)^{2} = 0
\end{align*}
and thus $A$ has eigenvalues $\lambda = 1,2$. We then consider these separately and find their respective eigenspaces.

\paragraph{}
So begin with $\lambda = 1$, and to find the eigenspace, we find the equivalent nullspace of $A-\lambda I$:
\begin{align*}
\left[
\begin{array}{c c c | c}
4 & 6 & -3 & 0\\
-1 & -1 & 1 & 0 \\
2 & 2 & -2 & 0
\end{array}
\right]
&\to
\left[
\begin{array}{c c c | c}
4 & 6 & -3 & 0\\
0 & 1/2 & 1/4 & 0 \\
0 & 0 & 0 & 0
\end{array}
\right]
\end{align*}
We then let $v\in\mathbb{R}^{3}$ be in the nullspace of $A-\lambda I$, and take $v_{3}\in\mathbb{R}$ to be a free variable. This allows us to see:
\begin{align*}
\frac{v_{2}}{2} + \frac{v_{3}}{4} &= 0 \\
\frac{v_{2}}{2} &= \frac{-v_{3}}{4} \\
v_{2} &= \frac{-v_{3}}{2} \\
\end{align*}
and 
\begin{align*}
4v_{1} + 6v_{2} -3v_{3} &= 0 \\
4v_{1} &= -6v_{2} +3v_{3} \\
4v_{1} &= 3v_{3} +3v_{3} \\
4v_{1} &= 6v_{3} \\
v_{1} &= \frac{3v_{3}}{2} \\
\end{align*}

Then we have that vectors in the eigenspace for $\lambda = 1$ are of the form
\[
v_{3}
\begin{bmatrix}
3/2\\ -1/2\\ 1
\end{bmatrix}
=
\frac{v_{3}}{2}
\begin{bmatrix}
3\\ -1\\ 2
\end{bmatrix} 
\]

\paragraph{}
So now let us consider $\lambda = 2$. The associated $A-\lambda I$ reduces as follows:
\begin{align*}
\left[
\begin{array}{c c c | c}
3 & 6 & -3 & 0 \\
-1 & -2 & 1 & 0 \\
2 & 2 & -3 & 0
\end{array}
\right]
&\to
\left[
\begin{array}{c c c | c}
3 & 6 & -3 & 0 \\
0 & 0 & 0 & 0 \\
0 & -2 & -1 & 0
\end{array}
\right]
\end{align*}
and for $v\in\mathbb{R}^{3}$ in the nullspace of $A-\lambda I$, we take $v_{2}$ to be a free variable this time so that:
\begin{align*}
-2v_{2} -v_{3} &= 0 \\
v_{3} &= -2v_{2}
\end{align*}
and then
\begin{align*}
3v_{1} +6v_{2} -3v_{3} &= 0 \\
3v_{1} &= -6v_{2} +3v_{3} \\
3v_{1} &= -6v_{2} -6v_{2} \\
3v_{1} &= -12v_{2} \\
v_{1} &= -4v_{2} \\
\end{align*}
Showing that vectors in the eigenspace for $\lambda = 2$ are of the form:
\[
v_{2}
\begin{bmatrix}
-4\\ 1\\ -2
\end{bmatrix}
=
-v_{2}
\begin{bmatrix}
4\\ -1\\ 2
\end{bmatrix}
\]
\end{solution}

\setcounter{section}{13}
\setcounter{problem}{879}
\begin{problem}
Let $A \in \mathcal{M}_{n\times n}(\mathbb{R})$ differ from $I$ and $O$. If $A$ is idempotent, show that its Jordan canonical form is a diagonal matrix.
\end{problem}

\begin{solution}
Let $A$ be as given in the problem.

\paragraph{}
Then by Proposition 13.7, there is some matrix $B\in \mathcal{M}_{n\times n}(\mathbb{R})$ expressed in canonical form that is similar to $A$. So then if we take the resulting $P \in \mathcal{M}_{n\times n}(\mathbb{R})$ where $A = P^{-1}BP$, we can then use the idepotence of $A$ to see
\begin{align*}
A^{2} &= P^{-1}BPP^{-1}BP = A = P^{-1}BP \\
&= P^{-1}B^{2}P = P^{-1}BP
\end{align*}
which implies that $B^{2}=B$, showing $B$ is also idepotent.

\paragraph{}
Since $B$ is in canonical form, then there is an $m\in\mathbb{N}$ where for any $j\in\mathbb{N}$ such that $j\le m$, we have $c_{j}\in\mathbb{R}$ where blocks of the form
\[
B_{j} = 
\begin{bmatrix}
c_{j} & 0 & 0 & \cdots & 0 \\
1 & c_{j} & 0 & \cdots & 0 \\
0 & 1 & c_{j} & \cdots & 0 \\
\vdots & \vdots & \ddots & \ddots & \vdots \\
0 & 0 & \cdots & 1 & c_{j}
\end{bmatrix}
\]
make up the matrix $B$ in the following way:
\[
B =
\begin{bmatrix}
B_{1} & O & \cdots & O \\
0 & B_{2} & \cdots & O \\
\vdots & \vdots & \ddots & \vdots \\
O & O & \cdots & B_{m}
\end{bmatrix}
\]

So now we notice that 
\[
B^{2} =
\begin{bmatrix}
B_{1}^{2} & O & \cdots & O \\
0 & B_{2}^{2} & \cdots & O \\
\vdots & \vdots & \ddots & \vdots \\
O & O & \cdots & B_{m}^{2}
\end{bmatrix}
\]
where for each $j$, the idepotence of $B$ implies that 
\begin{align*}
B_{j}^{2} &= B_{j}\\
\begin{bmatrix}
c_{j} & 0 & 0 & \cdots & 0 \\
1 & c_{j} & 0 & \cdots & 0 \\
0 & 1 & c_{j} & \cdots & 0 \\
\vdots & \vdots & \ddots & \ddots & \vdots \\
0 & 0 & \cdots & 1 & c_{j}
\end{bmatrix}
\begin{bmatrix}
c_{j} & 0 & 0 & \cdots & 0 \\
1 & c_{j} & 0 & \cdots & 0 \\
0 & 1 & c_{j} & \cdots & 0 \\
\vdots & \vdots & \ddots & \ddots & \vdots \\
0 & 0 & \cdots & 1 & c_{j}
\end{bmatrix}
&= 
\begin{bmatrix}
c_{j} & 0 & 0 & \cdots & 0 \\
1 & c_{j} & 0 & \cdots & 0 \\
0 & 1 & c_{j} & \cdots & 0 \\
\vdots & \vdots & \ddots & \vdots \\
0 & 0 & \cdots & 1 & c_{j}
\end{bmatrix}
\\
\begin{bmatrix}
c_{j}^{2} & 0 & 0 & 0 & \cdots & 0 \\
2c_{j} & c_{j}^{2} & 0 & 0 & \cdots & 0 \\
1 & 2c_{j} & c_{j}^{2} & 0 & \cdots & 0 \\
0 & 1 & 2c_{j} & c_{j}^{2} & \cdots & 0 \\
\vdots & \vdots & \ddots & \ddots & \ddots & \vdots \\
0 & 0 & \cdots & 1 & 2c_{j} & c_{j}^{2}
\end{bmatrix}
&= 
\begin{bmatrix}
c_{j} & 0 & 0 & \cdots & 0 \\
1 & c_{j} & 0 & \cdots & 0 \\
0 & 1 & c_{j} & \cdots & 0 \\
\vdots & \vdots & \ddots & \vdots \\
0 & 0 & \cdots & 1 & c_{j}
\end{bmatrix}
\end{align*}
Since this implies that each $c_{j}^{2} = c_{j}$ and $2c_{j} = 1$, then $c_{j}$ must be simultaneously $1/2$ and either $1$ or $0$. This is a contradiction unless our blocks are only one row by one column:
\[ B_{j} = [c_{j}] \]
showing that $B$ is a diagonal matrix.
\end{solution}

\end{document}