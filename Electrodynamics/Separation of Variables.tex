%###############################################################################
% Packages and Format Definitions
\documentclass{article}	% This document is formatted as an article

\usepackage{amsmath,amsfonts,amssymb,amsthm}	% American Mathematical Society formatting
\usepackage{enumerate}							% Format items and labels in lists
\usepackage{arydshln}							% Use dashed or dotted lines in arrays
\usepackage{listings,color}						% Insert code segments and define custom colors
\usepackage{graphicx}							% Insert external graphics into the document

\definecolor{dkgreen}{rgb}{0,0.6,0}		% Dark Green rather than the stock neon green
\definecolor{gray}{rgb}{0.5,0.5,0.5}	% Gray for comments and things to be made less visible
\definecolor{mauve}{rgb}{0.58,0,0.82}	% Mauve is like a muted magenta; stock magenta is too bright

%###############################################################################
% Opening Information:

\title{ Field Arithmetic\\
and
Solving Laplace's Equation\\
Using Separation of Variables
}
\author{Neal D. Nesbitt}

%###############################################################################
% Initialization:

\theoremstyle{definition}
\newtheorem{definition}{Definition}[section]

\begin{document}
\maketitle

%===============================================================================
% Main Document:

\section{Vector Fields Operations}

\begin{definition}{Scalar Field}
content...
\end{definition}

\begin{definition}{Vector Field}
content...
\end{definition}

\begin{definition}{The Gradient of a Scalar Field}
Let $n\in\mathbb{N}$ and $Psi: D\subset\mathbb{R}^{n} \to R\subset\mathbb{R}$ be a scalar field on $D$.
\end{definition}

\begin{definition}{The Divergence of a Vector Field}
content...
\end{definition}

\begin{definition}{ The \underline{Laplacian} of a Scalar Field\\\\}
If $\Psi:\mathbb{R}^{n}\to\mathbb{R}$ is a scalar field, then $\forall (x_{1},\cdots,x_{n})\in\mathbb{R}$ the \underline{Laplacian of $\Psi$} at $(x_{1},\cdots,x_{n})$ is:
\begin{multline}
\nabla^{2}\Psi(x_{1},\cdots,x_{n}) = \nabla \cdot \nabla \Psi(x_{1},\cdots,x_{n}) \\
= \nabla \cdot \left[ \sum_{k=1}^{n}\frac{\partial}{\partial x_{k}} \left[ \Psi(x_{1},\dots,x_{n}) \right] \mathbf{\hat{x}}_{k} \right] = \sum_{k=1}^{n}\frac{\partial^{2}}{\partial x_{k}^{2}} \left[ \Psi(x_{1},\dots,x_{n}) \right]
\end{multline}
\end{definition}

\section{Cartesian Coordinates - $\mathbf{x}=(x_{1},\cdots,x_{n})$}
Let $\Phi:\mathbb{R}^{n}\to\mathbb{R}$ be a scalar field, and recall the Laplacian of $\Phi$:
\[ \nabla^{2}\Phi = \nabla \cdot \nabla \Phi = \nabla \cdot \left[ \sum_{k=1}^{n}\frac{\partial}{\partial x_{k}} \left[ \Phi(x_{1},\dots,x_{n}) \right] \mathbf{\hat{x}_{k}} \right] = \sum_{k=1}^{n}\frac{\partial^{2}}{\partial x_{k}^{2}} \left[ \Phi(x_{1},\dots,x_{n}) \right] \]

The crux of the separation of variables technique relies on the assumption that $\Phi$ can be written as the product of $n$ independent scalar functions: $\exists X_{1}(x_{1}),\dots,X_{n}(x_{n}):\mathbb{R}\to\mathbb{R} \text{ st. } \Phi(x_{1},\dots,x_{n}) =  X_{1}(x_{1}) \dots X_{n}(x_{n})$.

Then if we substitute this into the Laplacian we can find
\begin{align*}
\nabla^{2}\Phi &= \sum_{k=1}^{n}\frac{\partial^{2}}{\partial x_{k}^{2}} \left[ \Phi(x_{1},\dots,x_{n}) \right] = \sum_{k=1}^{n}\frac{\partial^{2}}{\partial x_{k}^{2}} X_{1}(x_{1}) \dots X_{n}(x_{n})\\
&= \sum_{k=1}^{n} X_{1}(x_{1}) \dots X_{k-1}(x_{k-1}) X_{k+1}(x_{k+1}) \dots X_{n}(x_{n}) \frac{d^{2}}{dx_{k}^{2}} X_{k}(x_{k})\\
\frac{\nabla^{2}\Phi }{\Phi} &= \sum_{k=1}^{n} \frac{1}{X_{k}(x_{k})} \frac{d^{2}}{dx_{k}^{2}} X_{k}(x_{k})
\end{align*}

Then each of the components of the sum are functions of an independent variable, and since the terms must always sum to the same potential, each must be constant. We can name these constants $\alpha_{1},\cdots,\alpha_{n}$ so that we can separate the sum and solve each differential equation separately.
\begin{align*}
\frac{\nabla^{2}\Phi }{\Phi} = \sum_{k=1}^{n} \frac{1}{X_{k}(x_{k})} \frac{d^{2}}{dx_{k}^{2}} X_{k}(x_{k}) &= \sum_{k=1}^{n} \alpha_{k}\\
\frac{1}{X_{k}(x_{k})} \frac{d^{2}}{dx_{k}^{2}} X_{k}(x_{k}) &= \alpha_{k}\\
\frac{d^{2}}{dx_{k}^{2}} X_{k}(x_{k}) &= \alpha_{k}X_{k}(x_{k})\\
X_{k}(x_{k}) &= \sum_{j=1}^{\infty} \tilde{A}_{(k,j)}e^{\pm x_{k}\sqrt{\alpha_{k}}}\\
X_{k}(x_{k}) &= A_{k}e^{+x_{k}\sqrt{\alpha_{k}}} + B_{k}e^{-x_{k}\sqrt{\alpha_{k}}}\\
\Phi(x_{1},\cdots,x_{n}) &= \prod_{k=1}^{n} \left( A_{k}e^{+x_{k}\sqrt{\alpha_{k}}} + B_{k}e^{-x_{k}\sqrt{\alpha_{k}}} \right)
\end{align*}

\section{Polar Coordinates - $\mathbf{x}=(r\cos\phi,r\sin\phi)$}
Consider Laplace's equation now in polar coordinates about the x-y plane $\mathbf{x} = (r\cos\phi,r\sin\phi)$ such that we have $\Phi(\mathbf{x})=\Phi(r\cos\phi,r\sin\phi)=\Phi_{p}(r,\phi)=\Phi_{p}(\mathbf{x_{p}})$. Then Laplace's equation in polar coordinates becomes:
\begin{align*}
\nabla^{2} \Phi &= 0\\
\frac{1}{r} \frac{\partial}{\partial r} \left( r \frac{\partial \Phi}{\partial r} \right) + \frac{1}{r^{2}} \frac{\partial^{2} \Phi}{\partial \phi^{2}} &= 0
\end{align*}
If we then assume $\Phi=R(r)F(\phi)$ for some scalar functions $R,F$, we can use the product rule to see
\begin{align*}
\frac{1}{r} \frac{\partial}{\partial r} \left( r \frac{\partial}{\partial r} \left[ R(r)F(\phi) \right] \right) + \frac{1}{r^{2}} \frac{\partial^{2}}{\partial \phi^{2}} \left[ R(r)F(\phi) \right] &= 0\\
\frac{F(\phi)}{r} \frac{\partial}{\partial r} \left( r \frac{\partial R}{\partial r}(r) \right) + \frac{R(r)}{r^{2}} \frac{\partial^{2} F}{\partial \phi^{2}}(\phi) &= 0\\
\frac{F(\phi)}{r} \left( \frac{\partial R}{\partial r}(r) +  r \frac{\partial^{2} R}{\partial r^{2}}(r) \right) + \frac{R(r)}{r^{2}} \frac{\partial^{2} F}{\partial \phi^{2}}(\phi) &= 0\\
\end{align*}
Then multiplying through by $r^{2}/\Phi = r^{2}RF$ will separate the remaining variables:
\[ \frac{r}{R(r)} \left( \frac{\partial R}{\partial r}(r) +  r \frac{\partial^{2} R}{\partial r^{2}}(r) \right) + \frac{1}{F(\phi)} \frac{\partial^{2} F}{\partial \phi^{2}}(\phi) = 0\\ \]
We then have two differential equations that each depend on a different variable, and add to zero. We can call the constants $\pm\alpha$ and then solve each component of the sum as a separate ordinary differential equations:
\[ \frac{r}{R(r)} \left( \frac{\partial R}{\partial r}(r) + r \frac{\partial^{2} R}{\partial r^{2}}(r) \right) + \frac{1}{F(\phi)} \frac{\partial^{2} F}{\partial \phi^{2}}(\phi) = \alpha + (-\alpha) \]
\begin{align*}
\frac{r}{R(r)} \left( \frac{dR}{dr}(r) + r \frac{d^{2} R}{dr^{2}}(r) \right) &= \alpha
&
\frac{-1}{F(\phi)} \frac{d^{2} F}{d\phi^{2}}(\phi) &= \alpha\\
\frac{dR}{dr}(r) + r \frac{d^{2} R}{dr^{2}}(r) &= \frac{\alpha}{r} R(r)
&
\frac{d^{2} F}{d\phi^{2}}(\phi) &= -\alpha F(\phi)\\
\frac{d^{2} R}{d r^{2}}(r) &= \frac{-1}{r} \frac{d R}{d r}(r) + \frac{\alpha}{r^{2}} R(r)
&
\frac{d^{2} F}{d \phi^{2}}(\phi) &= -\alpha F(\phi)\\
\end{align*}
Here for constants $A$ and $B$, $F$ has solutions
\[ F(\phi)= Ae^{i\phi\sqrt{\alpha}} + Be^{-i\phi\sqrt{\alpha}} \]

\paragraph{}
Then we look for solutions for $R(r)$ multiplying through by $r^{2}$
\begin{align*}
\frac{d^{2} R}{dr^{2}}(r) &= \frac{-1}{r} \frac{dR}{dr}(r) + \frac{\alpha}{r^{2}} R(r)\\
0 &= r^{2}\frac{d^{2} R}{dr^{2}}(r) +r \frac{dR}{dr}(r) -\alpha R(r)\\
\end{align*}
This form of differential polynomial is called an Euler differential equation and can be solved if we consider some $z$ such that $r=e^{z}\implies z=\ln r$. With this transformation of variables we will have a different equation, call it $Z(z)=Z(\ln r)=R(r)$ where we can then reduce our equation.
\begin{align*}
0 &= r^{2}\frac{d^{2} R}{dr^{2}}(r) +r \frac{dR}{dr}(r) -\alpha R(r)\\
0 &= r^{2}\frac{d^{2} Z}{dr^{2}}(\ln r) +r \frac{dZ}{dr}(\ln r) -\alpha Z(\ln r)\\
0 &= r^{2}\frac{d}{dr}\left( \frac{1}{r} \frac{dZ}{dz}(\ln r) \right) +\frac{dZ}{dz}(z) -\alpha Z(z)\\
0 &= r^{2}\left( \frac{-1}{r^{2}} \frac{dZ}{dz}(z) +\frac{1}{r^{2}} \frac{d{2} Z}{dz^{2}}(z) \right) +\frac{dZ}{dz}(z) -\alpha Z(z)\\
0 &= \frac{d^{2} Z}{dz^{2}}(z) -\alpha Z(z)\\
\frac{d^{2} Z}{dz^{2}}(z) &= \alpha Z(z)\\
\end{align*}

This is like the equation we had before and it gives us the solution
\[ Z(z) = \sum_{n} \mathcal{C}_{n}e^{\pm z\sqrt{\alpha}} = Ce^{z\sqrt{\alpha}} + De^{-z\sqrt{\alpha}} \]
so transforming back into $R$
\begin{align*}
R(r) &= Ce^{\ln r\sqrt{\alpha}} + De^{-\ln r\sqrt{\alpha}}\\
R(r) &= Cr^{\sqrt{\alpha}} + Dr^{-\sqrt{\alpha}}
\end{align*}

Which gives us our final form:
\[ \Phi (r,\phi) = R(r)F(\phi) = \left( Cr^{\sqrt{\alpha}} + Dr^{-r\sqrt{\alpha}} \right) \left( Ae^{i\phi\sqrt{\alpha}} + Be^{-i\phi\sqrt{\alpha}} \right) \]

\section{Cylindrical - $\mathbf{x}=(\rho\cos\phi,\rho\sin\phi,z)$}


\section{Spherical - $\mathbf{x}=(r\sin\theta\cos\phi,r\sin\theta\sin\phi,r\cos\theta)$}
Now let us consider Laplace's equation in spherical coordinates
\[ \frac{1}{r^{2}} \frac{\partial}{\partial r}(r^{2}\frac{d\Phi}{dr}) +\frac{1}{r^{2}\sin\theta} \frac{\partial}{\partial\theta}\left( \sin\theta \frac{\partial\Phi}{\partial\theta} \right) +\frac{1}{r^{2}\sin^{2}\theta} \frac{\partial^{2}\Phi}{\partial\phi^{2}} = 0 \]
Assume that we can represent
\[ \Phi(r,\phi,\theta)=R(r)F(\phi)T(\theta) \]
Substitute this into the equation, and begin the separation:
\begin{align*}
\frac{1}{r^{2}} \frac{\partial^{2}}{\partial r^{2}}(r^{2}R(r)F(\phi)T(\theta)) +\frac{1}{r^{2}\sin\theta} \frac{\partial}{\partial\theta}\left( \sin\theta \frac{\partial}{\partial\theta}R(r)F(\phi)T(\theta) \right) +\frac{1}{r^{2}\sin^{2}\theta} \frac{\partial^{2}}{\partial\phi^{2}}R(r)F(\phi)T(\theta) &= 0\\
\frac{F(\phi)T(\theta)}{r^{2}} \frac{\partial^{2}}{\partial r^{2}}(r^{2}R(r)) +\frac{R(r)F(\phi)}{r^{2}\sin\theta} \frac{\partial}{\partial\theta}\left( \sin\theta \frac{\partial}{\partial\theta}T(\theta) \right) +\frac{R(r)T(\theta)}{r^{2}\sin^{2}\theta} \frac{\partial^{2}}{\partial\phi^{2}}F(\phi) &= 0\\
\end{align*}
Then multiply through by 
$
\frac{r^{2}\sin^{2}\theta}{R(r)F(\phi)T(\theta)}
$
to 
\begin{align*}
\frac{\sin^{2}\theta}{R(r)} \frac{\partial^{2}}{\partial r^{2}}(r^{2}R(r)) +\frac{\sin\theta}{P(\theta)} \frac{\partial}{\partial\theta}\left( \sin\theta \frac{\partial}{\partial\theta}P(\theta) \right) +\frac{1}{Q(\phi)} \frac{\partial^{2}}{\partial\phi^{2}}Q(\phi) &= 0\\
\end{align*}
We then have $\phi$ isolated in a single term of the sum. It must be a constant, or else we change the value of one side of the sum without altering the other side. This would create a contradiction, so we call
\begin{align*}
\frac{1}{F(\phi)} \frac{\partial^{2}}{\partial\phi^{2}}F(\phi) &= M\\
\frac{\partial^{2}}{\partial\phi^{2}}F(\phi) &= MF(\phi)\\
F(\phi)&= Ae^{\phi\sqrt{M}} +Be^{-\phi\sqrt{M}}\\
\end{align*}
and notice that since we are working in spherical coordinates, $\forall k\in\mathbb{Z}$ and $\forall \phi\in\mathbb{R}$, $\phi + 2k\pi = \phi$. Thus 
\begin{align*}
Ae^{(\phi+2k\pi)\sqrt{M}} +Be^{-(\phi+2k\pi)\sqrt{M}} &= Ae^{\phi\sqrt{M}} +Be^{-\phi\sqrt{M}}\\
Ae^{(\phi+2k\pi)\sqrt{M}} -Ae^{\phi\sqrt{M}} &= Be^{-\phi\sqrt{M}} -Be^{-(\phi+2k\pi)\sqrt{M}}\\
Ae^{\phi\sqrt{M}} \left( e^{2k\pi\sqrt{M}} -1\right) &= Be^{-\phi\sqrt{M}} \left( 1 -e^{-2k\pi\sqrt{M}} \right)\\
Ae^{\phi\sqrt{M}} \left( 1 -e^{-2k\pi\sqrt{M}}\right) e^{2k\pi\sqrt{M}} &= Be^{-\phi\sqrt{M}} \left( 1 -e^{-2k\pi\sqrt{M}} \right)\\
Ae^{\phi\sqrt{M}} e^{2k\pi\sqrt{M}} &= Be^{-\phi\sqrt{M}}\\
e^{2\phi\sqrt{M}} e^{2k\pi\sqrt{M}} &= B/A\\
e^{(\phi+k\pi)2\sqrt{M}} &= B/A\\
\end{align*}



\begin{align*}
\frac{\sin^{2}\theta}{R(r)} \frac{\partial^{2}}{\partial r^{2}}(r^{2}R(r)) +\frac{\sin\theta}{P(\theta)} \frac{\partial}{\partial\theta}\left( \sin\theta \frac{\partial}{\partial\theta}P(\theta) \right) -m^{2} &= 0\\
\frac{1}{R(r)} \frac{\partial^{2}}{\partial r^{2}}(r^{2}R(r)) +\frac{1}{P(\theta)\sin\theta} \frac{\partial}{\partial\theta}\left( \sin\theta \frac{\partial}{\partial\theta}P(\theta) \right) -\frac{m^{2}}{\sin^{2}\theta} &= 0\\
\end{align*}

We then have the radial component $R(r)$ isolated, so again, to keep the sum constant, we can set the radial part to a constant and solve for it separately:
\begin{align*}
\frac{1}{R(r)} \frac{d^{2}}{dr^{2}}(r^{2}R(r)) &= L\\
\frac{1}{R(r)} \frac{d}{dr}\left(2rR(r)+r^{2}\frac{dR}{dr}(r)\right) &= L\\
\frac{1}{R(r)} \left( 2R(r) +2r\frac{dR}{dr}(r) +2r\frac{dR}{dr}(r) +r^{2}\frac{d^{2}R}{dr^{2}}(r) \right) &= L\\
\frac{1}{R(r)} \left( R(r) +4r\frac{dR}{dr}(r) +r^{2}\frac{d^{2}R}{dr^{2}}(r) \right) &= L\\
4r\frac{dR}{dr}(r) +r^{2}\frac{d^{2}R}{dr^{2}}(r) &= (L-1)R(r)\\
\end{align*}

This is a Euler differential equation that can again be solved by changing into exponential variables: $r=e^{z} \iff z=\ln r$
\[ R(r) = R(e^{z}) = Z(\ln r) = Z(z)\]
\begin{align*}
4r\frac{dZ}{dr}(z) +r^{2}\frac{d^{2}Z}{dr^{2}}(z) &= (L-1)Z(z)\\
4r\frac{dZ}{dz}\frac{dz}{dr}(z) +r^{2}\frac{d}{dr}\left( \frac{dZ}{dz}\frac{dz}{dr}(z) \right) &= (L-1)Z(z)\\
4\frac{dZ}{dz}(z) +r^{2}\frac{d}{dr}\left( \frac{1}{r}\frac{dZ}{dz}(z) \right) &= (L-1)Z(z)\\
4\frac{dZ}{dz}(z) +r^{2}\left( \frac{-1}{r^{2}}\frac{dZ}{dz}(z) +\frac{1}{r}\frac{dZ}{dz}\frac{dz}{dr}(z) \right) &= (L-1)Z(z)\\
4\frac{dZ}{dz}(z) +r^{2}\left( \frac{-1}{r^{2}}\frac{dZ}{dz}(z) +\frac{1}{r^{2}}\frac{dZ}{dz}(z) \right) &= (L-1)Z(z)\\
\frac{d^{2}Z}{dz^{2}}(z) &= -3\frac{dZ}{dz}(z) +(L-1)Z(z)\\
\end{align*}
So now take
\begin{align*}
\mathbf{Z}(z) &= 
\begin{bmatrix}
Z_{1}\\
Z_{2}
\end{bmatrix}
(z)
=
\begin{bmatrix}
dZ/dz\\
Z
\end{bmatrix}
(z)
&\implies
\dot{\mathbf{Z}}(z) &=
\begin{bmatrix}
d^{2}Z/dz^{2}\\
dZ/dz
\end{bmatrix}
(z)
=
\begin{bmatrix}
d^{2}Z/dz^{2}\\
Z_{1}
\end{bmatrix}
(z)
\end{align*}
Then $dZ_{2}/dz = Z_{2}$ and
\begin{align*}
\dot{\mathbf{Z}}(z) &=
\begin{bmatrix}
-3	&	(L-1)	\\
1	&	0
\end{bmatrix}
\mathbf{Z}(z)
\end{align*}

Let us now find the eigenvalues of this matrix by finding the characteristic polynomial:
\begin{align*}
\begin{vmatrix}
-3-\lambda	&	(L-1)	\\
1			&	-\lambda
\end{vmatrix}
= \lambda(\lambda+3)-(L-1) &= 0\\
\lambda^{2} +2\lambda +3 +(1-L) &= 0\\
\lambda^{2} +2\lambda -(2+L) &= 0
\end{align*}

We can make this a perfect square to simplify the math by choosing $l$ such that $L=l(l+1)$ and then
\[\lambda^{2} +2\lambda -(L+2) = 0\\\]
\begin{align*}
\lambda &= \frac{-2 \pm \sqrt{4 +4(L+2)}}{2}\\
&= \frac{-2 \pm 2\sqrt{1 +(L+2)}}{2}\\
&= -1 \pm \sqrt{L+3}\\
&= -1 \pm \sqrt{l(l+1)+3}\\
&= -1 \pm \sqrt{l^{2}+l+3}\\
\end{align*}


If we take $A$ to be this matrix of coefficients, this can be solved using eigenvalue decomposition where $\lambda_{1},\lambda_{2}$ are the eigenvalues of $A$, and $\mathbf{v}_{1},\mathbf{v}_{2}$ are the corresponding eigenvectors. Then if we take $P=[\mathbf{v}_{1},\mathbf{v}_{2}]$, we can do one more change of variables to find a separable solution:
\begin{align*}
\dot{\mathbf{Z}}(z) &= A \mathbf{Z}(z)\\
\mathbf{Y}(z) &= P^{-1}\mathbf{Z}(z)	& \implies	\mathbf{Z}(z) &= P\mathbf{Y}(z)\\
\dot{\mathbf{Y}}(z) &= P^{-1}\dot{\mathbf{Z}}(z) = P^{-1}A \mathbf{Z}(z) = P^{-1}AP \mathbf{Y}(z)
\end{align*}

\paragraph{}
So finding the eigenvalues of $A$ gives the characteristic polynomial
\[ 0 =
\begin{vmatrix}
-2-\lambda	&	(l-1)	\\
1			&	-\lambda
\end{vmatrix}
= \lambda(2+\lambda) - (l-1) = \lambda^{2} +2\lambda +(1-l)
\]
So that 
\[ \lambda = \frac{-2 \pm \sqrt{4-4(1-l)}}{2} = \frac{-2 \pm 2\sqrt{1-1+l}}{2} = -1 \pm \sqrt{l} \]
where substituting this into the second row gives
\[ v_{1} +(1\mp\sqrt{l})v_{2} = 0 \]
so that if we take our free variable to be $-v_{2}$
\begin{align*}
P^{-1} &= \frac{1}{2\sqrt{l}}
\begin{bmatrix}
-1	&	-(1+\sqrt{l})	\\
1	&	(1-\sqrt{l})
\end{bmatrix},
&
P &= 
\begin{bmatrix}
(1-\sqrt{l})	&	(1+\sqrt{l})	\\
-1				&	-1
\end{bmatrix}
\end{align*}

\begin{align*}
P^{-1}AP &= 
\frac{1}{2\sqrt{l}}
\begin{bmatrix}
-1	&	-(1+\sqrt{l})	\\
1	&	(1-\sqrt{l})
\end{bmatrix}
\begin{bmatrix}
-2		&	(l-1)	\\
1		&	0
\end{bmatrix}
\begin{bmatrix}
(1-\sqrt{l})	&	(1+\sqrt{l})	\\
-1				&	-1
\end{bmatrix}
\\
&= 
\frac{1}{2\sqrt{l}}
\begin{bmatrix}
2-(1+\sqrt{l})	&	-(l-1)	\\
-2+(1-\sqrt{l})	&	(l-1)
\end{bmatrix}
\begin{bmatrix}
(1-\sqrt{l})	&	(1+\sqrt{l})	\\
-1				&	-1
\end{bmatrix}
\\
&= 
\frac{1}{2\sqrt{l}}
\begin{bmatrix}
1-\sqrt{l}	&	-(l-1)	\\
-(1+\sqrt{l})	&	(l-1)
\end{bmatrix}
\begin{bmatrix}
(1-\sqrt{l})	&	(1+\sqrt{l})	\\
-1				&	-1
\end{bmatrix}
\\
&= 
\frac{1}{2\sqrt{l}}
\begin{bmatrix}
(1-\sqrt{l})^{2}+(l-1)	&	0	\\
0						&	-(1+\sqrt{l})^{2}-(l-1)
\end{bmatrix}
\\
&= 
\frac{1}{2\sqrt{l}}
\begin{bmatrix}
2(-\sqrt{l}+l)	&	0	\\
0						&	-2(l+\sqrt{l})
\end{bmatrix}
\\
&= 
\begin{bmatrix}
-1+\sqrt{l}	&	0	\\
0						&	-1-\sqrt{l}
\end{bmatrix}
\\
\end{align*}
Then since 
\[ \dot{\mathbf{Y}}(z) = P^{-1}AP \mathbf{Y}(z) = \begin{bmatrix}
-1+\sqrt{l}	&	0	\\
0			&	-1-\sqrt{l}
\end{bmatrix} 
\mathbf{Y}(z)
\]
We can solve the now separable equation and back substitute for our original variables:
\begin{align*}
\mathbf{Y}(z) &=
\begin{bmatrix}
e^{-z(1-\sqrt{l})}	&	0	\\
0					&	e^{-z(1+\sqrt{l})}
\end{bmatrix}
\mathbf{Y}(0)
\\
P^{-1}\mathbf{Z}(z) &=
\begin{bmatrix}
e^{-z(1-\sqrt{l})}	&	0	\\
0					&	e^{-z(1+\sqrt{l})}
\end{bmatrix}
P^{-1}\mathbf{Z}(0)
\\
\mathbf{Z}(z) &= P
\begin{bmatrix}
e^{-z(1-\sqrt{l})}	&	0	\\
0					&	e^{-z(1+\sqrt{l})}
\end{bmatrix}
P^{-1}\mathbf{Z}(0)
\\
\mathbf{Z}(z) &=
\frac{1}{2\sqrt{l}}
\begin{bmatrix}
(1-\sqrt{l})	&	(1+\sqrt{l})	\\
-1				&	-1
\end{bmatrix}
\begin{bmatrix}
e^{-z(1-\sqrt{l})}	&	0	\\
0						&	e^{-z(1+\sqrt{l})}
\end{bmatrix}
\begin{bmatrix}
-1	&	-(1+\sqrt{l})	\\
1	&	(1-\sqrt{l})
\end{bmatrix}
\mathbf{Z}(0)
\\
\mathbf{Z}(z) &=
\frac{1}{2\sqrt{l}}
\begin{bmatrix}
(1-\sqrt{l})	&	(1+\sqrt{l})	\\
-1				&	-1
\end{bmatrix}
\begin{bmatrix}
-e^{-z(1-\sqrt{l})}	&	-(1+\sqrt{l})e^{-z(1-\sqrt{l})}	\\
e^{-z(1+\sqrt{l})}	&	(1-\sqrt{l})e^{-z(1+\sqrt{l})}
\end{bmatrix}
\mathbf{Z}(0)
\\
\mathbf{Z}(z)
&=
\frac{1}{2\sqrt{l}}
\begin{bmatrix}
(1+\sqrt{l})e^{-z(1+\sqrt{l})} -(1-\sqrt{l})e^{-z(1-\sqrt{l})}	&	(1-l)\left( e^{-z(1+\sqrt{l})} -e^{-z(1-\sqrt{l})} \right)	\\
e^{-z(1-\sqrt{l})} -e^{-z(1+\sqrt{l})}							&	(1+\sqrt{l})e^{-z(1-\sqrt{l})} -(1-\sqrt{l})e^{-z(1+\sqrt{l})}
\end{bmatrix}
\mathbf{Z}(0)
\end{align*}
So that as defined
\begin{multline*}
\begin{bmatrix}
dZ/dz\\
Z
\end{bmatrix}
(z)
= \\
\frac{1}{2\sqrt{l}}
\begin{bmatrix}
(1+\sqrt{l})e^{-z(1+\sqrt{l})} -(1-\sqrt{l})e^{-z(1-\sqrt{l})}	&	(1-l)\left( e^{-z(1+\sqrt{l})} -e^{-z(1-\sqrt{l})} \right)	\\
e^{-z(1-\sqrt{l})} -e^{-z(1+\sqrt{l})}							&	(1+\sqrt{l})e^{-z(1-\sqrt{l})} -(1-\sqrt{l})e^{-z(1+\sqrt{l})}
\end{bmatrix}
\begin{bmatrix}
dZ/dz\\
Z
\end{bmatrix}
(0) 
\end{multline*}
and thus if we call $C=Z(0),D=(dZ/dz)(0)$ and 
\begin{align*}
Z(z) &= \frac{1}{2\sqrt{l}}\left[D\left(e^{z(\sqrt{l}-1)} -e^{-z(\sqrt{l}+1)}\right) +C\left((\sqrt{l}+1)e^{z(\sqrt{l}-1)} +(\sqrt{l}-1)e^{-z(\sqrt{l}+1)}\right)\right]\\
Z(z) &= \frac{1}{2\sqrt{l}}\left[\left((\sqrt{l}+1)C+D \right)e^{z(\sqrt{l}-1)} +\left((\sqrt{l}-1)C-D \right)e^{-z(\sqrt{l}+1)} \right]\\
Z(z) &= \frac{1}{2\sqrt{l}}\left[ \left[ C\sqrt{l}e^{z(\sqrt{l}-1)} +C\sqrt{l}e^{-z(\sqrt{l}+1)}\right] +(C+D)\left[e^{z(\sqrt{l}-1)} -e^{-z(\sqrt{l}+1)}\right] \right]\\
Z(z) &= \frac{C}{2}\left[ e^{z(\sqrt{l}-1)} +e^{-z(\sqrt{l}+1)}\right] +\left(\frac{C+D}{2\sqrt{l}}\right)\left[e^{z(\sqrt{l}-1)} -e^{-z(\sqrt{l}+1)}\right] \\
\end{align*}
We didn't need to separate this far though. If we absorb and rename the constants $C,D$ we can get rid of the leading term and substitute back in for $R$ to find
\[ Z(z) = \frac{1}{2\sqrt{l}}\left[\left((\sqrt{l}+1)C+D \right)e^{z(\sqrt{l}-1)} +\left((\sqrt{l}-1)C-D \right)e^{-z(\sqrt{l}+1)} \right]\\ \]
\[ R(r) = Cr^{(\sqrt{l}-1)} +Dr^{-(\sqrt{l}+1)} \]

\paragraph{}
We now substitute back in to our original separation and solve for the last variable
\begin{align*}
\frac{1}{R(r)} \frac{\partial^{2}}{\partial r^{2}}(r^{2}R(r)) +\frac{1}{P(\theta)\sin\theta} \frac{\partial}{\partial\theta}\left( \sin\theta \frac{\partial}{\partial\theta}P(\theta) \right) -\frac{m}{\sin^{2}\theta} &= 0\\
\frac{1}{P(\theta)\sin\theta} \frac{\partial}{\partial\theta}\left( \sin\theta \frac{\partial}{\partial\theta}P(\theta) \right) +l -\frac{m}{\sin^{2}\theta} &= 0\\
\end{align*}

\end{document}
%###############################################################################