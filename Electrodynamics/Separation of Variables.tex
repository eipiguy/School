\documentclass{article}

\usepackage{amsmath,amsfonts,amssymb,amsthm}
\usepackage{enumerate}
\usepackage{graphicx}


% Opening
\title{Solving Poisson's Equation\\
Using Separation of Variables\\
In Multiple Coordinates}
\author{Neal D. Nesbitt}

\begin{document}
\maketitle

\section{Cartesian Coordinates}
Let $\Phi:\mathbb{R}^{n}\to\mathbb{R}$ be a scalar field, and recall that the Laplacian of $\Phi$ is
\[ \nabla^{2}\Phi = \nabla \cdot \nabla \Phi = \nabla \cdot \left[ \sum_{k=1}^{n}\frac{\partial}{\partial x_{k}} \left[ \Phi(x_{1},\dots,x_{n}) \right] \mathbf{\hat{x}_{k}} \right] = \sum_{k=1}^{n}\frac{\partial^{2}}{\partial x_{k}^{2}} \left[ \Phi(x_{1},\dots,x_{n}) \right] \]

The crux of the separation of variables technique relies on the assumption that $\Phi$ can be written as the product of $n$ independent scalar functions: $\exists X_{1}(x_{1}),\dots,X_{n}(x_{n}):\mathbb{R}\to\mathbb{R} \text{ st. } \Phi(x_{1},\dots,x_{n}) =  X_{1}(x_{1}) \dots X_{n}(x_{n})$.

Then if we substitute this into the Laplacian we can find
\begin{align*}
\nabla^{2}\Phi &= \sum_{k=1}^{n}\frac{\partial^{2}}{\partial x_{k}^{2}} \left[ \Phi(x_{1},\dots,x_{n}) \right] = \sum_{k=1}^{n}\frac{\partial^{2}}{\partial x_{k}^{2}} X_{1}(x_{1}) \dots X_{n}(x_{n})\\
&= \sum_{k=1}^{n} X_{1}(x_{1}) \dots X_{k-1}(x_{k-1}) X_{k+1}(x_{k+1}) \dots X_{n}(x_{n}) \frac{d^{2}}{dx_{k}^{2}} X_{k}(x_{k})\\
\frac{\nabla^{2}\Phi }{\Phi} &= \sum_{k=1}^{n} \frac{1}{X_{k}(x_{k})} \frac{d^{2}}{dx_{k}^{2}} X_{k}(x_{k})
\end{align*}

Then each of the components of the sum are functions of an independent variable, and since the terms must always sum to the same potential, each must be constant. We can name these constants $\alpha_{1},\cdots,\alpha_{n}$ so that we can separate the sum and solve each differential equation separately.
\begin{align*}
\frac{\nabla^{2}\Phi }{\Phi} = \sum_{k=1}^{n} \frac{1}{X_{k}(x_{k})} \frac{d^{2}}{dx_{k}^{2}} X_{k}(x_{k}) &= \sum_{k=1}^{n} \alpha_{k}\\
\frac{1}{X_{k}(x_{k})} \frac{d^{2}}{dx_{k}^{2}} X_{k}(x_{k}) &= \alpha_{k}\\
\frac{d^{2}}{dx_{k}^{2}} X_{k}(x_{k}) &= \alpha_{k}X_{k}(x_{k})\\
X_{k}(x_{k}) &= \sum_{j=1}^{\infty} \tilde{A}_{(k,j)}e^{\pm x_{k}\sqrt{\alpha_{k}}}\\
X_{k}(x_{k}) &= A_{k}e^{+x_{k}\sqrt{\alpha_{k}}} + B_{k}e^{-x_{k}\sqrt{\alpha_{k}}}\\
\Phi(x_{1},\cdots,x_{n}) &= \prod_{k=1}^{n} \left( A_{k}e^{+x_{k}\sqrt{\alpha_{k}}} + B_{k}e^{-x_{k}\sqrt{\alpha_{k}}} \right)
\end{align*}

\section{Polar}
If we have azimuthal symmetry, and we can just consider the problem in polar coordinates about any longitudinal plane $\mathbf{x} = (r,\phi)$. Since there is no charge in our annulus of interest $a < r < b$, we can use Laplace's equation in polar coordinates to find:
\begin{align*}
\nabla^{2} \Phi &= 0\\
\frac{1}{r} \frac{\partial}{\partial r} \left( r \frac{\partial \Phi}{\partial r} \right) + \frac{1}{r^{2}} \frac{\partial^{2} \Phi}{\partial \phi^{2}} &= 0
\end{align*}
If we then assume $\Phi=R(r)F(\phi)$ for some scalar functions $R,F$, we can use the product rule to see
\begin{align*}
\frac{1}{r} \frac{\partial}{\partial r} \left( r \frac{\partial}{\partial r} \left[ R(r)F(\phi) \right] \right) + \frac{1}{r^{2}} \frac{\partial^{2}}{\partial \phi^{2}} \left[ R(r)F(\phi) \right] &= 0\\
\frac{F(\phi)}{r} \frac{\partial}{\partial r} \left( r \frac{\partial R}{\partial r}(r) \right) + \frac{R(r)}{r^{2}} \frac{\partial^{2} F}{\partial \phi^{2}}(\phi) &= 0\\
\frac{F(\phi)}{r} \left( \frac{\partial R}{\partial r}(r) +  r \frac{\partial^{2} R}{\partial r^{2}}(r) \right) + \frac{R(r)}{r^{2}} \frac{\partial^{2} F}{\partial \phi^{2}}(\phi) &= 0\\
\end{align*}
Then multiplying through by $r^{2}/\Phi = r^{2}RF$ will separate the remaining variables:
\[ \frac{r}{R(r)} \left( \frac{\partial R}{\partial r}(r) +  r \frac{\partial^{2} R}{\partial r^{2}}(r) \right) + \frac{1}{F(\phi)} \frac{\partial^{2} F}{\partial \phi^{2}}(\phi) = 0\\ \]
We then have two differential equations that each depend on a different variable, and add to zero. We can call them $\pm\alpha$ and find
\[ \frac{r}{R(r)} \left( \frac{\partial R}{\partial r}(r) + r \frac{\partial^{2} R}{\partial r^{2}}(r) \right) + \frac{1}{F(\phi)} \frac{\partial^{2} F}{\partial \phi^{2}}(\phi) = \alpha + (-\alpha) \]
\begin{align*}
\frac{r}{R(r)} \left( \frac{\partial R}{\partial r}(r) + r \frac{\partial^{2} R}{\partial r^{2}}(r) \right) &= \alpha
&
\frac{-1}{F(\phi)} \frac{\partial^{2} F}{\partial \phi^{2}}(\phi) &= \alpha\\
\frac{\partial R}{\partial r}(r) + r \frac{\partial^{2} R}{\partial r^{2}}(r) &= \frac{\alpha}{r} R(r)
&
\frac{\partial^{2} F}{\partial \phi^{2}}(\phi) &= -\alpha F(\phi)\\
\frac{\partial^{2} R}{\partial r^{2}}(r) &= \frac{-1}{r} \frac{\partial R}{\partial r}(r) + \frac{\alpha}{r^{2}} R(r)
&
\frac{\partial^{2} F}{\partial \phi^{2}}(\phi) &= -\alpha F(\phi)\\
\end{align*}
Here for constants $A$ and $B$, $F$ has solutions
\[ F(\phi)= Ae^{i\phi\sqrt{\alpha}} + Be^{-i\phi\sqrt{\alpha}} \]

\paragraph{}
Then we look for solutions for $R(r)$ multiplying through by $r^{2}$
\begin{align*}
\frac{\partial^{2} R}{\partial r^{2}}(r) &= \frac{-1}{r} \frac{\partial R}{\partial r}(r) + \frac{\alpha}{r^{2}} R(r)\\
 r^{2}\frac{\partial^{2} R}{\partial r^{2}}(r) +r \frac{\partial R}{\partial r}(r) &= \alpha R(r)\\
\end{align*}

While we have to separate $R$ into parts where
\[ S(r) = \frac{\partial R}{\partial r} (r) \implies \frac{\partial S}{\partial r} (r) =  \frac{\partial^{2} R}{\partial r^{2}} (r) \]
\begin{align*}
\frac{\partial S}{\partial r}(r) &= \frac{-1}{r} S(r) + \frac{\alpha}{r^{2}} R(r)
&
\frac{\partial R}{\partial r} (r) &= S(r)\\
\end{align*}
\[
\boxed{\begin{bmatrix}
(\partial S/\partial r) \\
(\partial R/\partial r) \\
\end{bmatrix}
(r) = 
\frac{1}{r^{2}}
\begin{bmatrix}
-r	&	\alpha	\\
r^{2}	&	0	\\
\end{bmatrix}
\begin{bmatrix}
S(r)	\\
R(r)	\\
\end{bmatrix}}
\]
We can find the eigenvalues $\lambda$ with the characteristic polynomial
\[ |A-\lambda I| = \begin{vmatrix}
-r - \lambda	&	\alpha		\\
r^{2}			&	-\lambda	\\
\end{vmatrix} = 0 \]
\begin{align*}
\lambda(r + \lambda) - \alpha r^{2} &= 0\\
\lambda^{2} + \lambda r - \alpha r^{2} &= 0\\
\end{align*}
\begin{align*}
\lambda &= \frac{1}{2}\left( -r \pm \sqrt{r^{2} + 4\alpha r^{2}} \right)\\
\lambda &= \boxed{\frac{-r}{2}\left( 1 \mp \sqrt{1 + 4\alpha} \right)}
\end{align*}
Then substituting this back into our original equation we find
\begin{align*}
\begin{bmatrix}
-r + \frac{r}{2}\left( 1 \mp \sqrt{1 + 4\alpha} \right)	&	\alpha		\\
r^{2}														&	\frac{r}{2}\left( 1 \mp \sqrt{1 + 4\alpha} \right)	\\
\end{bmatrix}
\begin{bmatrix}
v_{1}\\
v_{2}
\end{bmatrix}
&=
\begin{bmatrix}
0\\
0
\end{bmatrix}
\\
\begin{bmatrix}
\frac{-r}{2}\left( 1 \pm \sqrt{1 + 4\alpha} \right)	&	\alpha		\\
r^{2}												&	\frac{r}{2}\left( 1 \mp \sqrt{1 + 4\alpha} \right)	\\
\end{bmatrix}
\begin{bmatrix}
v_{1}\\
v_{2}
\end{bmatrix}
&=
\begin{bmatrix}
0\\
0
\end{bmatrix}
\\
\begin{bmatrix}
\frac{-r}{2}\left( 1 \pm \sqrt{1 + 4\alpha} \right)	&	\alpha		\\
0												&	\frac{r}{2}\left( 1 \mp \sqrt{1 + 4\alpha} \right) - \frac{2\alpha r}{\left( 1 \pm \sqrt{1 + 4\alpha} \right)}	\\
\end{bmatrix}
\begin{bmatrix}
v_{1}\\
v_{2}
\end{bmatrix}
&=
\begin{bmatrix}
0\\
0
\end{bmatrix}
\\
\begin{bmatrix}
\frac{-r}{2}\left( 1 \pm \sqrt{1 + 4\alpha} \right)	&	\alpha		\\
0	&	\frac{r}{2}\left( 1 \mp \sqrt{1 + 4\alpha} \right) + \frac{2\alpha r \left( 1 \mp \sqrt{1 + 4\alpha} \right) }{\left( 1 - (1 + 4\alpha) \right)}	\\
\end{bmatrix}
\begin{bmatrix}
v_{1}\\
v_{2}
\end{bmatrix}
&=
\begin{bmatrix}
0\\
0
\end{bmatrix}
\\
\begin{bmatrix}
\frac{-r}{2}\left( 1 \pm \sqrt{1 + 4\alpha} \right)	&	\alpha		\\
0	&	\frac{r}{2}\left( 1 \mp \sqrt{1 + 4\alpha} \right) - \frac{r}{2} \left( 1 \mp \sqrt{1 + 4\alpha} \right)\\
\end{bmatrix}
\begin{bmatrix}
v_{1}\\
v_{2}
\end{bmatrix}
&=
\begin{bmatrix}
0\\
0
\end{bmatrix}
\\
\begin{bmatrix}
\frac{-r}{2}\left( 1 \pm \sqrt{1 + 4\alpha} \right)	&	\alpha		\\
0	&		0\\
\end{bmatrix}
\begin{bmatrix}
v_{1}\\
v_{2}
\end{bmatrix}
&=
\begin{bmatrix}
0\\
0
\end{bmatrix}
\end{align*}
Showing the system is singular as expected, and we can trust our eigenvalue computations a little more.

\paragraph{}
We then use one of the equations to find our eigenvector lines. Using the second one with a little algebra gives
\[ -2rv_{1} = \left( 1 \mp \sqrt{1 + 4\alpha} \right) v_{2} \]
where if we pick an arbitrary values of $v_{2}$ to simplify the math, $v_{2}=-2r$ will make $v_{1} = \left( 1 \mp \sqrt{1 + 4\alpha} \right)$
the matrix of eigenvectors:
\[ P = 
\boxed{\begin{bmatrix}
\left( 1 - \sqrt{1 + 4\alpha} \right)	&	\left( 1 + \sqrt{1 + 4\alpha} \right)	\\
-2r										&	-2r	
\end{bmatrix}}
 \]
Then 
\begin{align*}
P^{-1} &= 
\begin{bmatrix}
a&b\\
c&d
\end{bmatrix}^{-1}
=
\frac{1}{|P|}
\begin{bmatrix}
d	&	-b\\
-c	&	a
\end{bmatrix}\\
P^{-1} &= \frac{1}{|P|}
\begin{bmatrix}
-2r	&	-\left( 1 + \sqrt{1 + 4\alpha} \right)	\\
2r	&	\left( 1 - \sqrt{1 + 4\alpha} \right)	
\end{bmatrix}
\\
P^{-1} &= \frac{1}{\left| -2r\left( 1 - \sqrt{1 + 4\alpha} \right) +2r\left( 1 + \sqrt{1 + 4\alpha} \right) \right|}
\begin{bmatrix}
-2r	&	-\left( 1 + \sqrt{1 + 4\alpha} \right)	\\
2r	&	\left( 1 - \sqrt{1 + 4\alpha} \right)	
\end{bmatrix}
\\
P^{-1} &= \frac{1}{\left| 4r\sqrt{1 + 4\alpha} \right|}
\begin{bmatrix}
-2r	&	-\left( 1 + \sqrt{1 + 4\alpha} \right)	\\
2r	&	\left( 1 - \sqrt{1 + 4\alpha} \right)	
\end{bmatrix}
\end{align*}
and since $r$ is always positive, and we always take the positive root unless specified otherwise, we can just write
\[ P^{-1} = \boxed{\frac{1}{ 4r\sqrt{1 + 4\alpha} }
\begin{bmatrix}
-2r	&	-\left( 1 + \sqrt{1 + 4\alpha} \right)	\\
2r	&	\left( 1 - \sqrt{1 + 4\alpha} \right)	
\end{bmatrix}} \]
So double check
\begin{align*}
P^{-1}AP &=
\frac{1}{ 4r\sqrt{1 + 4\alpha} }
\begin{bmatrix}
-2r	&	-\left( 1 + \sqrt{1 + 4\alpha} \right)	\\
2r	&	\left( 1 - \sqrt{1 + 4\alpha} \right)	
\end{bmatrix}
\begin{bmatrix}
-r		&	\alpha	\\
r^{2}	&	0	\\
\end{bmatrix}
\begin{bmatrix}
\left( 1 - \sqrt{1 + 4\alpha} \right)	&	\left( 1 + \sqrt{1 + 4\alpha} \right)	\\
-2r										&	-2r	
\end{bmatrix}
\\
&=
\frac{1}{ 4r\sqrt{1 + 4\alpha} }
\begin{bmatrix}
2r^{2}-r^{2}\left( 1 + \sqrt{1 + 4\alpha} \right) & -2\alpha r	\\
-2r^{2}+r^{2}\left( 1 - \sqrt{1 + 4\alpha} \right)	& 2\alpha r
\end{bmatrix}
\begin{bmatrix}
\left( 1 - \sqrt{1 + 4\alpha} \right)	&	\left( 1 + \sqrt{1 + 4\alpha} \right)	\\
-2r										&	-2r	
\end{bmatrix}
\\
&=
\frac{1}{ 4r\sqrt{1 + 4\alpha} }
\begin{bmatrix}
r^{2}\left( 1 - \sqrt{1 + 4\alpha} \right) & -2\alpha r	\\
-r^{2}\left( 1 + \sqrt{1 + 4\alpha} \right)	& 2\alpha r
\end{bmatrix}
\begin{bmatrix}
\left( 1 - \sqrt{1 + 4\alpha} \right)	&	\left( 1 + \sqrt{1 + 4\alpha} \right)	\\
-2r										&	-2r	
\end{bmatrix}
\\
&=
\frac{1}{ 4r\sqrt{1 + 4\alpha} }
\begin{bmatrix}
r^{2}\left( 1 - \sqrt{1 + 4\alpha} \right)^{2} +4\alpha r^{2}	&	-4\alpha r +4\alpha r	\\
4\alpha r^{2}-4\alpha r^{2}	& -r^{2}\left( 1 + \sqrt{1 + 4\alpha} \right)^{2} -4\alpha r^{2}
\end{bmatrix}
\\
&=
\frac{r}{ 4\sqrt{1 + 4\alpha} }
\begin{bmatrix}
1 - 2\sqrt{1 + 4\alpha} +(1+4\alpha) +4\alpha	&	0	\\
0	& -\left( 1 + 2\sqrt{1 + 4\alpha} +(1+4\alpha)\right) -4\alpha
\end{bmatrix}
\\
&=
\frac{r}{ 4\sqrt{1 + 4\alpha} }
\begin{bmatrix}
2 +8\alpha -2\sqrt{1 + 4\alpha}	&	0	\\
0	& -\left( 2 +8\alpha +2\sqrt{1 + 4\alpha}\right)
\end{bmatrix}
\\
&=
\frac{r}{ 2\sqrt{1 + 4\alpha} }
\begin{bmatrix}
(1 +4\alpha) -\sqrt{1 + 4\alpha}	&	0	\\
0	& -\left( (1 +4\alpha) +\sqrt{1 + 4\alpha}\right)
\end{bmatrix}
\\
&=
\frac{r}{ 2 }
\begin{bmatrix}
\sqrt{1 + 4\alpha} -1	&	0	\\
0	& -\left( \sqrt{1 + 4\alpha} +1 \right)
\end{bmatrix}
\\
P^{-1}AP &=
\begin{bmatrix}
\frac{r}{2}(-1 +\sqrt{1 + 4\alpha})	&	0	\\
0	& \frac{r}{2}(-1 -\sqrt{1 + 4\alpha})
\end{bmatrix}
= \text{diag}\left(\lambda_{1},\lambda_{2}\right)
\end{align*}
So our $P$ matrices can be used as expected to uncouple the system, and we can trust more in our eigenvector computations.

\paragraph{}
Then if we start with the system $\dot{\mathbf{x}} = cA\mathbf{x}$ and use the change of variables $\mathbf{y}=P\mathbf{x}$ such that $\mathbf{x}=P^{-1}\mathbf{y}$ 
\[ \dot{\mathbf{y}} = P\dot{\mathbf{x}} = PcA\mathbf{x} = cPAP^{-1}\mathbf{y} = c\text{ diag}\{ \lambda_{1},\lambda_{2} \} \mathbf{y} \]
which gives a diagonal matrix with the solution
\[ \mathbf{y}(r) = e^{cPAP^{-1}(r)}\mathbf{y}(0) = \text{diag}\{ e^{c\lambda_{1}(r)}, e^{c\lambda_{2}(r)} \}\mathbf{y}(0)\\ \]
and then since $\mathbf{y}=P\mathbf{x}$ and 
\[ 
P = 
\begin{bmatrix}
\left( 1 - \sqrt{1 + 4\alpha} \right)	&	\left( 1 + \sqrt{1 + 4\alpha} \right)	\\
-2r										&	-2r	
\end{bmatrix},
P^{-1} = \frac{1}{ 4r\sqrt{1 + 4\alpha} }
\begin{bmatrix}
-2r	&	-\left( 1 + \sqrt{1 + 4\alpha} \right)	\\
2r	&	\left( 1 - \sqrt{1 + 4\alpha} \right)	
\end{bmatrix}
\]
\begin{align*}
P\mathbf{x}(r) &= \text{diag}\{ e^{c\lambda_{1}(r)}, e^{c\lambda_{2}(r)} \}P\mathbf{x}(0)\\
\mathbf{x}(r) &= e^{c}P^{-1}\text{diag}\{ e^{\lambda_{1}(r)}, e^{c\lambda_{2}(r)} \}P\mathbf{x}(0)
\end{align*}
\begin{multline*}
\begin{bmatrix}
S(r)\\
R(r)
\end{bmatrix}
= \frac{e^{1/r^{2}}}{ 4r\sqrt{1 + 4\alpha} }\\
\begin{bmatrix}
-2r	&	-\left( 1 + \sqrt{1 + 4\alpha} \right)	\\
2r	&	\left( 1 - \sqrt{1 + 4\alpha} \right)	
\end{bmatrix}
\begin{bmatrix}
e^{r^{2}(-1 +\sqrt{1 + 4\alpha})/2}	&	0	\\
0	& e^{-r^{2}(1+\sqrt{1 + 4\alpha})/2}
\end{bmatrix}
\begin{bmatrix}
\left( 1 - \sqrt{1 + 4\alpha} \right)	&	\left( 1 + \sqrt{1 + 4\alpha} \right)	\\
-2r										&	-2r	
\end{bmatrix}
\begin{bmatrix}
S(0)\\
R(0)
\end{bmatrix}
\end{multline*}


\begin{multline*}
\begin{bmatrix}
S(r)\\
R(r)
\end{bmatrix}
= \frac{e^{1/r^{2}}}{ 4r\sqrt{1 + 4\alpha} }\\
\begin{bmatrix}
-2r	&	-\left( 1 + \sqrt{1 + 4\alpha} \right)	\\
2r	&	\left( 1 - \sqrt{1 + 4\alpha} \right)	
\end{bmatrix}
\begin{bmatrix}
e^{-r^{2}/2}e^{r^{2}\sqrt{1 + 4\alpha}/2}	&	0	\\
0	& e^{-r^{2}/2}e^{-r^{2}\sqrt{1 + 4\alpha}/2}
\end{bmatrix}
\begin{bmatrix}
\left( 1 - \sqrt{1 + 4\alpha} \right)	&	\left( 1 + \sqrt{1 + 4\alpha} \right)	\\
-2r										&	-2r	
\end{bmatrix}
\begin{bmatrix}
S(0)\\
R(0)
\end{bmatrix}
\end{multline*}


\begin{multline*}
\begin{bmatrix}
S(r)\\
R(r)
\end{bmatrix}
= \frac{e^{1/r^{2}}e^{-r^{2}/2}}{ 4r\sqrt{1 + 4\alpha} }\\
\begin{bmatrix}
-2r	&	-\left( 1 + \sqrt{1 + 4\alpha} \right)	\\
2r	&	\left( 1 - \sqrt{1 + 4\alpha} \right)	
\end{bmatrix}
\begin{bmatrix}
e^{r^{2}\sqrt{1 + 4\alpha}/2}	&	0	\\
0	& e^{-r^{2}\sqrt{1 + 4\alpha}/2}
\end{bmatrix}
\begin{bmatrix}
\left( 1 - \sqrt{1 + 4\alpha} \right)	&	\left( 1 + \sqrt{1 + 4\alpha} \right)	\\
-2r										&	-2r	
\end{bmatrix}
\begin{bmatrix}
S(0)\\
R(0)
\end{bmatrix}
\end{multline*}

\begin{multline*}
\begin{bmatrix}
S(r)\\
R(r)
\end{bmatrix}
= \frac{e^{r^{-2}-r^{2}/2}}{ 4r\sqrt{1 + 4\alpha} }\\
\begin{bmatrix}
-2re^{r^{2}\sqrt{1 + 4\alpha}/2}	&	-\left( 1 + \sqrt{1 + 4\alpha} \right)e^{-r^{2}\sqrt{1 + 4\alpha}/2}	\\
2re^{r^{2}\sqrt{1 + 4\alpha}/2}	&	\left( 1 - \sqrt{1 + 4\alpha} \right)e^{-r^{2}\sqrt{1 + 4\alpha}/2}	
\end{bmatrix}
\begin{bmatrix}
\left( 1 - \sqrt{1 + 4\alpha} \right)	&	\left( 1 + \sqrt{1 + 4\alpha} \right)	\\
-2r										&	-2r	
\end{bmatrix}
\begin{bmatrix}
S(0)\\
R(0)
\end{bmatrix}
\end{multline*}

\begin{multline*}
\begin{bmatrix}
S(r)\\
R(r)
\end{bmatrix}
= \frac{e^{r^{-2}-r^{2}/2}}{ 4r\sqrt{1 + 4\alpha} }\\
\left[
\begin{matrix}
-2r\left( 1 - \sqrt{1 + 4\alpha} \right)e^{r^{2}\sqrt{1 + 4\alpha}/2} +2r\left( 1 + \sqrt{1 + 4\alpha} \right)e^{-r^{2}\sqrt{1 + 4\alpha}/2}	\\
2r\left( 1 - \sqrt{1 + 4\alpha} \right)e^{r^{2}\sqrt{1 + 4\alpha}/2} -2r\left( 1 - \sqrt{1 + 4\alpha} \right)e^{-r^{2}\sqrt{1 + 4\alpha}/2}	
\end{matrix}
\right.
\\
\left.
\begin{matrix}
-2r\left( 1 + \sqrt{1 + 4\alpha} \right)e^{r^{2}\sqrt{1 + 4\alpha}/2} +2r\left( 1 + \sqrt{1 + 4\alpha} \right)e^{-r^{2}\sqrt{1 + 4\alpha}/2}	\\
2r\left( 1 + \sqrt{1 + 4\alpha} \right)e^{r^{2}\sqrt{1 + 4\alpha}/2} -2r\left( 1 - \sqrt{1 + 4\alpha} \right)e^{-r^{2}\sqrt{1 + 4\alpha}/2}	
\end{matrix}
\right]
\begin{bmatrix}
S(0)\\
R(0)
\end{bmatrix}
\end{multline*}

\begin{multline*}
\begin{bmatrix}
S(r)\\
R(r)
\end{bmatrix}
= \frac{e^{r^{-2}-r^{2}/2}}{ 2\sqrt{1 + 4\alpha} }\\
\left[
\begin{matrix}
-\left( 1 - \sqrt{1 + 4\alpha} \right)e^{r^{2}\sqrt{1 + 4\alpha}/2} +\left( 1 + \sqrt{1 + 4\alpha} \right)e^{-r^{2}\sqrt{1 + 4\alpha}/2}	\\
\left( 1 - \sqrt{1 + 4\alpha} \right)e^{r^{2}\sqrt{1 + 4\alpha}/2} -\left( 1 - \sqrt{1 + 4\alpha} \right)e^{-r^{2}\sqrt{1 + 4\alpha}/2}	
\end{matrix}
\right.
\\
\left.
\begin{matrix}
-\left( 1 + \sqrt{1 + 4\alpha} \right)e^{r^{2}\sqrt{1 + 4\alpha}/2} +\left( 1 + \sqrt{1 + 4\alpha} \right)e^{-r^{2}\sqrt{1 + 4\alpha}/2}	\\
\left( 1 + \sqrt{1 + 4\alpha} \right)e^{r^{2}\sqrt{1 + 4\alpha}/2} -\left( 1 - \sqrt{1 + 4\alpha} \right)e^{-r^{2}\sqrt{1 + 4\alpha}/2}	
\end{matrix}
\right]
\begin{bmatrix}
S(0)\\
R(0)
\end{bmatrix}
\end{multline*}
Now we attempt to solve this system of equations using Gaussian elimination. Take the first pivot coefficient to be
\[
\frac{\left( 1 - \sqrt{1 + 4\alpha} \right)e^{r^{2}\sqrt{1 + 4\alpha}/2} -\left( 1 - \sqrt{1 + 4\alpha} \right)e^{-r^{2}\sqrt{1 + 4\alpha}/2}}{-\left( 1 - \sqrt{1 + 4\alpha} \right)e^{r^{2}\sqrt{1 + 4\alpha}/2} +\left( 1 + \sqrt{1 + 4\alpha} \right)e^{-r^{2}\sqrt{1 + 4\alpha}/2}}
\]
such that the second column of the first row is scaled to
\[
\frac{ 4\alpha e^{r^{2}\sqrt{1+4\alpha}} -8\alpha +4\alpha e^{-r^{2}\sqrt{1+4\alpha}} }{ -\left( 1 - \sqrt{1 + 4\alpha} \right)e^{r^{2}\sqrt{1 + 4\alpha}/2} +\left( 1 + \sqrt{1 + 4\alpha} \right)e^{-r^{2}\sqrt{1 + 4\alpha}/2} }
\]



\end{document}