\documentclass{article}

\usepackage{amsmath,amsfonts,amssymb,amsthm}
\usepackage{listings,color}
\usepackage{graphicx}


% Opening
\title{Convince Yourself}
\author{Neal D. Nesbitt}

\begin{document}
\maketitle

\theoremstyle{definition}
\newtheorem{problem}{Problem}

\begin{problem} 
\textbf{Why Do Green Functions Work?}\\
I am told to start with a point charge $q$ in the specified and find the corresponding image charge that will fulfill the boundary conditions. This is then worked into cylindrical and so on to solve the problem. Why am I doing this?\\
\\How do we know that finding the Green function for these imaginary charges will solve the original potential problem?\\
\\
Even if the Green function cancels out on our boundary conditions, how can we be assured that the resulting potential integrals will as well?\\
\\
Plus there is still a redundancy in the original integral requiring us to know the potential to compute the potential. Does plugging in the Green function cancel this part of the integral?
\end{problem}
Given that
\[ \Phi = \frac{1}{4\pi\epsilon_{0}} \int_{V} \rho(\mathbf{x'}) G(\mathbf{x},\mathbf{x'}) d^{3}x' + \frac{1}{4\pi} \oint_{S} \left[ G(\mathbf{x},\mathbf{x'}) \frac{\partial\Phi}{\partial n'} -\Phi(\mathbf{x}) \frac{\partial G(\mathbf{x},\mathbf{x'})}{\partial n'}) \right] da' \]

\begin{align*}
G(\mathbf{x},\mathbf{x'}) &= \frac{1}{\left| \mathbf{x}-\mathbf{x'} \right|} + F(\mathbf{x},\mathbf{x'})\\
\nabla'^{2} G(\mathbf{x},\mathbf{x'}) &= -4\pi\delta(\mathbf{x}-\mathbf{x'})
\end{align*}
and that inside the specified volume $V$ 
\[ \nabla'^{2} F(\mathbf{x},\mathbf{x'}) = 0 \]

If we then work backwards through the derivation of this formula using $\Phi=\phi$ and $1/\left|\mathbf{x}-\mathbf{x'}\right|=\psi$, we should arrive at Green's theorem
\[ \int_{V} \left( \phi\nabla^{2}\psi - \psi\nabla^{2}\phi \right) d^{3}x = \oint_{s} \left[ \phi\frac{\partial\psi}{\partial n} - \psi\frac{\partial\phi}{\partial n} \right] da \]
which is derived from the divergence theorem
\[ \int_{V} \nabla\cdot\mathbf{A} d^{3}x = \oint_{S} \mathbf{A}\cdot\mathbf{\hat{n}} da \]
where our vector field in this case is 
\[ \mathbf{A}=\phi\nabla\psi = \Phi\nabla\left(\frac{1}{\left|\mathbf{x}-\mathbf{x'}\right|}\right) = \Phi \frac{-(\mathbf{x}-\mathbf{x'})}{\left|\mathbf{x}-\mathbf{x'}\right|^{3}} = \frac{-\Phi\mathbf{\hat{r}}}{\left|\mathbf{x}-\mathbf{x'}\right|^{2}} \]

\begin{problem}
What is the potential for a point charge $q$ located at $\mathbf{x'}$?
\end{problem}



\end{document}