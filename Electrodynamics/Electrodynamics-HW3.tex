\documentclass{article}

\usepackage{amsmath,amsfonts,amssymb,amsthm}
\usepackage{listings,color}
\usepackage{graphicx}


% Opening
\title{Electrodynamics HW3\\
Ch2 - 7,11,23 (pg85)}
\author{Neal D. Nesbitt}

\begin{document}
\maketitle

\theoremstyle{definition}
\newtheorem{problem}{Problem}

\setcounter{problem}{6}
\begin{problem}
Consider a potential problem in the half-space defined by $z\ge 0$, with Derichlet boundary conditions on the plane $z=0$ (and at infinity)
\begin{itemize}
\item Write down the appropriate Green function $G(\mathbf{x},\mathbf{x}')$
\item If the potential on the plane $z=0$ is specified to be $\Phi=V$ inside a circle of radius $a$ centered at the origin, and $\Phi = 0$ outside that circle, find an integral expression for the potential at the point $P$ specified in terms of cylindrical coordinates $(\rho,\phi,z)$
\item Show that, along the axis of the circle $(\rho = 0)$, the potential is given by
\[ \Phi = V \left( 1 - \frac{z}{\sqrt{a^{2}+z^{2}}} \right) \]
\item Show that at large distances $(\rho^{2}+z^{2} \gg a^{2} )$ the potential can be expanded in a power series in $(\rho^{2}+z^{2})^{-1}$, and that the leading terms are
\[ \Phi = \frac{Va^{2}}{2} \frac{z}{(\rho^{2}+z^{2})^{3/2}} \left[ 1 - \frac{3a^{2}}{4(\rho^{2}+z^{2})} + \frac{5(3\rho^{2}a^{2}+a^{4})}{8(\rho^{2}+z^{2})^{2}} + \cdots \right] \]
Verify the consistence of the solution with your previous results.
\end{itemize}
\end{problem}

We know from the book's derivations using Green's theorem that
\[ \Phi = \frac{1}{4\pi\epsilon_{0}} \int_{V} \rho(\mathbf{x'}) G(\mathbf{x},\mathbf{x'}) d^{3}x' + \frac{1}{4\pi} \oint_{S} \left[ G(\mathbf{x},\mathbf{x'}) \frac{\partial\Phi}{\partial n'}(\mathbf{x'}) -\Phi(\mathbf{x'}) \frac{\partial G(\mathbf{x},\mathbf{x'})}{\partial n'}) \right] da' \]


Where for Dirichlet boundary conditions, $G_{D}(\mathbf{x},\mathbf{x'})=0$ for $\mathbf{x'}$ on $S$ ( $\mathbf{x'}=(x,y,0)$), and this reduces to
\[ \Phi = \frac{1}{4\pi\epsilon_{0}} \int_{V} \rho(\mathbf{x'}) G_{D}(\mathbf{x},\mathbf{x'}) d^{3}x' + \frac{1}{4\pi} \oint_{S} \Phi(\mathbf{x'}) \frac{\partial G_{D}(\mathbf{x},\mathbf{x'})}{\partial n'}) da' \]

We also know that a Green function is defined by
\begin{align*}
G(\mathbf{x},\mathbf{x'}) &= \frac{1}{\left| \mathbf{x}-\mathbf{x'} \right|} + F(\mathbf{x},\mathbf{x'})\\
\nabla'^{2} G(\mathbf{x},\mathbf{x'}) &= -4\pi\delta(\mathbf{x}-\mathbf{x'})
\end{align*}
and that inside the specified volume (in this case $z\ge 0$) 
\[ \nabla'^{2} F(\mathbf{x},\mathbf{x'}) = 0 \]

We are then looking for the function $F_{D}(\mathbf{x},\mathbf{x'})$ that makes the Green function match the boundary conditions, but still satisfies the restrictions on their Laplacians.

To use the method of images, we notice that the first term of the Green function $1/|\mathbf{x}-\mathbf{x'}|$ is the potential of a point charge $q=4\pi\epsilon_{0}$. We then search for a corresponding distribution of imaginary charges (outside of the given volume) whose potential will satisfy the boundary conditions when added to $1/|\mathbf{x}-\mathbf{x'}|$.

This will replace the problem with the given boundary conditions to an equivalent problem where we only need to compute the potential of the charge distribution. The imaginary charges ``stand in" for the boundary because they produce the same effect, and by including them in the charge distribution they allow us to remove the boundary from the problem. We then set $F$ as the potential of these imaginary charges, and this will properly complete the Green function.

Because this imaginary charge distribution would be outside of the volume, we are guaranteed its potential, given by $F$, will satisfy Laplace's equation inside the volume ($\nabla'^{2} F(\mathbf{x},\mathbf{x'}) = 0$). We also know by definition of the Green function that $\nabla'^{2} G(\mathbf{x},\mathbf{x'}) = -4\pi\delta(\mathbf{x}-\mathbf{x'})$, so if we can find such a setup, the resulting Green function $G$ will provide a unique solution to the original potential problem.\\
\\
In cases of symmetrical boundaries, such as this flat sheet, this scheme may work out, and the imaginary charges needed to produce a usable $F$ will hopefully be simple to find. But in general cases with potentially wild boundaries they may not be. It is important to note that the boundaries are what matter. Since we are only finding the Green function, the charge distribution for the associated problem plays no role. We only have the first term of this function $1/|\mathbf{x}-\mathbf{x'}|$, which represents the potential of one charge, to account for with image potentials from $F$. So it is how our placement of the image charges (giving image potential $F$) account for the boundary conditions alone that form the bulk of the problem for finding the full Green function.

\paragraph{}
In this case we imagine a plane with the single charge above it, and then, given the symmetry of a plane, it does not seem a large step to guess that another charge opposite the plane from the first may produce the desired result.

We write the position of the charge above the plane in Cartesian coordinates $(\mathbf{x}=(x,y,z))$, and if for each coordinate $i$, we call $\Delta x_{i}=(x_{i}-x_{i}')$, our definition for $G$ becomes
\[ G_{D}(\mathbf{x},\mathbf{x'}) = \frac{1}{\sqrt{\Delta x^{2}+\Delta y^{2}+\Delta z^{2}}} + F_{D}(\mathbf{x},\mathbf{x'}) \]
We then use our guess and take $F$ to represent the potential of a charge equal and opposite to the first, and therefore of charge $-4\pi\epsilon_{0}$ located at $(x,y,-z)$ such that
\begin{align*}
F_{D}(\mathbf{x},\mathbf{x'}) &= \frac{-1}{\sqrt{\Delta x^{2}+\Delta y^{2}+(z+z')^{2}}}\\
G_{D}(\mathbf{x},\mathbf{x'}) &= \boxed{\frac{1}{\sqrt{\Delta x^{2}+\Delta y^{2}+\Delta z^{2}}} - \frac{1}{\sqrt{\Delta x^{2}+\Delta y^{2}+(z+z')^{2}}}}
\end{align*}
Notice that by construction this Green function satisfies the boundary conditions
\[ G_{D}(\mathbf{x}\ni(z=0),\mathbf{x'}) = \frac{1}{\sqrt{\Delta x^{2}+\Delta y^{2}+(z')^{2}}} - \frac{1}{\sqrt{\Delta x^{2}+\Delta y^{2}+(z')^{2}}} = 0 \]
It also satisfies the restrictions on the Laplacians as previously mentioned, and therefore represents the unique function needed for any charge distribution problem with this given boundary.

\paragraph{}
Now we turn our attention to finding the potential at some point in space $P=(\rho,\phi,z)$ given that inside a circle of radius $a$ centered at the origin on the plane $z'=0$ there is a fixed potential $\Phi=V$, and everywhere else on the same plane the potential is $\Phi=0$.

Let us begin by noticing that we are still using Dirichlet boundaries by specifying the potential on the plane. There is also no specified charge in the volume, only a fixed potential on the surface to account for. Thus our original motivating equation reduces as before, and more so: $G_{D}(\mathbf{x},\mathbf{x'})=0$ for $\mathbf{x'}$ on $S$, and for $\mathbf{x'}$ in $V$, $\rho(\mathbf{x'})=0$
\begin{align*}
\Phi &= \frac{1}{4\pi\epsilon_{0}} \int_{V} \rho(\mathbf{x'}) G(\mathbf{x},\mathbf{x'}) d^{3}x' + \frac{1}{4\pi} \oint_{S} \left[ G(\mathbf{x},\mathbf{x'}) \frac{\partial\Phi}{\partial n'}(\mathbf{x'}) -\Phi(\mathbf{x'}) \frac{\partial G(\mathbf{x},\mathbf{x'})}{\partial n'}) \right] da'\\
\Phi &= \frac{1}{4\pi} \oint_{S} \Phi(\mathbf{x'}) \frac{\partial G_{D}(\mathbf{x},\mathbf{x'})}{\partial n'}) da'
\end{align*}

Then, since we are using the same planar boundary, even with the new potential specifications, the restrictions on the Green function itself have not changed, allowing us to reuse the function found previously:
\begin{align*}
G_{D}(\mathbf{x},\mathbf{x'}) &= \frac{1}{\sqrt{\Delta x^{2}+\Delta y^{2}+\Delta z^{2}}} - \frac{1}{\sqrt{\Delta x^{2}+\Delta y^{2}+(z+z')^{2}}}\\
\frac{\partial G_{D}}{\partial n'}(\mathbf{x},\mathbf{x'}) = \frac{\partial G_{D}}{\partial z'}(\mathbf{x},\mathbf{x'}) &= \frac{z-z'}{[\Delta x^{2}+\Delta y^{2}+\Delta z^{2}]^{3/2}} + \frac{z+z'}{[\Delta x^{2}+\Delta y^{2}+(z+z')^{2}]^{3/2}}
\end{align*}
(notice the chain rule causes a change in sign) 

If we then switch to cylindrical coordinates
\begin{align*}
\Delta x^{2} = (x-x')^{2} = (\rho\cos\phi - \rho'\cos\phi')^{2} &= \rho^{2}\cos^{2}\phi - 2\rho\rho'\cos\phi\cos\phi' + (\rho')^{2}\cos^{2}\phi'\\
\Delta y^{2} = (y-y')^{2} = (\rho\sin\phi - \rho'\sin\phi')^{2} &= \rho^{2}\sin^{2}\phi - 2\rho\rho'\sin\phi\sin\phi' + (\rho')^{2}\sin^{2}\phi'\\
\Delta x^{2} + \Delta y^{2} &= \rho^{2} + (\rho')^{2} - 2\rho\rho'(\cos\phi\cos\phi' + \sin\phi\sin\phi')\\
&= \rho^{2} + (\rho')^{2} - 2\rho\rho'\cos(\phi-\phi')
\end{align*}
and thus
\begin{multline*}
\frac{\partial G_{D}}{\partial z'}(\mathbf{x},\mathbf{x'}) = \frac{z-z'}{[\rho^{2} + (\rho')^{2} - 2\rho\rho'\cos(\phi-\phi') + (z-z')^{2}]^{3/2}}\\ + \frac{z+z'}{[\rho^{2} + (\rho')^{2} - 2\rho\rho'\cos(\phi-\phi') + (z+z')^{2}]^{3/2}}
\end{multline*}
\begin{multline*}
\Phi = \frac{1}{4\pi} \oint_{S} \Phi(\mathbf{x'}) \left( \frac{z-z'}{[\rho^{2} + (\rho')^{2} - 2\rho\rho'\cos(\phi-\phi') + (z-z')^{2}]^{3/2}} \right. \\ \left. + \frac{z+z'}{[\rho^{2} + (\rho')^{2} - 2\rho\rho'\cos(\phi-\phi') + (z+z')^{2}]^{3/2}} \right) da'
\end{multline*}

Now, we are integrating over a surface that should contain the upper half space. So imagine a pillbox resting on the x-y plane that grows to infinity on the sides and upwards on the z-axis. The side components cancel as in any pillbox setup, the roof falls out because we specified $\Phi(\mathbf{x'}\ni(z\to\infty))=0$, and we are left with the surface on the plane $z'=0$. Also, $\Phi(\mathbf{x'})=0$ outside our circle of radius $a$, so all together our integral reduces to
\begin{align*}
\Phi &= \frac{1}{4\pi} \int_{(z'=0)} \frac{2z\Phi(\mathbf{x'})}{[\rho^{2} + (\rho')^{2} - 2\rho\rho'\cos(\phi-\phi') + z^{2}]^{3/2}} da'\\
\Phi &= \boxed{\frac{Vz}{2\pi} \int_{0}^{2\pi}\int_{0}^{a} \frac{\rho' d\rho'd\phi'}{[\rho^{2} + (\rho')^{2} - 2\rho\rho'\cos(\phi-\phi') + z^{2}]^{3/2}}}\\
\end{align*}

If we then set $\rho=0$ we can see what the potential would be along the z-axis
\begin{align*}
\Phi(\rho=0) &= \frac{Vz}{2\pi} \int_{0}^{2\pi}\int_{0}^{a} \frac{\rho' d\rho'd\phi'}{[(\rho')^{2} + z^{2}]^{3/2}}\\
&= Vz \int_{0}^{a} \frac{\rho' d\rho'}{[(\rho')^{2} + z^{2}]^{3/2}}\\
&= -Vz \left.\frac{1}{\sqrt{(\rho')^{2} + z^{2}}}\right|_{\rho'=0}^{a}\\
\Phi(\rho=0) &= \frac{-Vz}{\sqrt{a^{2} + z^{2}}} - \frac{-Vz}{\sqrt{z^{2}}} = \boxed{V\left( 1 - \frac{z}{\sqrt{a^{2}+z^{2}}} \right)}
\end{align*}
Although we really need to be careful here. $z/\sqrt{z^{2}}$ is actually $z/|z|$ which breaks us into two cases. Since we are above the x-y plane, $(z\ge 0)\to z/|z|=1$ and this equation is fine. If on the other hand we wished to include points below the plane, we would have to adjust our Green function for the new boundary, and the given reduction would change: $(z<0)\to z/|z|=-1$. But symmetry allows us to find the full solution with relative ease if we note that the potential at both $\pm z$ should be the same.
\[ \Phi(\rho=0) = \boxed{V\left( 1 - \frac{|z|}{\sqrt{a^{2}+z^{2}}} \right)} \]

Now, going back to the general potential in the upper half space, we consider the case where $(\rho^{2}+z^{2}\gg a^{2})$. So let us look again at the formula we calculated
\begin{align*}
\Phi &= \frac{Vz}{2\pi} \int_{0}^{2\pi}\int_{0}^{a} \frac{\rho' d\rho'd\phi'}{[\rho^{2} + (\rho')^{2} - 2\rho\rho'\cos(\phi-\phi') + z^{2}]^{3/2}}\\
&= \frac{Vz}{2\pi} \int_{0}^{2\pi}\int_{0}^{a} \left[\rho^{2} + z^{2} + (\rho')^{2} - 2\rho\rho'\cos(\phi-\phi')\right]^{-3/2}\rho' d\rho'd\phi'\\
&= \frac{Vz}{2\pi} \int_{0}^{2\pi}\int_{0}^{a} \left[(\rho^{2} + z^{2})\left[1 + \frac{(\rho')^{2} - 2\rho\rho'\cos(\phi-\phi')}{(\rho^{2} + z^{2})}\right]\right]^{-3/2}\rho' d\rho'd\phi'\\
&= \frac{Vz}{2\pi(\rho^{2} + z^{2})^{3/2}} \int_{0}^{2\pi}\int_{0}^{a} \left[1 + \frac{(\rho')^{2} - 2\rho\rho'\cos(\phi-\phi')}{(\rho^{2} + z^{2})}\right]^{-3/2}\rho' d\rho'd\phi'\\
\end{align*}

We then expand the integrand using the generalized binomial theorem.
\begin{align*}
(1+x)^{z} &= \sum_{k=0}^{\infty} {z\choose k} x^{k} = 1 + zx + \frac{z(z-1)}{2!}x^{2} + \cdots\\
\left[1 + \frac{(\rho')^{2} - 2\rho\rho'\cos(\phi-\phi')}{(\rho^{2} + z^{2})} \right]^{-3/2} &= \sum_{k=0}^{\infty} {-3/2\choose k} \left( \frac{(\rho')^{2} - 2\rho\rho'\cos(\phi-\phi')}{(\rho^{2} + z^{2})} \right)^{k}
\end{align*}

This only converges properly if 
\[ \left| \frac{(\rho')^{2} - 2\rho\rho'\cos(\phi-\phi')}{(\rho^{2} + z^{2})} \right| < 1 \]
or equivalently
\[ \left| (\rho')^{2} - 2\rho\rho'\cos(\phi-\phi') \right| < (\rho^{2} + z^{2}) \]
But since by the triangle inequality, and the fact that $\rho'\le a$
\[ \left| (\rho')^{2} - 2\rho\rho'\cos(\phi-\phi') \right| \le \left| (\rho')^{2} \right| + \left| 2\rho\rho'\cos(\phi-\phi') \right| \le a^{2} + 2\rho a\]
Then $a^{2} \ll \rho^{2} + z^{2}$ implies we can assume the smaller $2\rho a$ term drops out, and
\[ \left| (\rho')^{2} - 2\rho\rho'\cos(\phi-\phi') \right| \le a^{2} + 2\rho a < (\rho^{2} + z^{2}) \]
which verifies that the expansion is valid.\\
\\
So expanding using generalized binomials:
\begin{multline*} 
\left[1 + \frac{(\rho')^{2} - 2\rho\rho'\cos(\phi-\phi')}{(\rho^{2} + z^{2})}\right]^{-3/2} = 1 - \frac{3}{2}\left( \frac{(\rho')^{2} - 2\rho\rho'\cos(\phi-\phi')}{(\rho^{2} + z^{2})} \right)\\
+ \frac{(-\frac{3}{2})(-\frac{3}{2}-1)}{2!}\left( \frac{(\rho')^{2} - 2\rho\rho'\cos(\phi-\phi')}{(\rho^{2} + z^{2})} \right)^{2} - \cdots \end{multline*}

We then integrate this term by term 
\begin{align*} 
&\int_{0}^{2\pi}\int_{0}^{a} \left[1 + \frac{(\rho')^{2} - 2\rho\rho'\cos(\phi-\phi')}{(\rho^{2} + z^{2})}\right]^{-3/2} \rho' d\rho'd\phi'\\
=& \int_{0}^{2\pi}\int_{0}^{a} \rho' d\rho'd\phi'\\
&- \int_{0}^{2\pi}\int_{0}^{a} \frac{3}{2}\left( \frac{(\rho')^{2} - 2\rho\rho'\cos(\phi-\phi')}{(\rho^{2} + z^{2})} \right) \rho' d\rho'd\phi'\\
&+ \int_{0}^{2\pi}\int_{0}^{a} \frac{(-\frac{3}{2})(-\frac{3}{2}-1)}{2!}\left( \frac{(\rho')^{2} - 2\rho\rho'\cos(\phi-\phi')}{(\rho^{2} + z^{2})} \right)^{2} \rho' d\rho'd\phi'\\
&- \cdots\\
\end{align*}

Separating each term to make the integration easier to follow:
\[ \int_{0}^{2\pi}\int_{0}^{a} \rho' d\rho'd\phi' = \boxed{\pi a^{2}} \]
\\
\[ \int_{0}^{2\pi}\int_{0}^{a} \frac{-3}{2}\left( \frac{(\rho')^{2} - 2\rho\rho'\cos(\phi-\phi')}{(\rho^{2} + z^{2})} \right) \rho' d\rho'd\phi' \]
\begin{align*} 
&= \frac{-3}{2(\rho^{2} + z^{2})} \int_{0}^{2\pi}\int_{0}^{a} \left[ (\rho')^{3} - 2\rho(\rho')^{2}\cos(\phi-\phi') \right] d\rho'd\phi'\\
&= \frac{-3}{2(\rho^{2} + z^{2})} \left[ \frac{\pi a^{4}}{2} - 2\rho \int_{0}^{2\pi}\int_{0}^{a} \left[ (\rho')^{2}\cos(\phi-\phi') \right] d\rho'd\phi' \right]\\
&= \frac{-3}{2(\rho^{2} + z^{2})} \left[ \frac{\pi a^{4}}{2} - \frac{2a\rho}{3} \int_{0}^{2\pi} \cos(\phi-\phi') d\phi' \right]\\
&= \frac{-3}{2(\rho^{2} + z^{2})} \left[ \frac{\pi a^{4}}{2} + \frac{2a\rho}{3} \left[ \sin(\phi-2\pi) - \sin(\phi) \right] \right]\\
&= \frac{-3}{2(\rho^{2} + z^{2})} \left[ \frac{\pi a^{4}}{2} \right] = \boxed{\frac{-3\pi a^{4}}{4(\rho^{2} + z^{2})}}\\
\end{align*}

\[ \int_{0}^{2\pi}\int_{0}^{a} \frac{(-\frac{3}{2})(-\frac{3}{2}-1)}{2!}\left( \frac{(\rho')^{2} - 2\rho\rho'\cos(\phi-\phi')}{(\rho^{2} + z^{2})} \right)^{2} \rho' d\rho'd\phi' \]
\begin{align*}
&= \frac{15}{8(\rho^{2} + z^{2})^{2}} \int_{0}^{2\pi}\int_{0}^{a} \left[ (\rho')^{2} - 2\rho\rho'\cos(\phi-\phi') \right]^{2} \rho' d\rho'd\phi'\\
&= \frac{15}{8(\rho^{2} + z^{2})^{2}} \int_{0}^{2\pi}\int_{0}^{a} \left[ (\rho')^{4} - 4\rho(\rho')^{3}\cos(\phi-\phi') + 4\rho^{2}(\rho')^{2}\cos^{2}(\phi-\phi') \right] \rho' d\rho'd\phi'\\
&= \frac{15}{8(\rho^{2} + z^{2})^{2}} \int_{0}^{2\pi}\int_{0}^{a} \left[ (\rho')^{5} - 4\rho(\rho')^{4}\cos(\phi-\phi') + 4\rho^{2}(\rho')^{3}\cos^{2}(\phi-\phi') \right] d\rho'd\phi'\\
&= \frac{3}{8(\rho^{2} + z^{2})^{2}} \left[ \frac{\pi a^{6}}{3} - 4\rho \int_{0}^{2\pi}\int_{0}^{a} \left[ (\rho')^{4}\cos(\phi-\phi') - \rho(\rho')^{3}\cos^{2}(\phi-\phi') \right] d\rho'd\phi' \right]\\
&= \frac{15}{8(\rho^{2} + z^{2})^{2}} \left[ \frac{\pi a^{6}}{3} - 4\rho \left( \frac{a^{5}}{5} \int_{0}^{2\pi} \cos(\phi-\phi') d\phi' - \rho \int_{0}^{2\pi}\int_{0}^{a} (\rho')^{3}\cos^{2}(\phi-\phi') d\rho'd\phi' \right) \right]\\
&= \frac{15}{8(\rho^{2} + z^{2})^{2}} \left[ \frac{\pi a^{6}}{3} + 4\rho^{2} \int_{0}^{2\pi}\int_{0}^{a} (\rho')^{3}\cos^{2}(\phi-\phi') d\rho'd\phi' \right]\\
&= \frac{15}{8(\rho^{2} + z^{2})^{2}} \left[ \frac{\pi a^{6}}{3} + \rho^{2}a^{4} \int_{0}^{2\pi} \cos^{2}(\phi-\phi') d\phi' \right] = \frac{15}{8(\rho^{2} + z^{2})^{2}} \left[ \frac{\pi a^{6}}{3} - \pi\rho^{2}a^{4} \right]\\
&= \frac{15}{8(\rho^{2} + z^{2})^{2}} \left[ \frac{\pi a^{6} - 3\pi\rho^{2}a^{4}}{3} \right] = \boxed{\frac{5\pi a^{4}(a^{2} - 3\rho^{2})}{8(\rho^{2} + z^{2})^{2}}}
\end{align*}

Adding these back together shows
\begin{multline*}
\int_{0}^{2\pi}\int_{0}^{a} \left[1 + \frac{(\rho')^{2} - 2\rho\rho'\cos(\phi-\phi')}{(\rho^{2} + z^{2})}\right]^{-3/2} \rho' d\rho'd\phi'\\
= \pi a^{2} - \frac{3\pi a^{4}}{4(\rho^{2} + z^{2})} + \frac{5\pi a^{4}(a^{2} - 3\rho^{2})}{8(\rho^{2} + z^{2})^{2}} - \cdots = \boxed{\pi a^{2} \left[ 1 - \frac{3a^{2}}{4(\rho^{2} + z^{2})} + \frac{5a^{2}(a^{2} - 3\rho^{2})}{8(\rho^{2} + z^{2})^{2}} - \cdots \right]}
\end{multline*}

And thus
\begin{multline*}
\Phi = \frac{Vz}{2\pi(\rho^{2} + z^{2})^{3/2}} \int_{0}^{2\pi}\int_{0}^{a} \left[1 + \frac{(\rho')^{2} - 2\rho\rho'\cos(\phi-\phi')}{(\rho^{2} + z^{2})}\right]^{-3/2}\rho' d\rho'd\phi'\\
= \frac{Vz}{2\pi(\rho^{2} + z^{2})^{3/2}}\left[ \pi a^{2} \left[ 1 - \frac{3a^{2}}{4(\rho^{2} + z^{2})} + \frac{5a^{2}(a^{2} - 3\rho^{2})}{8(\rho^{2} + z^{2})^{2}} - \cdots \right] \right]\\
= \boxed{\frac{Vza^{2}}{2(\rho^{2} + z^{2})^{3/2}}\left[ 1 - \frac{3a^{2}}{4(\rho^{2} + z^{2})} + \frac{5a^{2}(a^{2} - 3\rho^{2})}{8(\rho^{2} + z^{2})^{2}} - \cdots \right]}
\end{multline*}
which is what we were looking for.

Finally we check that this is consistent with what we got previously by setting $\rho=0$:
\begin{align*}
\Phi(\rho=0) &= \frac{Vza^{2}}{2(z^{2})^{3/2}}\left[ 1 - \frac{3a^{2}}{4(z^{2})} + \frac{5a^{2}(a^{2})}{8(z^{2})^{2}} - \cdots \right]\\
&= \frac{Va^{2}}{2z^{2}}\left[ 1 - \frac{3a^{2}}{4z^{2}} + \frac{5a^{4}}{8z^{4}} - \cdots \right]\\
&= V\left[ \frac{1}{2}\left(\frac{a^{2}}{z^{2}}\right) - \frac{3}{8}\left(\frac{a^{2}}{z^{2}}\right)^{2} + \frac{5}{16}\left(\frac{a^{2}}{z^{2}}\right)^{3} - \cdots \right]\\
&= V\left[ 1 - \left( 1 - \frac{1}{2}\left(\frac{a^{2}}{z^{2}}\right) + \frac{3}{4\cdot2}\left(\frac{a^{2}}{z^{2}}\right)^{2} - \frac{5\cdot3}{8\cdot6}\left(\frac{a^{2}}{z^{2}}\right)^{3} + \cdots \right)\right]\\
&= V\left[ 1 - \left( 1 - \frac{1}{2} \left(\frac{a^{2}}{z^{2}}\right) + \frac{\frac{3}{4}}{2!} \left(\frac{a^{2}}{z^{2}}\right)^{2} + \frac{\frac{-15}{8}}{3!} \left(\frac{a^{2}}{z^{2}}\right)^{3} + \cdots \right)\right]\\
&= V\left[ 1 - \left( 1 - \frac{1}{2} \left(\frac{a^{2}}{z^{2}}\right) + \frac{\frac{-1}{2}(\frac{-3}{2})}{2!} \left(\frac{a^{2}}{z^{2}}\right)^{2} + \frac{\frac{-1}{2}(\frac{-3}{2})\frac{-5}{2}}{3!} \left(\frac{a^{2}}{z^{2}}\right)^{3} + \cdots \right)\right]\\
&= V\left[ 1 - \left( 1 - \frac{1}{2} \left(\frac{a^{2}}{z^{2}}\right) + \frac{\frac{-1}{2}(\frac{-1}{2}-1)}{2!} \left(\frac{a^{2}}{z^{2}}\right)^{2} + \frac{\frac{-1}{2}(\frac{-1}{2}-1)(\frac{-1}{2}-2)}{3!}\left(\frac{a^{2}}{z^{2}}\right)^{3} + \cdots \right)\right]
\end{align*}

So if we recognize that this is a binomial expansion that converges when \[\left|\frac{a^{2}}{z^{2}}\right|<1 \implies a^{2} < z^{2} \implies a < |z| \]
We can substitute and find
\begin{align*}
\Phi(\rho=0) &= V\left[ 1 - \left( 1 + \frac{a^{2}}{z^{2}} \right)^{-1/2} \right]\\
&= V\left[ 1 - \left(\frac{z^{2}+a^{2}}{z^{2}} \right)^{-1/2} \right]\\
\Phi(\rho=0) &= \boxed{V\left[ 1 - \frac{|z|}{\sqrt{a^{2}+z^{2}}} \right]}
\end{align*}
which matches up with our previous answer under the given conditions.

\setcounter{problem}{10}
\begin{problem}
A line charge with linear charge density $\tau$ is placed parallel to, and a distance $R$ away from, the axis of a conducting cylinder of radius $b$ held at a fixed voltage such that the potential vanishes at infinity. Find
\begin{itemize}
\item The magnitude and position of the image charge(s)
\item The potential at any point (expressed in polar coordinates with the origin at the axis of the cylinder and the direction from the origin to the line charge as the x axis), including the asymptotic form far from the cylinder
\item The induced surface-charge density, and plot it as a function of angle for $R/b = 2,4$ in units of $\tau/2\pi b$
\item The force per unit length on the line charge
\end{itemize}
\end{problem}

So let us first imagine the scenario. Put the axis of a cylinder of radius $b$ along the z-axis, and regulate a constant voltage $v$ on its surface. Add a line charge with charge density $\tau$ parallel to the cylinder and a distance $R$ away from its axis. If we keep the surface of the cylinder fixed at this constant potential, and say that the potential drops off as you move away from the axis, we can imagine this as a boundary value problem as before.

\paragraph{}
So it seems easiest to work in cylindrical coordinates given the geometry of the problem. So as usual, call $\mathbf{x}=(\rho,\phi,z)$ the observation point, and $\mathbf{x'}=(\rho',\phi',z')$ the points from which our integration components are calculated.

It also seems reasonable to imagine a parallel imaginary line charge on the same radial vector as the given one, but inside the cylinder. This we will say is a distance $R'$ away from the axis of the cylinder, and carries a charge density of $\tau'$.

So we start by finding the potential of each of the line charges, real and imaginary, and then find the position and magnitude of the image line such that the boundary conditions are satisfied.

Construct a Gaussian cylinder of radius $\alpha$ and length $L$ around our line charge. This makes the radial field the line produces constant along the curved surface of the Gaussian cylinder, and assures the field also points normal to it at every point. The top and bottom cancel for sufficiently large $L$, and thus Gauss' law gives us the magnitude of e the field at he radius $\alpha$ away from the line:
\begin{align*}
\oint_{S} \mathbf{E}\cdot\mathbf{n} da &= \frac{1}{\epsilon_{0}} \int_{V} \rho(\mathbf{x'}) d^{3}x\\
\left| \mathbf{E} \right| \oint_{S} da &= \frac{\tau L}{\epsilon_{0}}\\
\left| \mathbf{E} \right| 2\pi \alpha L &= \frac{\tau L}{\epsilon_{0}}\\
\left| \mathbf{E} \right| &= \frac{\tau}{2\pi \alpha \epsilon_{0}}\\
\end{align*}
Where this was assumed to point radially away from the line charge.

Thus we use our definition of the potential, and if we call $\mathbf{\hat{\alpha}}$ the radial vector pointing away from the line, we can dot our gradient to show that the potential at the same radius is
\begin{align*}
\mathbf{E}&=-\nabla\Phi\\
\mathbf{E}\cdot\mathbf{\hat{\alpha}}&=\frac{-\partial\Phi}{\partial \alpha}\\
\frac{\partial\Phi}{\partial \alpha} &= \frac{-\tau}{2\pi \alpha \epsilon_{0}}\\
\Phi &= \frac{-\tau}{2\pi\epsilon_{0}} \int \frac{d\alpha}{\alpha} = \frac{-\tau}{2\pi\epsilon_{0}}\ln\alpha\\
\end{align*}
where we can have no integration constant because potential is arbitrarily set against some constant value. So imagine taking the integral from some arbitrarily defined point, say the surface of the cylinder, to infinity. This removes the constant of integration, and makes life a little easier.

If we construct a Gaussian cylinder of radius $\beta$ around the image line in a similar fashion, we will come up with an equivalent formula for its potential a radius $\beta$ away from it:
\[ \Phi' = \frac{-\tau'}{2\pi\epsilon_{0}}\ln\beta \]

We then recall the distance formula in cylindrical coordinates as computed in the previous problem
\begin{align*}
\Delta x^{2} = (x-x')^{2} = (\rho\cos\phi - \rho'\cos\phi')^{2} &= \rho^{2}\cos^{2}\phi - 2\rho\rho'\cos\phi\cos\phi' + (\rho')^{2}\cos^{2}\phi'\\
\Delta y^{2} = (y-y')^{2} = (\rho\sin\phi - \rho'\sin\phi')^{2} &= \rho^{2}\sin^{2}\phi - 2\rho\rho'\sin\phi\sin\phi' + (\rho')^{2}\sin^{2}\phi'\\
\Delta x^{2} + \Delta y^{2} &= \rho^{2} + (\rho')^{2} - 2\rho\rho'(\cos\phi\cos\phi' + \sin\phi\sin\phi')\\
&= \rho^{2} + (\rho')^{2} - 2\rho\rho'\cos(\phi-\phi')
\end{align*}
\[ \Delta\mathbf{x} = \sqrt{\Delta x^{2} + \Delta y^{2} + \Delta z^{2}} = \sqrt{\rho^{2} + (\rho')^{2} - 2\rho\rho'\cos(\phi-\phi') + (z-z')^{2}} \]

Since we imagined the line charge and image line charge to be on the same radial vector, their angles are the same, and we can imagine them on the x-axis so $\phi'=0$. Since they are considered infinitely long, we can ignore the z components of their distances to the observation point by just picking the same one as the observation point itself ($[z'=z]\to[(z-z')^{2}=0]$). If we can find the charge density of the image line, we then have only the radial position of the line and image lines and to account for, which we called $R$ and $R'$ respectively.

So substitute in the appropriate values for $\alpha$ and $\beta$ and add together the potentials to get
\begin{align*}
\Phi_{Total} &= \frac{-\tau}{2\pi\epsilon_{0}}\ln\alpha + \frac{-\tau'}{2\pi\epsilon_{0}}\ln\beta\\
&= \frac{-1}{2\pi\epsilon_{0}}\left[ \tau\ln\alpha + \tau'\ln\beta \right]\\
&= \frac{-1}{2\pi\epsilon_{0}}\ln(\alpha^{\tau}\beta^{\tau'})\\
\Phi_{Total} &= \frac{-1}{2\pi\epsilon_{0}}\ln([\rho^{2} + R^{2} - 2R\rho\cos\phi]^{\tau/2}[\rho^{2} + (R')^{2} - 2R'\rho\cos\phi]^{\tau'/2})
\end{align*}

So to make this a little more palatable let us take $\rho\to\infty$ where then $\Phi\to 0$. We will leave the $\rho^{2}$ terms that dominate in this case to isolate $\tau$ and $\tau'$:
\begin{align*}
0 &= \frac{-1}{2\pi\epsilon_{0}}\ln((\rho^{2})^{\tau/2}(\rho^{2})^{\tau'/2})\\
0 &= \ln(\rho^{\tau+\tau'})\\
\tau+\tau' &= 0\\
\tau' &= \boxed{-\tau}
\end{align*}

Substituting this in gives
\begin{align*}
\Phi_{Total} &= \frac{-1}{2\pi\epsilon_{0}} \ln(\alpha^{\tau}\beta^{\tau'})\\
&= \frac{-\tau}{2\pi\epsilon_{0}} \ln\left(\frac{\alpha}{\beta}\right)\\
&= \frac{-\tau}{2\pi\epsilon_{0}} \ln\left(\frac{\sqrt{\rho^{2} + R^{2} - 2R\rho\cos\phi}}{\sqrt{\rho^{2} + (R')^{2} - 2R'\rho\cos\phi}}\right)\\
\Phi_{Total} &= \frac{-\tau}{4\pi\epsilon_{0}}\ln\left(\frac{\rho^{2} + R^{2} - 2R\rho\cos\phi}{\rho^{2} + (R')^{2} - 2R'\rho\cos\phi}\right)
\end{align*}
and if we then use our main boundary condition $\rho=b\to\Phi=V$, we can isolate $\cos\phi$ and take advantage of symmetry
\begin{align*}
V &= \frac{-\tau}{4\pi\epsilon_{0}}\ln\left(\frac{b^{2} + R^{2} - 2Rb\cos\phi}{b^{2} + (R')^{2} - 2R'b\cos\phi}\right)\\
\frac{-4\pi\epsilon_{0}V}{\tau} &= \ln\left(\frac{b^{2} + R^{2} - 2Rb\cos\phi}{b^{2} + (R')^{2} - 2R'b\cos\phi}\right)\\
\exp\left(\frac{-4\pi\epsilon_{0}V}{\tau}\right) &= \frac{b^{2} + R^{2} - 2Rb\cos\phi}{b^{2} + (R')^{2} - 2R'b\cos\phi}\\
\exp\left(\frac{-4\pi\epsilon_{0}V}{\tau}\right)(b^{2} + (R')^{2} - 2R'b\cos\phi) &= b^{2} + R^{2} - 2Rb\cos\phi\\
\exp\left(\frac{-4\pi\epsilon_{0}V}{\tau}\right)(b^{2} + (R')^{2}) - (b^{2} + R^{2}) &= \exp\left(\frac{-4\pi\epsilon_{0}V}{\tau}\right)(2R'b\cos\phi) - 2Rb\cos\phi\\
\exp\left(\frac{-4\pi\epsilon_{0}V}{\tau}\right)(b^{2} + (R')^{2}) - (b^{2} + R^{2}) &= \left[\exp\left(\frac{-4\pi\epsilon_{0}V}{\tau}\right)(2R') - 2R\right]b\cos\phi\\
b\cos\phi &= \frac{\exp\left(\frac{-4\pi\epsilon_{0}V}{\tau}\right)(b^{2} + (R')^{2}) - (b^{2} + R^{2})}{\exp\left(\frac{-4\pi\epsilon_{0}V}{\tau}\right)(2R') - 2R} 
\end{align*}

Then since the potential on the cylinder must be the same from any angle,
\[ b\cos\phi = \frac{\exp\left(\frac{-4\pi\epsilon_{0}V}{\tau}\right)(b^{2} + (R')^{2}) - (b^{2} + R^{2})}{\exp\left(\frac{-4\pi\epsilon_{0}V}{\tau}\right)(2R') - 2R} \]
must be true for any angle $\phi$ from which we view the cylinder.

Thus the numerator must be zero to account for when we look down the y-axis $(\phi=\pi/2)\to(\cos\phi=0)$, and the numerator and denominator must be proportional to account for looking down the x-axis $(\phi=0)\to(\cos\phi=1)$. This means the denominator must be zero as well since both these values of $\phi$ must give the same result. Note that this is only the case where we are looking at the potential on the cylinder. If we were accounting for the potential anywhere in space, there would not be the symmetry we are making use of, and we would have to account for the line and image effects separately.

\paragraph{}
But given our situation currently we proceed by setting the numerator and denominator to zero to find $R'$
\begin{align*}
\exp\left(\frac{-4\pi\epsilon_{0}V}{\tau}\right)(b^{2} + (R')^{2}) - (b^{2} + R^{2}) &= 0 & \exp\left(\frac{-4\pi\epsilon_{0}V}{\tau}\right)(2R') - 2R&=0\\
\exp\left(\frac{-4\pi\epsilon_{0}V}{\tau}\right)(b^{2} + (R')^{2}) &= b^{2} + R^{2} & \exp\left(\frac{-4\pi\epsilon_{0}V}{\tau}\right)&=\frac{R}{R'}\\
\frac{R}{R'}(b^{2} + (R')^{2}) &= b^{2} + R^{2}\\
\frac{b^{2}}{R'} + R' &= \frac{b^{2}}{R} + R\\
b^{2} \left(\frac{1}{R'} - \frac{1}{R}\right) &= R - R'\\
b^{2} \left(\frac{R-R'}{RR'}\right) &= R - R'\\
b^{2} &= RR'\\
R' &= \boxed{\frac{b^{2}}{R}}
\end{align*}

We then have the full solution to the potential by substituting in the new found values. I now drop the ``Total" subscript and flip the numerator and denominator in the log with the negative constant, as I should have done previously.
\[ \Phi = \boxed{\frac{\tau}{4\pi\epsilon_{0}}\ln\left(\frac{\rho^{2} + (\frac{b^{2}}{R})^{2} - \frac{2b^{2}}{R}\rho\cos\phi}{\rho^{2} + R^{2} - 2R\rho\cos\phi}\right)} \]
If we then set this up to expand $\ln$ in a power series
\begin{align*}
\Phi &= \frac{\tau}{4\pi\epsilon_{0}}\ln\left(1-1+\frac{\rho^{2} + (\frac{b^{2}}{R})^{2} - \frac{2b^{2}}{R}\rho\cos\phi}{\rho^{2} + R^{2} - 2R\rho\cos\phi}\right)\\
&= \frac{\tau}{4\pi\epsilon_{0}}\ln\left(1+\frac{\rho^{2} + (\frac{b^{2}}{R})^{2} - \frac{2b^{2}}{R}\rho\cos\phi}{\rho^{2} + R^{2} - 2R\rho\cos\phi} - \frac{\rho^{2} + R^{2} - 2R\rho\cos\phi}{\rho^{2} + R^{2} - 2R\rho\cos\phi}\right)\\
&= \frac{\tau}{4\pi\epsilon_{0}}\ln\left(1+ \frac{(\frac{b^{2}}{R})^{2} - R^{2} - \left( \frac{b^{2}}{R} - R \right)2\rho\cos\phi}{\rho^{2} + R^{2} - 2R\rho\cos\phi}\right)\\
\Phi &= \frac{\tau}{4\pi\epsilon_{0}}\ln\left(1+ \frac{(b^{4} - R^{4})\frac{1}{R^{2}} + (R^{2}-b^{2})\frac{2\rho}{R}\cos\phi}{\rho^{2} + R^{2} - 2R\rho\cos\phi}\right)
\end{align*}
and taking $\rho\gg R$, and $\rho\gg b$, we can approximate the potential very far away from the system and drop all but the largest terms
\begin{align*}
\Phi &= \frac{\tau}{4\pi\epsilon_{0}}\ln\left(1+ \frac{(b^{4} - R^{4})\frac{1}{R^{2}} + (R^{2}-b^{2})\frac{2\rho}{R}\cos\phi}{\rho^{2} + R^{2} - 2R\rho\cos\phi}\right)\\
\Phi &= \frac{\tau}{4\pi\epsilon_{0}}\ln\left(1+ \frac{(R^{2}-b^{2})\frac{2\rho}{R}\cos\phi}{\rho^{2}}\right)\\
\Phi &= \frac{\tau}{4\pi\epsilon_{0}}\ln\left(1+ \frac{(R^{2}-b^{2})2\cos\phi}{R\rho}\right)
\end{align*}

We then use the McLauren series expansion of $\ln(1+x)$
\[ \ln(1+x) = x - \frac{x^2}{2!} + \frac{2x^3}{3!} - \cdots \]
to show that for $\ln(1+x)\approx x$ we can use the simpler approximation
\[ \Phi = \boxed{\frac{\tau}{2\pi\epsilon_{0}}\left( \frac{(R^{2}-b^{2})\cos\phi}{R\rho} \right)} \]

\setcounter{problem}{22}
\begin{problem}
A hollow cube has conducting walls defined by six planes $x=0,y=0,z=0$ and $x=a,y=a,z=a$. The walls $z=0$ and $z=a$ are held at a constant potential $V$. The other four sides are at zero potential.
\begin{itemize}
\item Find the potential $\Phi(x,y,z)$ at any point in the cube
\item Evaluate the potential at the center of the cube numerically, accurate to three significant figures. How many terms in the series is it necessary to keep in order to attain this accuracy? Compare your numerical result with the average value of the potential on the walls. See Problem 2.28
\item Find the surface-charge density on the wall $z=a$
\end{itemize}
\end{problem}
\end{document}