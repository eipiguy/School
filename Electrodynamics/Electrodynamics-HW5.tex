\documentclass{article}

\usepackage{amsmath,amsfonts,amssymb,amsthm}
\usepackage{enumerate}
\usepackage{graphicx}


% Opening
\title{Electrodynamics HW5\\
Ch4 - 1,3,7 (pg135)}
\author{Neal D. Nesbitt}

\begin{document}
\maketitle

\theoremstyle{definition}
\newtheorem{problem}{Problem}
\newtheorem{solution}{Solution}[problem]

\begin{problem}
Calculate the multipole moments $q_{lm}$ of the charge distributions shown as parts a and b. Try to obtain results for the nonvanishing moments valid for all $l$, but in each case find the first two sets of nonvanishing moments at the very least.
\begin{enumerate}[(a)]
\item For the charge distribution of the second set b, write down the multipole expansion for the potential. Keeping only the lowest order term in the expansion, plot the potential in the x-y plane as a function of distance from the origin for distances greater than $a$.
\item Calculate directly from Coulomb's Law the exact potential for b in the x-y plane. Plot it as a function of distance and compare with the result performed in part c.
\end{enumerate}
Divide out the asymptotic form in parts c and d to see the behavior at large distances more clearly.
\end{problem}

\begin{solution}
content...
\end{solution}

\setcounter{problem}{2}
\begin{problem}
content...
\end{problem}

\begin{solution}
content...
\end{solution}

\setcounter{problem}{6}
\begin{problem}
content...
\end{problem}

\begin{solution}
content...
\end{solution}

\end{document}