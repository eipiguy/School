\documentclass{article}

\usepackage{amsmath,amsfonts,amssymb,amsthm}
\usepackage{enumerate}
\usepackage{listings,color}
\usepackage{graphicx}


% Opening
\title{Complex Analysis HW1\\}
\author{Neal D. Nesbitt}

\begin{document}
\maketitle

\theoremstyle{definition}
\newtheorem{problem}{Problem}
\newtheorem{solution}{Solution}[problem]
\renewcommand{\thesolution}{\theproblem}

\begin{problem}
Show that
\begin{enumerate}[(a)]
\item \[ \text{Re}(iz) = -\text{Im}(z) \]
\item \[ \text{Im}(iz) = -\text{Re}(z) \]
\end{enumerate}
\end{problem}


\begin{solution}
\begin{enumerate}[(a)]
\item
\begin{proof}
Let $z\in\mathbb{C}$ such that $z=x+iy$ for some $x,y\in\mathbb{R}$.

\paragraph{}
Then since $i^{2}=-1$,
\[ \text{Re}(iz) = \text{Re}(i(x+iy)) = \text{Re}(-y+ix) = -y \]
and
\[ -\text{Im}(z) = -\text{Im}(x+iy) = -y \]
implying
\[ \text{Re}(iz) = -\text{Im}(z) \]
\end{proof}
\item
\begin{proof}
Let $z\in\mathbb{C}$ such that $z=x+iy$ for some $x,y\in\mathbb{R}$.

\paragraph{}
Then since $i^{2}=-1$,
\[ \text{Im}(iz) = \text{Im}(i(x+iy)) = \text{Im}(-y+ix) = x \]
and
\[ \text{Re}(z) = \text{Re}(x+iy) = x \]
implying
\[ \text{Im}(iz) = \text{Re}(z) \]
\end{proof}
\end{enumerate}
\end{solution}

\begin{problem}
Solve the equation $z^{2}+z+1=0$ for $z=(x,y)$ by writing
\[ (x,y)(x,y) + (x,y) + (1,0) = (0,0) \]
\end{problem}

\begin{solution}
Let our notation be as above, and then work out $z^{2}$, and match the real and imaginary parts to find:
\begin{align*}
(x,y)(x,y) + (x,y) + (1,0) &= (0,0)\\
(x^{2}-y^{2},2xy) + (x,y) + (1,0) &= (0,0)\\
x^{2}-y^{2} + x + 1 &= 0	&	2xy + y &= 0\\
x^{2}+x &= y^{2}-1	&	(2x+1)y &= 0
\end{align*}

The imaginary component's equation implies that potential solutions have components $x=-1/2$ and $y=0$.

\paragraph{}
So beginning with the first possibility we plug $x=-1/2$ back into the real component's equation to see
\begin{align*}
x^{2}+x &= y^{2}-1 \\
\frac{1}{4}-\frac{1}{2} &= y^{2}-1 \\
\frac{-1}{4} &= y^{2}-1 \\
\frac{3}{4} &= y^{2} \\
y &= \pm\sqrt{\frac{3}{4}} = \pm\sqrt{3}/2
\end{align*}
giving the pair of complex solutions $z=\left(-1\pm i\sqrt{3}\right)/2$.

\paragraph{}
Similarly, using $y=0$ in the same equation would show
\begin{align*}
x^{2}+x &= -1 \\
x^{2}+x+1 &= 0 \\
x &= \left( -1 \pm \sqrt{1-4} \right)/2 \\
x &= \left( -1 \pm \sqrt{-3} \right)/2 \\
x &= \left( -1 \pm i\sqrt{3} \right)/2 \\
\end{align*}
giving the same pair of solutions, but requiring the quadratic formula.
\end{solution}

\begin{problem}
Reduce each of these quantities to a real number:
\begin{enumerate}[(a)]
\item 
\[ \frac{1+i2}{3-i4} + \frac{2-i}{5i} \]
\item $(1-i)^{4}$
\end{enumerate}
\end{problem}

\begin{solution}
\begin{enumerate}[(a)]
\item Start by multiplying the first fraction by th complex conjugate of the denominator over itself. Then simplify:
\begin{align*}
\frac{1+i2}{3-i4} + \frac{2-i}{5i} &= \frac{(1+i2)(3-i4)}{(9+16)} + \frac{1+i2}{5} \\
&= \frac{3+8+i(6-4)}{9+16} + \frac{1+i2}{5} \\
&= \frac{11+i2}{25} + \frac{1+i2}{5} \\
&= \frac{11+i2}{25} + \frac{5+i10}{25} \\
&= \frac{11+i2}{25} + \frac{5+i10}{25} \\
&= \boxed{\frac{16+i12}{25}}
\end{align*}
\item
Using $i^{2}=-1$, we can find
\begin{align*}
(1-i)^{4} &= \left( (1-i)^{2} \right)^{2} \\
&= \left( -2i \right)^{2} = \boxed{-4}
\end{align*}
\end{enumerate}
\end{solution}

\begin{problem}
Verify that $\sqrt{2}|z| \ge |\text{Re}(z)| + |\text{Im}(z)|$.
\end{problem}

\begin{solution}
I we call $z=x+iy$, then we can see
\begin{align*}
(|x|-|y|)^{2} &\ge 0 \\
|x|^{2} +|y|^{2} -2|x||y| &\ge 0 \\
|x|^{2} +|y|^{2} &\ge 2|x||y| \\
|z|^{2} &\ge 2|x||y| \\
2|z|^{2} &\ge 2|x||y| + |x|^{2} +|y|^{2} \\
2|z|^{2} &\ge \left(|x| +|y|\right)^{2} \\
\sqrt{2}|z| &\ge |x| +|y| = |\text{Re}(z)| + |\text{Im}(z)| \\
\end{align*}
\end{solution}

\begin{problem}
In each case, sketch the set of points determined by the given condition:
\begin{enumerate}[(a)]
\item $|z-1+i|=1$
\item $|z+i|\le 3$
\item $|z-4i|\ge 4$
\end{enumerate}
\end{problem}

\begin{solution}
\begin{enumerate}[(a)]
\item
\end{enumerate}
\end{solution}

\begin{problem}
Use properties of conjugates and moduli to show that
\begin{enumerate}[(a)]
\item $\overline{z+3i} = z-3i$
\item $\overline{iz}=-i\overline{z}$
\end{enumerate}
\end{problem}

\begin{solution}
\begin{enumerate}[(a)]
\item $\overline{z+3i} = \overline{z}-3i$

If we call $z=x+iy$, then 
\[ \overline{z+3i} = \overline{x+iy+3i} = \overline{x+i(y+3)} = x-i(y+3) = x-iy-3i = \overline{z}-3i \]
\item Similarly 
\[ \overline{iz} = \overline{i(x+iy)} = \overline{ix-y} = -ix-y = -i(x+iy) -i\overline{z} \]
\end{enumerate}
\end{solution}

\begin{problem}
Sketch the set of points determined by the condition $\text{Re}(\bar{z}-i)=2$.
\end{problem}

\begin{solution}
If $z=x+iy$, then
\[ \text{Re}(\bar{z}-i)=\text{Re}(x-iy-i)=\text{Re}(x-i(y+1))=x=2 \]
Showing that this is the vertical line in the complex plane of all points with real part 2.
\end{solution}

\begin{problem}
By factoring $z^{4}-4z^{2}+3$ into two quadratic factors and using inequality (8), Section 4, show that if $z$ lies on the circle $|z|=2$, then
\[ \left|\frac{1}{z^{4}-4z^{2}+3}\right| \le \frac{1}{3} \]
\end{problem}

\begin{solution}
Take $z$ such that $|z|=2$. Then $|z|=|x+iy|=\sqrt{x^{2}+y^{2}}=2$. So let us examine the following:
\[ \left|\frac{1}{z^{4}-4z^{2}+3}\right| = \frac{1}{|z^{2}-3||z^{2}-1|} \]

Then by our given inequality, $|z^{2}-3| \ge \left||z|^{2}-3\right|$ and $|z^{2}-1| \ge \left||z|^{2}-1\right|$ which implies
\[ \frac{1}{|z^{2}-3||z^{2}-1|} \le \frac{1}{\left||z|^{2}-3\right|\left||z|^{2}-1\right|} = \frac{1}{\left|4-3\right|\left|4-1\right|} = \frac{1}{3} \]
\end{solution}

\end{document}