\documentclass{article}

\usepackage{amsmath,amsfonts,amssymb,amsthm}
\usepackage{enumerate}
\usepackage{arydshln}
\usepackage{listings,color}
\usepackage{graphicx}

\definecolor{dkgreen}{rgb}{0,0.6,0}
\definecolor{gray}{rgb}{0.5,0.5,0.5}
\definecolor{mauve}{rgb}{0.58,0,0.82}

\title{Recurrence in the Left Shift}
\author{Neal Nesbitt}

\begin{document}
\maketitle

\theoremstyle{definition}
\newtheorem{definition}{Definition}[section]

\lstset{basicstyle=\ttfamily,
		language=Matlab,
		keywordstyle=\color{blue},
		commentstyle=\color{dkgreen},
		stringstyle=\color{mauve},
		identifierstyle=\bf,
		numbers=left,
		numberstyle=\color{gray}
		}
		
		
\section{Notation}

\paragraph{}
Consider some arbitrary system $\mathcal{A}^{\mathbb{Z}}$ with a dynamic $T:\mathcal{A}\to\mathcal{A}$ that causes $\mathcal{A}$ to take on a finite number of states $\mathcal{A}={a_{0},\cdots,a_{n}}$. We call these states our \underline{alphabet}, and say the dynamic represents ``motion" of the system from one state to another under this dynamic: $T(a^{\{k\}})=a^{\{k+1\}}$.

\begin{definition}{Alphabet of a system $\mathcal{A}^{\mathbb{Z}}$\\\\}
Let $\mathcal{A}^{\mathbb{Z}}$ have a range $\mathcal{A}$ that takes on finitely many states. We then call the collection of these states the system's \underline{alphabet}.
\end{definition}

\paragraph{}
Thus if we know the state of the system at some point $k\in\mathbb{Z},a^{\{k\}}\in\mathcal{A}$, we can mark prior and subsequent applications of the dynamic in a sequence $a \in \mathcal{A}^{\mathbb{Z}}$
\[ a = \{ \cdots,a^{k-1},a^{k},a^{k+1},\cdots \} \]
I call this the \underline{state sequence} of $a$ at $k$.

\begin{definition}{\underline{State Sequence} of an element in a dynamic system at some point\\}
content...
\end{definition}

\end{document}