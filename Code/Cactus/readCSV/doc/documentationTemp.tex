% *======================================================================*
%  Cactus Thorn template for ThornGuide documentation
%  Author: Ian Kelley
%  Date: Sun Jun 02, 2002
%  $Header$
%
%  Thorn documentation in the latex file doc/documentation.tex
%  will be included in ThornGuides built with the Cactus make system.
%  The scripts employed by the make system automatically include
%  pages about variables, parameters and scheduling parsed from the
%  relevant thorn CCL files.
%
%  This template contains guidelines which help to assure that your
%  documentation will be correctly added to ThornGuides. More
%  information is available in the Cactus UsersGuide.
%
%  Guidelines:
%   - Do not change anything before the line
%       % START CACTUS THORNGUIDE",
%     except for filling in the title, author, date, etc. fields.
%        - Each of these fields should only be on ONE line.
%        - Author names should be separated with a \\ or a comma.
%   - You can define your own macros, but they must appear after
%     the START CACTUS THORNGUIDE line, and must not redefine standard
%     latex commands.
%   - To avoid name clashes with other thorns, 'labels', 'citations',
%     'references', and 'image' names should conform to the following
%     convention:
%       ARRANGEMENT_THORN_LABEL
%     For example, an image wave.eps in the arrangement CactusWave and
%     thorn WaveToyC should be renamed to CactusWave_WaveToyC_wave.eps
%   - Graphics should only be included using the graphicx package.
%     More specifically, with the "\includegraphics" command.  Do
%     not specify any graphic file extensions in your .tex file. This
%     will allow us to create a PDF version of the ThornGuide
%     via pdflatex.
%   - References should be included with the latex "\bibitem" command.
%   - Use \begin{abstract}...\end{abstract} instead of \abstract{...}
%   - Do not use \appendix, instead include any appendices you need as
%     standard sections.
%   - For the benefit of our Perl scripts, and for future extensions,
%     please use simple latex.
%
% *======================================================================*
%
% Example of including a graphic image:
%    \begin{figure}[ht]
% 	\begin{center}
%    	   \includegraphics[width=6cm]{MyArrangement_MyThorn_MyFigure}
% 	\end{center}
% 	\caption{Illustration of this and that}
% 	\label{MyArrangement_MyThorn_MyLabel}
%    \end{figure}
%
% Example of using a label:
%   \label{MyArrangement_MyThorn_MyLabel}
%
% Example of a citation:
%    \cite{MyArrangement_MyThorn_Author99}
%
% Example of including a reference
%   \bibitem{MyArrangement_MyThorn_Author99}
%   {J. Author, {\em The Title of the Book, Journal, or periodical}, 1 (1999),
%   1--16. {\tt http://www.nowhere.com/}}
%
% *======================================================================*

% If you are using CVS use this line to give version information
% $Header$

\documentclass{article}

% Use the Cactus ThornGuide style file
% (Automatically used from Cactus distribution, if you have a
%  thorn without the Cactus Flesh download this from the Cactus
%  homepage at www.cactuscode.org)
\usepackage{cactus}

\begin{document}

% The author of the documentation
\author{Neal Nesbitt \textless NesbittN6830@uhcl.edu\textgreater}

% The title of the document (not necessarily the name of the Thorn)
\title{readCSV}

% the date your document was last changed, if your document is in CVS,
% please use:
%    \date{$ $Date$ $}
\date{May 05 2016}

\maketitle

% Do not delete next line
% START CACTUS THORNGUIDE

% Add all definitions used in this documentation here
%   \def\mydef etc

% Add an abstract for this thorn's documentation
\begin{abstract}
This function takes as a parameter the directory location of a comma separated value file and reads it into memory. It then calculates the maximum and minimum value for each column and displays it to screen.
\end{abstract}

\section{Introduction}

\paragraph{}
This is one of a suite of functions in development for implementing I/O with comma separated value (csv) files. They are intended to allow easy portability of scalar and vector points in the global computational grid into and out of human readable .csv files as used by programs such as Microsoft Excel and MATLAB.

\section{Physical System}

\paragraph{}
The points in question to be read or written are considered either scalar or vector quantities. Currently readCSV will read any csv with constant column length into memory, but it does not translate that into a grid variable. The next section of code to be added will compute the minimum spacing between each coordinate (column) of the points (rows) given, and use half of this value as the spacing for the computational grid. Currently it only computes the minimums and maximums for each coordinate, which will be used to specify the grid boundaries.

\section{Numerical Implementation}
The program is currently a single section of code meant to stream through the data and read the values as floating point numbers into memory. This is done character by character with flagged procedures for special delimiters. Because of this, the file can be as large as memory allows, both in the number of rows and columns. The only constraint is that each row must have the same number of columns.

\section{Using This Thorn}
The parameter file must contain the variable ``\verb|filePath|" that holds a string with the file path of the csv file to be read in. 

%\subsection{Obtaining This Thorn}

\subsection{Basic Usage}
Depending on the directory location of the csv file, the parameter file only needs the lines:\\
\\
\verb|ActiveThorns = "readCSV"|\\
\verb|readCSV::filePath = "/home/myUserName/csvFiles/myCSVFile.csv"|\\
\\
For this example my login name for the cluster is \verb|myUserName|, and the file \verb|myCSVFile.csv| is in a folder in my personal directory called \verb|csvFiles/|

\subsection{Special Behaviour}
The function should throw errors and abort if: 
\begin{itemize}
\item the file doesn't open or is corrupted,
\item the rows have different numbers of columns,
\item the entries don't represent numbers in floating point format,
\item or if any of the internal buffers fail to resize properly.
\end{itemize}

%\subsection{Interaction With Other Thorns}

%\subsection{Examples}

%\subsection{Support and Feedback}

%\section{History}

%\subsection{Thorn Source Code}

%\subsection{Thorn Documentation}

%\subsection{Acknowledgements}


%\begin{thebibliography}{9}

%\end{thebibliography}

% Do not delete next line
% END CACTUS THORNGUIDE

\end{document}
